 % !TEX encoding = UTF-8 Unicode
\documentclass[12pt,a4paper,oneside,fleqn,leqno]{article}
\usepackage[utf8]{inputenc}
%\usepackage[T1]{fontenc}
\usepackage{geometry}
\usepackage{graphicx}
\usepackage{amssymb}
\usepackage{amsmath}
\usepackage{wrapfig}
\usepackage{float}
\usepackage{caption}
\usepackage{subcaption}
\usepackage{multirow}
\usepackage{amsmath}
\usepackage{listings}
\usepackage{epstopdf}
\usepackage{url}
\usepackage{multirow}
\usepackage{amsthm}
\usepackage{longtable}
\usepackage{indentfirst}
\usepackage{array}
\usepackage{titletoc}
\usepackage{tocloft,calc}
%\usepackage{algorithm}
%\usepackage[center]{caption}
\geometry{a4paper}
\usepackage[russian]{babel}

\setlength{\topmargin}{-1.0in}
\setlength{\textheight}{10.5in}
\setlength{\oddsidemargin}{-.125in}
\setlength{\textwidth}{6.5in}

\newtheorem{theorem}{Теорема}[section]
\newtheorem{statement}{Утверждение}
\newtheorem{lemma}{Лемма}
\theoremstyle{definition}
\newtheorem{example}{Пример}[section]

\renewcommand{\thesubsubsection}{\S\arabic{subsection}.\arabic{subsubsection}}
\renewcommand{\thesubsection}{Глава \arabic{subsection}.}
%\renewcommand{\cftsubsecpresnum}{Глава }
\AtBeginDocument{\addtolength\cftsubsecnumwidth{\widthof{Глава }}}
%\titlecontents*{subsection}% <section-type>
%  [0pt]% <left>
%  {}% <above-code>
%  {\bfseries \thecontentslabel \quad}% <numbered-entry-format>
%  {}% <numberless-entry-format>
%  {\bfseries23\hfill42\contentspage}% <filler-page-format>

\newcolumntype{x}[1]{>{\centering\hspace{0pt}}p{#1}}

%\documentclass[12pt,fleqn]{article}
%\usepackage[cp1251]{inputenc}
\usepackage[T2A]{fontenc}
\usepackage{amssymb,amsmath,mathrsfs,amsthm}
\usepackage[russian]{babel}
\usepackage{graphicx}
%\usepackage[footnotesize]{caption2}
\usepackage{indentfirst}
%\usepackage[ruled,section]{algorithm}
\usepackage[ruled, linesnumbered]{algorithm2e}
\usepackage[noend]{algorithmic}
%\usepackage[all]{xy}

%\renewcommand{\algorithmicrequire}{\textbf{Вход:}}
%\renewcommand{\algorithmicensure}{\textbf{Выход:}}
%\renewcommand{\algorithmicfor}{\textbf{Для}}
\SetKwInOut{Parameter}{Параметры}
\SetKwInput{KwData}{Вход}
\SetKwInput{KwResult}{Выход}
\SetKwInput{KwIn}{Входные данные}
\SetKwInput{KwOut}{Выходные данные}
\SetKwIF{If}{ElseIf}{Else}{если}{то}{иначе\ если}{иначе}{}%{конец\ условия}
\SetKwFor{While}{До\ тех\ пор,\ пока}{выполнять}{}%{конец\ цикла}
\SetKw{KwTo}{от}
\SetKw{KwRet}{возвратить}
\SetKw{Return}{возвратить}
\SetKwBlock{Begin}{начало\ блока}{конец\ блока}
\SetKwSwitch{Switch}{Case}{Other}{Проверить\ значение}{и\ выполнить}{вариант}{в\ противном\ случае}{конец\ варианта}{конец\ проверки\ значений}
\SetKwFor{For}{Для}{выполнять}{}%{конец\ цикла}
\SetKwFor{ForEach}{для\ каждого}{выполнять}{конец\ цикла}
\SetKwRepeat{Repeat}{повторять}{до\ тех\ пор,\ пока}
\SetAlgorithmName{Алгоритм}{алгоритм}{Список алгоритмов}

% Параметры страницы
\textheight=24cm
\textwidth=16cm
\oddsidemargin=5mm
\evensidemargin=-5mm
\marginparwidth=36pt
\topmargin=-1cm
\footnotesep=3ex
%\flushbottom
\raggedbottom
\tolerance 3000
% подавить эффект "висячих стpок"
\clubpenalty=10000
\widowpenalty=10000
\renewcommand{\baselinestretch}{1.1}
%\renewcommand{\baselinestretch}{1.5} %для печати с большим интервалом

\begin{document}

\begin{titlepage}
\begin{center}
    Московский государственный университет имени М. В. Ломоносова

    \bigskip
    \includegraphics[width=50mm]{msu.eps}

    \bigskip
    Факультет Вычислительной Математики и Кибернетики\\
    Кафедра Математических Методов Прогнозирования\\[10mm]

    \textsf{\large\bfseries
        ДИПЛОМНАЯ РАБОТА СТУДЕНТА 517 ГРУППЫ\\[10mm]
        <<Построение дизъюнктивных нормальных форм для специальных классов булевых функций>>
    }\\[10mm]

    \begin{flushright}
        \parbox{0.5\textwidth}{
            Выполнил:\\
            студент 5 курса 517 группы\\
            \emph{Арбузова Дарья Андреевна}\\[5mm]
            Научный руководитель:\\
            д.ф-м.н., академик РАН\\
            \emph{Журавлёв Юрий Иванович}
        }
    \end{flushright}

    \begin{tabular}{p{0.45\textwidth}p{0.45\textwidth}}
        Заведующий кафедрой\newline
        Математических Методов\newline
        Прогнозирования, академик РАН
        &
        ~\newline~\newline
        \hfill\hbox to 0.45\textwidth{\hrulefill~Ю. И. Журавлёв}
    \\[20mm]
        К защите допускаю\newline
        \hbox to 0.4\textwidth{<<\hbox to 12mm{\hrulefill}>> \hrulefill~2015 г.}
        &
        К защите рекомендую\newline
        \hbox to 0.45\textwidth{<<\hbox to 12mm{\hrulefill}>> \hrulefill~2015 г.}
    \end{tabular}

    \vspace{\fill}
    Москва, 2015
\end{center}
\end{titlepage}

\newpage
\renewcommand{\contentsname}{Содержание}
\tableofcontents

\newpage
\begin{abstract}
    
\end{abstract}

\newpage
\section*{Введение}
\addcontentsline{toc}{section}{Введение}
	При минимизации булевых функций типичным является метод, когда на первом этапе строится сокращённая ДНФ, а затем из неё удаляются <<лишние>> конъюнкции. Под лишними понимаются такие конъюнкции, соответствующие интервалы которых покрываются интервалами других конъюнкций, входящих в ДНФ.\par
	Задача удаления или оставления конъюнкции может решаться прямым перебором по точкам испытуемой конъюнкции или с помощью критерия поглощения. Известно, что в общем случае проверка на то, равна ли функция тождественной единице, является сложной задачей. Однако при проверке поглощения мы рассматриваем относительно простой частный случай. При этом есть надежда, что иногда такую проверку можно осуществить достаточно быстро.\par
	Одна из возможностей состоит в применении проверки ДНФ с уже удалёнными буквами испытуемой конъюнкции на тождественное равенство единице. Для этого можно применить метод Блейка построения сокращённой ДНФ, и если рассматриваемая функция равна единице, то и её сокращённая ДНФ тождественно равна единице, что покажет метод Блейка. Если в процессе применения метода Блейка получится любая отличная от единицы ДНФ, то построенная функция не равна единице, и, следовательно, испытуемая конъюнкция не поглощается совокупностью конъюнкций, входящих в окрестность первого порядка (или их частью).\par
	В данной работе выделяются случаи, когда метод Блейка быстро приводит к ответу и, наоборот, когда применения этого метода связано с большими, невыполнимыми объёмами вычислений.
%Во введении рассказывается, где возникает данная задача, и~почему её решение так важно.
%Вводится на неформальном уровне минимум терминов, необходимый для понимания постановки задачи.
%Приводится краткий анализ источников информации (литературный обзор):
%как эту задачу решали до сих пор, в~чем недостаток этих решений, и~что нового предлагает автор.
%Формулируются цели исследования. 
%В~конце введения даётся краткое содержание работы по разделам; 
%при этом отмечается, какие подходы, методы, алгоритмы предлагаются автором впервые. 
%При~упоминании ключевых разделов кратко формулируются основные результаты и~наиболее важные выводы.

%Цель введения: дать достаточно полное представление о~выполненном исследовании 
%и~полученных результатах, понятное широкому кругу специалистов. 
%Большинство читателей прочтут именно введение и, быть может, заключение. 
%Во~введении автор решает сложную оптимизационную проблему: 
%как сообщить только самое важное, потратив минимум времени читателя,
%да~так, чтобы максимум читателей поняли, о~чём вообще идёт речь.

%Введение лучше писать напоследок, так как в~ходе работы обычно происходит переосмысление постановки задачи.
%Если же введение писать, когда работа еще не~готова, задача усложняется вдвойне.
%В~конце обычно приходит понимание, что всё получилось совсем не~так, как планировалось в~начале,
%и~исходный вариант введения всё равно придётся переписывать. 
%Кстати, к~таким «потерям» надо относиться спокойно~--- в~хорошей работе почти каждый абзац многократно переделывается до~неузнаваемости.
 
%Введение имеет много общего с~текстом доклада на~защите, поэтому имеет смысл готовить их одновременно.

			%\section{Определения и~обозначения}
%Формальная постановка задачи. 
%Для известных понятий желательно придерживаться стандартных обозначений. 
%Общепринятые термины вводятся словом «называется». 
%Термины, придуманные самим автором, вводятся словами «назовём» или «будем называть». 
%Обычно этот раздел заканчивается формальной постановкой задачи.
%Именно с~этого раздела стоит начинать писать работу.

\section{Построение дизъюнктивных нормальных форм}

%Лучше, чтобы название этого подраздела было содержательным, 
%например, общепринятым названием задачи, проблемы или метода,
%рассматриваемого в~данной работе. 
%Перечисляются подходы, методы, факты, на которые существенно опирается данная работа, 
%но~которые могут быть не~известны широкому кругу читателей.
%Здесь ссылки на литературу обязательны. 
%Теоремы только формулируются, но не~доказываются.
%Данный раздел преследует две цели. 
%Во-первых, сделать работу самодостаточной~--- дать необходимый минимум информации тем читателям,
%которые не~очень хорошо ориентируются в~теме, но желают поближе познакомиться именно с~данной работой.
%Во-вторых, облегчить сопоставление полученных автором результатов с~ранее известными.
	Пусть $n$ --- число переменных и булева функция $f(\tilde x^n)$ задана набором $k$ своих нулей:
	$$
		(\alpha_{i1},\ldots,\alpha_{in}),\quad i=\overline{1,\,k}.
	$$\par
	Известно, что такая функция представима в виде совершенной конъюнктивной формы следующим образом:
	$$
		f(\tilde x^n) = \bigwedge_{i = 1}^k (x_1^{\alpha_{i1}} \vee \ldots \vee x_n^{\alpha_{in}}).
	$$\par
	Для того, чтобы преобразовать это выражение в дизъюнктивную нормальную форму, необходимо раскрыть скобки, используя закон дистрибутивности, и применить правила идемпотентности ($x\wedge x = x$) и поглощения ($K_1 \vee K_1\wedge K_2 = K_1$). %(алгоритм Нельсона?).
Нетрудно видеть, что число образованных слагаемых до поглощений составляет $n^k.$ Оно быстро растёт с увеличением числа переменных, и даже при относительно малом числе нулей большое $n$ может создать вычислительно неподъёмную задачу.\par
	Возникает вопрос о нахождении <<непрямых>> методов получения ДНФ функции, заданной набором своих нулей.
	\subsection{Обзор существующих подходов}
		%Исследования в этом направлении проводились ещё в середине прошлого столетия.
		Задача реализации булевых функций с небольшим числом нулей возникает во многих прикладных задачах и потому активно исследовалась. Коснёмся проведённых исследований в этом направлении.
		\subsubsection{Построение ДНФ по формуле С.\,В. Яблонского}
			Рассматривается частный пример функции $f(\tilde x^n)$, которая обращается в ноль только на нулевом и единичном наборах: $f(0,\ldots,0) = f(1,\ldots,1) = 0.$\par
			Соответствующая ей КНФ имеет вид
			$$
				f(\tilde x^n) = (x_1 \vee x_2 \vee \ldots \vee x_n)\wedge(\bar{x}_1 \vee \bar{x}_2 \vee \ldots \vee \bar{x}_n).
			$$\par
			После прямого раскрытия скобок образуется $n^2 - n$ конъюнкций (за исключением противоречивых конъюнкций вида $x_i\wedge\bar{x}_i, i = \overline{1,\,n}$), однако С.\,В. Яблонский заметил, что существует короткая форма:
			$$
				f(\tilde x^n) = x_1\bar{x}_2 \vee x_2\bar{x}_3 \vee \ldots \vee x_n \bar{x}_1.
			$$\par
			Длина этой ДНФ равна $n.$
		\subsubsection{Метод Журавлёва -- Когана} \label{zhuravlev_kogan}
			Опишем алгоритм, предложенный в работе Ю.\,И. Журавлёва и А.\,Ю. Когана [?].\par
			Пусть задана матрица нулей функции $f:$
			$$
				M_{k \times n} = \{\bar{\alpha}_i = (\alpha_{i1},\ldots,\alpha_{in}) \}_{i = 1}^k.
			$$\par
			Будем считать, что
			\begin{enumerate}
				\item
				в матрице $M$ отсутствуют нулевые и единичные столбцы (поскольку в результирующую ДНФ сопоставленные им переменные в соответствующих степенях войдут в качестве конъюнкции ранга 1);
				\item
				в матрице $M$ присутствует не более одного из двойственных столбцов (т.е. таких, один из которых является отрицанием второго). Этого можно добиться заменой переменных
				$$
					x_i \rightarrow x_i^{\bar{\alpha}_{1i}},\,\,i = \overline{1,\,n},
				$$
				при этом первая строка матрицы $M$ окажется нулевой.
				\item
				столбцы матрицы $M$ упорядочены так, чтобы одинаковые столбцы образовывали последовательные блоки. Это может быть получено заменой
					$$
					x_i \rightarrow x_{\pi(i)},\,\,i = \overline{1,\,n},
				$$
				где $\pi$ --- перестановка на множестве из $n$ элементов.\\
				Пусть было образовано $m$ групп, в каждой по $n_i, i = \overline{1,\,m},$ столбцов.
			\end{enumerate}\par
			Составим новую матрицу $M'_{k \times m},$ в которую столбцы из каждой группы входят ровно 1 раз, и пусть она является матрицей нулей некоторой функции $\varphi(\tilde z^m).$ Пусть $D_{\varphi}$ --- ДНФ функции $\varphi$ в исходных переменных, что можно образовать c помощью замены
			$$
				z_i \rightarrow x_{n_1 + \ldots + n_{i - 1} + 1},\,\,i = \overline{1,\,m}.
			$$\par
			Тогда
			$$
				D_f = x_1\bar{x}_2 \vee x_2\bar{x}_3 \vee \ldots \vee x_{n_1}\bar{x}_1 \vee x_{n_1 + 1}\bar{x}_{n_1 + 2} \vee x_{n_1 + 2}\bar{x}_{n_1 + 3} \vee \ldots \vee x_{n_2}\bar{x}_{n_1 + 1} \vee \ldots
			$$
			$$
				\ldots \vee x_{n_{m - 1} + 1}\bar{x}_{n_{m - 1} + 2} \vee x_{n_{m - 1} + 2}\bar{x}_{n_{m - 1} + 3} \vee \ldots \vee x_{n}\bar{x}_{n_{m - 1} + 1} \vee D_{\varphi}.
			$$\par
			Её длина $L(D_f) = L(D_{\varphi}) + n.$\par
			Определение. Булева функция $g$ называется {\it приведённой}, если матрица $M_g$ её нулевых точек состоит из различных столбцов, исключая нулевой и единичный, причём из каждых двух двойственных столбцов в $M_g$ содержится ровно один.\par
			Таким образом выше описано построение приведённой функции $\varphi,$ соответствующей исходной функции $f.$
	\subsection{Метод разбиения на полосы}
		Пусть функция $f(\tilde x^n)$ задана своей матрицей нулей $M_{k \times n}$ и пусть требуется построить её ДНФ как можно меньшей длины. Предложим алгоритм, который заключается в построении КНФ этой функции, состоящей из скобок небольшой сложности, с последующим их раскрытием.\par
		Выберем некоторым образом параметр $p$ и поделим матрицу $M$ на $r = \left[\frac{k}{p}\right]$ полос по $p$ строк в каждой.\par
		Для каждой из подматриц $M_i, i = \overline{1,\,r},$ построим ДНФ $D_i$, соответствующую функции с такой матрицей нулей.\par
		Исходная функция $f$ представима в виде
		$$
			f(\tilde x^n) = \bigwedge_{i = 1}^r D_i,
		$$
		и её ДНФ $D$ может быть получена раскрытием скобок в данном выражении и применением правила поглощения.\par
		Видно, что чем меньше длина каждой ДНФ $D_i,$ тем меньше длина и искомой результирующей ДНФ $D.$\par
		Для определения $D_i$ воспользуемся результатами А.\,Г. Дьяконова [?] построения ДНФ булевых функций, матрица нулей которых содержит единичную подматрицу, о чём подробнее будет рассказано далее (\ref{dj}).
		\subsubsection{Выбор параметра $p$}
			Параметр $p$ определяет, по сколько строк содержат полосы, на которые разделяется исходная матрица нулей $M.$\par
			При $p = 1$ описанный выше алгоритм сведётся к построению ДНФ по каждому отдельному нулю функции $f$ и записи совершенной конъюнктивной нормальной формы. При $p = k$ мы ничем не упростили задачу построения ДНФ функции с $k$ нулями. Покажем, что выбором $1 \leqslant p \leqslant k$ можно добиться улучшения результата.\par
		\subsubsection{Построение ДНФ $D_i$} \label{dj}
			В статье [?] А.\,Г. Дьяконова описывается метод построения ДНФ булевой функции, заданной матрицей своих нулей, в которой существует единичная подматрица (с точностью до перестановки столбцов и инвертирования их значений).\par
			Опишем полученные А.\,Г. Дьяконовым результаты построения ДНФ приведённой функции. Связь ДНФ исходной функции с ДНФ соответствующей ей приведённой подробно описана в \ref{zhuravlev_kogan}.\par
			Напомним используемые в статье обозначения.\par
			Столбцу $(\alpha_{1j}, \ldots, \alpha_{kj})^{\text{T}}$ соответствует переменная $x_{z(j)}, z(j) = \sum\limits_{i = 1}^k 2^{i - 1}\alpha_{ij},$ переменной $x_t, t\in \{z(1),\ldots,z(n)\}$ --- столбец $(\alpha_{1y(t)}, \ldots, \alpha_{ky(t)})^{\text{T}}.$\par
			Множества индексов нулевых/ненулевых элементов столбцов:
			$$
				Z(t) = \{j | \alpha_{jy(t)} = 0\},\quad E(t) = \{j | \alpha_{jy(t)} = 1\}
			$$\par
			Множество индексов переменных, соответствующие столбцы которых не входят в единичную подматрицу:
			$$
				V = \{z(1), \ldots, z(n)\} \backslash \left\{2^i | i = \overline{0,\,k - 1}\right\}
			$$\par
			Разбиение множества столбцов длины $(k - 1)$: $\Psi = \{\psi(t) = (\alpha_{1t}, \ldots, \alpha_{(k - 1) t})^{\text{T}} | t \in V\} \subset E^{k - 1}\backslash E^{k - 1}_0\backslash E^{k - 1}_1 \backslash E^{k - 1}_{k - 1}$ на непересекающиеся цепи:
			$$
				V = \bigcup_{i = 1}^qB^i,\quad B^i\cap B^j = \varnothing,\,\,i \neq j,\,\,i,j \in \{1,\ldots,q\},\quad B^i = \{b^i_1,\ldots, b^i_{\beta(i)} \},
			$$
			при этом $(\psi(y(b^i_1)), \ldots, \psi(y(b^i_{\beta(i)}))$ --- цепь в $\Psi.$\par
			ДНФ приведённой функции:
			$$
				D^{\gamma} = \bigvee_{t  \in V} \left(x_t \wedge \bigwedge_{i \in E(t)} \bar{x}_{2^{i - 1}} \vee \bar{x}_t \wedge \bigwedge_{i \in Z(t)} \bar{x}_{2^{i - 1}}\right) \vee \bigvee_{1 \leqslant i < j \leqslant k} x_{2^{i - 1}} x_{2^{j - 1}}
			$$\par
			Её длина $L(D^{\gamma}) = 2n + \frac{1}{2}(k^2 - 5k).$\par
			
			$$
				D^{\Gamma} = \bigvee_{i = 1}^q \left(\bigvee_{j = 1}^{\beta(i) - 1} \bar{x}_{b_j^i} \bar{x}_{b_{j + 1}^i - b_j^i} \bar{x}_{b_{j + 1}^i} \vee x_{b_1^i} \wedge \bigwedge_{j \in E(b_1^i)}\bar{x}_{2^{j - 1}} \vee \bar{x}_{b^i_{\beta(i)}} \wedge \bigwedge_{j \in Z(b^i_{\beta(i)})}\bar{x}_{2^{j - 1}} \right)
				 \vee \bigvee_{1 \leqslant i < j \leqslant k} x_{2^{i - 1}} x_{2^{j - 1}}
			$$\par
			Её длина $L(D^{\Gamma}) = n + \frac{1}{2}(k^2 - 5k) + q.$\par
			Обе ДНФ $D^{\gamma}$ и $D^{\Gamma}$ реализуют приведённую функцию $\varphi$ и являются тупиковыми. $D^{\gamma}$ является частным случаем $D^{\Gamma},$ когда разбиение булева куба на цепи представляет собой набор цепей по одному элементу. Поэтому для $D^{\gamma}$ формула выписывается явно, а для $D^{\Gamma}$ необходимо произвести предварительное разбиение булева куба $E^{k - 1}$ на непересекающиеся цепи.\par
			По заданной матрице нулей $M_i$ ДНФ $D_i$ можно строить любым из представленных выше двух способов. 
		\subsubsection{Разбиение строк матрицы $M$}
			Исходя из того, что
			\begin{itemize}\itemsep=0pt
				\item
				в каждой образованной $M_i$ должна быть единичная подматрица,
				\item
					чем меньше различных столбцов в матрице $M_i,$ тем меньше длина образуемой ДНФ,
			\end{itemize}
			предложим следующий жадный алгоритм.\par
			Предполагая, что выбранный параметр $p$ достаточно мал, мы можем позволить себе вычислительный объём полного перебора подножеств строк (например, при $p = 5$ и $k = 25$ необходимо рассмотреть $C_{25}^5 = 53130$ вариантов).\par
			Для первой группы строк перебираются все подмножества по $p$ из $k$ строк исходной матрицы $M$ и выбирается такое, в котором присутствует единичная подматрица и содержится как можно больше одинаковых столбцов. Для второй группы процедура повторяется выбором подножества по $p$ из оставшихся $k - p$ строк и так далее.\par
			Если в какой-то момент оказалось, что ни одно подмножество строк не образует единичную подматрицу, то параметр $p$ уменьшается, т.е. последующие полосы будут иметь меньшую <<ширину>>, и процесс для них повторяется. Заметим, что уменьшением $p$ всегда можно добиться получения единичной матрицы в выбранной подматрице: для $p = 1$ это выполнено всегда.\par
		\begin{algorithm}[H]
			\SetAlgoLined
			\KwData{$M \in \{0,\,1\}^{k \times n}$ --- бинарная матрица}
			\Parameter{$p$ --- число строк в образованных подматрицах}
		 	\KwResult{$\{M_i\}_{i = 1}^{r}$ --- разбиение матрицы $M$ по строкам, где $r = \left [ \frac{k}{p}\right ]$}
			$Inds \mathbin{:=} \{1,\ldots,k\}$ --- множество ещё не использованных индексов строк матрицы $M$\;
			\For{$i = \overline{1,\,r}$}{
				\For{\textnormal{каждого подмножества $subset$ по $p$ индексов из $Inds$}}{
					\If{\textnormal{в матрице, образованной выбранными строками $subset$, существует единичная подматрица и число различных столбцов меньше, чем в текущем выбранном наборе $subset_i,$}}{
						$subset_i \mathbin{:=} subset$
					}
				}
				$M_i \mathbin{:=} M(subset_i)$\;
				$Inds \mathbin{:=} Inds \backslash subset_i$\;
			}
			\caption{Разбиение матрицы нулей $M$}
		\label{slice}
		\end{algorithm}\par
		
			Отсутствие единичной подматрицы размера $p \times p$ с точностью до перестановки столбцов и их инвертирования означает, что в хотя бы $n - p$ столбцах присутствует хотя бы 2 нуля и 2 единицы.\par
	\subsection{Построение тупиковой ДНФ}
		Известно, что любая тупиковая ДНФ, в частности, минимальная, может быть получена из сокращённой путём удаления некоторых импликант. В общем случае поиск сводится к полному перебору всех возможных комбинаций. Существуют различные способы повышения эффективности алгоритма синтеза минимальных ДНФ.\par
			%Рассмотрим один из подходов \emph{приближённого} решения описанной задачи.
		Пусть дана сокращённая ДНФ $D_{f}$ функции $f$:
		$$
			D_{f} = \bigvee_{i = 1}^{n} K_{i}
		$$ \par
		Определение минимальной ДНФ этой функции --- задача, по-видимому, $NP$-полная, поэтому предложим вариант построения некоторой тупиковой ДНФ $f$, которая во многих случаях, вероятно, достаточно близка по сложности к минимальной.
		\subsubsection{Проверка поглощения конъюнкции}
			Пусть для конъюнкции $K$ требуется проверить, поглощается ли она остальными конъюнкциями ДНФ $D.$ Рассмотрим прямой алгоритм поточечной проверки интервала, соответствующего конъюнкции $K.$\par
			Определим окрестность 1-го порядка $S_1(K)$ конъюнкции $K.$ Будем по очереди перебирать конъюнкции $K' \in S_1(K)$ и пересекать соответствующий им интервал $N_{K'}$ c интервалом $N_K.$ Если в результате все точки интервала $N_K$ оказались покрытыми, то $K$ поглощается конъюнкциями из своей окрестности 1-го порядка.\par
			Пусть $K = x_{i_1}^{\sigma_{i_1}} \ldots x_{i_k}^{\sigma_{i_k}}, \text{rg}K = k,$ и $K' = x_{j_1}^{\sigma_{j_1}} \ldots x_{j_m}^{\sigma_{j_m}}, \text{rg}K' = m.$ Определим конъюнкцию, соответствующую пересечению интервалов $N_K$ и $N_{K'}.$ Она будет иметь вид $K'' = x_{i_1}^{\sigma_{i_1}} \ldots x_{i_k}^{\sigma_{i_k}} x_{j_1}^{\sigma_{j_1}} \ldots x_{j_m}^{\sigma_{j_m}},$ при этом некоторые буквы, входящие в запись $K$ и $K'$ могут совпадать, т.е. $\max\{k, m\} \leqslant \text{rg}K'' \leqslant k + m.$ Тогда сложность построения интервала пересечения составляет $\underline{O}(k + m) = \underline{O}(n),$ где $n$ --- число переменных.\par
			Пусть $|S_1(K)| = t,$ тогда построение всех пересечений займёт $\underline{O}(tn)$ операций.\par
			Упорядочим полученные конъюнкции-пересечения по убыванию их размерности, таким образом, при проверке покрытия точек конъюнкции $K$ мы начнём с тех, которые накроют максимальное число точек. Поскольку ранги конъюнкций --- целые числа, непревосходящие $n,$ то для их упорядочения можно применить алгоритм сортировки подсчётом, временная сложность которого линейна $\underline{O}(t + n).$ При этом можно удалить повторы, которые в отсортированном списке конъюнкций будут идти подряд.\par
			Для каждой из $t$ конъюнкций $K''$ необходимо отметить покрываемые ею точки, число которых $2^{n - \text{rg}K''}.$\par
		\begin{algorithm}[H]
			\SetAlgoLined
			\KwData{$K$ --- испытуемая конъюнкция, $S_1(K)$ --- её окрестность 1-го порядка}
		 	\KwResult{ответ, покрывается ли $K$ её окрестностью 1-го порядка}
			$X \mathbin{:=} \varnothing$ --- множество конъюнкций, соответствующих пересечениям\;
			\For{\textnormal{каждой $K' \in S_1(K)$}}{
				$X \mathbin{:=} X \cup \{K \wedge K'\}$\;
			}
			Сортировка $X$ в соответствии с размерностями интервалов конъюнкций\;
			Интервал $N_K$ конъюнкции $K$ заполняется нулями\;
			$c \mathbin{:=} 0$ --- обнуляется счётчик покрытых точек\;
			\For{\textnormal{каждой $K'' \in X$}}{
				$c \mathbin{:=} c + 2^{\text{rg}K''} - \text{число отмеченных единиц в $N_{K''}$}$
				Подынтервал $N_{K''}$ интервала $N_K$ заполняется единицами\;
				\If{$c = 2^{\text{rg}K},$}{
						Возвращается ответ, что конъюнкция $K$ поглощается, и выход\;
					}
				}
			Возвращается ответ, что конъюнкция $K$ не поглощается, и выход\;
			\caption{Проверка факта поглощения конъюнкции}
			\label{check_cover}
		\end{algorithm}
		\subsubsection{Очередь проверки}
			При построении тупиковой ДНФ по данной сокращённой требуется вычеркнуть из неё некоторые конъюнкции так, чтобы оставшиеся интервалы образовывали неприводимое покрытие. Опишем метод, предложенный в предшествующей курсовой работе, который показал неплохие результаты.\par
			Итак, пусть $D_f = \bigvee\limits_{i = 1}^r K_i$ --- сокращённая ДНФ функции $f$. Будем по очереди выбрасывать из $D_f$ те конъюнкции, соответствующие интервалы которых покрываются минимальным числом других интервалов из своей окрестности 1-го порядка, пока не останется неприводимое покрытие.\par
			\begin{algorithm}[H]
				\SetAlgoLined
				\KwData{$D_f$ --- сокращённая ДНФ функции $f$}
		 		\KwResult{$D'$ --- тупиковая ДНФ, реализующая функцию $f$}
				$D' \mathbin{:=} D_f$ --- искомая ДНФ\;	
				\For{\textnormal{каждой $K \in D$}}{
					Определяем её окрестность 1-го порядка $S_1(K)$\;
				}
				\While{\textnormal{$\exists K^{*} \in D',$ которая поглощается конъюнкциями из своей окрестности 1-го порядка,}}{
					\For{\textnormal{каждой такой $K^{*}$}}{
						Вычисляем mincover($K^{*}$) --- минимальное число конъюнкций из $S_1(K^{*}),$ которыми поглощается $K^{*}$\;
						\If{\textnormal{$\text{mincover}(K^{*}) < \text{mincover}(K_{\min}),$}}{
							$K_{\min} \mathbin{:=} K^{*}$ --- обновляется текущая лучшая найденная конъюнкция\;
						}
					}
					\For{\textnormal{каждой $K \in S_1(K_{\min})$}}{
						$S_1(K) \mathbin{:=} S_1(K) \backslash K_{\min}$\;
					}
					$D' \mathbin{:=} D' \backslash K_{\min}$\;
				}
			\caption{Построение тупиковой ДНФ}
			\label{get_irred}
		\end{algorithm}\par
		Основная сложность заключается в определении mincover --- минимального числа конъюнкций, которыми покрывается какая-то данная. Решение подробно описано в курсовой работе [?].
	\newpage
	\subsection{Критерий поглощения с алгоритмом Блейка}
		В ходе применения критерия поглощения возникает, вообще говоря, NP-полная задача определить, является ли полученная ДНФ $D = \bigvee\limits_{i = 1}^rK_i$ тождественной единицей.\par
		Для выполнения этой проверки воспользуемся методом Блейка построения сокращённой ДНФ булевой функции, задаваемой $D.$\par
		Правило Блейка:
		$$
			Ax \vee B\bar{x} = Ax \vee B\bar{x} \vee AB
		$$
		$$
			x \vee A\bar{x} = x \vee A
		$$\par
		Если обобщить правило Блейка на случай, когда и $A$, и $B$ состоят из пустого множества букв, т.е. $A = B = 1,$ то по этому правилу можно было бы получить конъюнкцию, соответствующую 1:
		$$
			x \vee \bar{x} = 1\& x \vee 1\&\bar{x} = 1 \& x \vee 1\&\bar{x} \vee 1 \& 1 = x \vee \bar{x} \vee 1 = 1
		$$\par
		Соответствующий этой единичной конъюнкции интервал будет максимальным. Т.к. по теореме результат работы метода Блейка --- сокращённая ДНФ, то если изначальная $D \equiv 1$, мы получим сокращённую ДНФ из одной конъюнкции: 1.\par
		Однако если считать элементарную конъюнкцию из пустого множества букв недопустимой, то в результате такая конъюнкция образована не будет.
		\begin{example}
			$$
				D = x_3 \vee \bar{x}_1 \vee x_1x_2 \vee x_1\bar{x}_2\bar{x}_3
			$$\par
			Применим правило Блейка и простое поглощение:
			$$
				x_3 \vee \bar{x}_1 \vee x_1x_2 \vee x_1\bar{x}_2\bar{x}_3 \vee x_2 \vee \bar{x}_2\bar{x}_3 \vee x_1\bar{x}_2 \vee x_1 \vee \bar{x}_2 \vee \bar{x}_3 \vee x_1\bar{x}_3 =
			$$
			$$
				= x_1 \vee \bar{x}_1 \vee x_2 \vee \bar{x}_2 \vee x_3 \vee \bar{x}_3
			$$\par
			Для каждой переменной $x_i$ в сокращённую ДНФ вошло $x_i \vee \bar{x}_i.$
		\end{example}
		\begin{example}
			$$
				D = \bar{x}_1 \vee \bar{x}_2 \vee x_1\bar{x}_3 \vee x_1x_2
			$$\par
			Применим правило Блейка и простое поглощение:
			$$
				\bar{x}_1 \vee \bar{x}_2 \vee x_1\bar{x}_3 \vee x_1x_2 \vee \bar{x}_3 \vee x_1 \vee x_2 =
			$$
			$$
				= x_1 \vee \bar{x}_1  \vee x_2 \vee \bar{x}_2 \vee \bar{x}_3
			$$\par
			Не для всех $x_i$ в сокращённую ДНФ вошло выражение $x_i \vee \bar{x}_i.$
		\end{example}
		Если $D \equiv 1,$ то, вообще говоря, не для всех переменных $x_i$ из её формулы в сокращённой ДНФ будет присутствовать $x_i \vee \bar{x}_i$, это зависит от формы записи исходной ДНФ $D.$\par
		\begin{statement}
			Пусть задана ДНФ $D$ и в результате применения метода Блейка получена ДНФ $D'.$ $D \equiv 1$ тогда и только тогда, когда $\exists i: x_i \vee \bar{x}_i$ входит в $D'.$
		\end{statement}
		\begin{proof}
			\textit{Необходимость.} Рассмотрим функцию $f(\tilde x^{n+1}) = x_{n + 1} \& D.$ Т.к. $D \equiv 1,$ то сокращённая ДНФ $D_f^{\text{сокр}} = x_{n + 1}.$\par
			По теореме применение метода Блейка к ДНФ $\bigvee\limits_{i = 1}^r x_{n + 1} \& K_i$ функции $f$ даст единственную конъюнкцию $x_{n + 1},$ которая могла быть получена из преобразования
			$$
				x_{n + 1} x_i \vee x_{n + 1} \bar{x}_i = x_{n + 1} x_i \vee x_{n + 1} \bar{x}_i \vee x_{n + 1}
			$$\par
			(При этом правило $x_i \vee x_{n + 1} \bar{x}_i = x_i \vee x_{n + 1}$ не могло быть применено, т.к. все конъюнкции на любом этапе содержат $x_{n + 1},$ а $\bar{x}_{n + 1}$ нигде не встречается и образовано быть не может.)\par
			Значит, данные $x_i$ и $\bar{x}_i$ встретятся в $D'.$\\
		\textit{Достаточность.} Т.к. $x_i \vee \bar{x}_i \equiv 1$, то $D' \equiv 1$. ДНФ $D'$ и $D$ задают одну и ту же функцию, поэтому из того, что $D' \equiv 1$, следует, что и $D \equiv 1.$
		\end{proof}\par
		Итак, пусть в ходе применения критерия поглощения образована ДНФ $D = \bigvee\limits_iK_i.$\par
		Будем исследовать поведение метода Блейка, применённого к полученной ДНФ, в зависимости от числа различных переменных, входящих в её запись в обеих степенях.\par
		\subsubsection{Тривиальный случай}
			Пусть все переменные входят в запись $D$ в одинаковых степенях.\par
			\begin{lemma}
				Пусть все переменные, входящие в любую из конъюнкций $K_i$, имеют степень $\sigma \in \{ 0, 1\}.$ Тогда $D \not\equiv 1$.
			\end{lemma}
			\begin{proof}
				Рассмотрим набор $(1 - \sigma, \ldots, 1 - \sigma).$ Функция, задаваемая ДНФ $D,$ принимает на нём значение 0, т.к. $(1 - \sigma)^{\sigma} = 0.$ Значит, существует точка, не покрываемая $D,$ и $D \not\equiv 1$.
			\end{proof}
		\subsubsection{Случай одной переменной}
			Пусть существует единственная переменная $x$, которая встречается в записи $D$ в обеих степенях: $x$ и $\bar{x}$.\par
			Пусть $x$ встречается в $k_1$ конъюнкциях, а $\bar{x}$ --- в $k_0.$ В записи одной и той же конъюнкции и $x$, и $\bar{x}$ встречаться не могут, поскольку иначе она была бы противоречивой.\par
			Правило Блейка применимо $k_1\cdot k_0$ раз, по числу различных пар конъюнкций вида $xA \vee \bar{x}B.$ Образованные в ходе метода Блейка конъюнкции имеют вид $AB$ и содержать ни одну из $x$ и $\bar{x}$ не могут.\par
			Таким образом, будет образовано не более $k_1\cdot k_0$ новых конъюнкций, и если длина исходной ДНФ равна $k$, то получаем оценку $O(k^2).$

\newpage
\subsection*{Условия на образование единичной конъюнкции}
		Пусть заданы $n$ --- число переменных, конъюнкция $K = x_1\ldots x_q$ ранга $q$ (с помощью замены переменных всегда можно добиться представления в таком виде) и ДНФ $\bigvee\limits_iK'_i.$\par
		В ходе применения критерия поглощения после удаления букв, входящих в конъюнкцию $K$, образована ДНФ $D = \bigvee\limits_{i = 1}^rK_i,$ в которую входят $2n - q$ различных букв: $\bar{x}_1,\ldots,\bar{x}_n, x_{q + 1},\ldots, x_n$.\par
		В ходе метода Блейка для ДНФ $D$ правило Блейка может быть применено к парам конъюнкций, содержащим только переменные $x_{q+1}$ и $\bar{x}_{q+1}$, $\ldots$, $x_{n}$ и $\bar{x}_{n},$ остальные же могут встречаться в $D$ только в одной степени. Т.е. не более $n - q$ переменных встречаются в $D$ в обоих степенях.\par
		Пусть $q = n - 2$. В этом случае правило Блейка применимо к конъюнкциям $x_{n - 1}A \vee \bar{x}_{n - 1}B$ и $x_{n}A \vee \bar{x}_{n}B.$ Новые образованные конъюнкции содержат только отрицания переменных $x_1,\ldots,x_q,$ и тождественная единица может быть получена только в виде $x_{n - 1} \vee \bar{x}_{n - 1}$ или $x_{n} \vee \bar{x}_{n}.$\par
		Выпишем некоторые достаточные условия образования пары конъюнкций $x_{n - 1} \vee \bar{x}_{n - 1}.$\par
		Напомним общий вид $D$:
		$$
				\bigvee_ix_{n - 1}A_i \vee \bigvee_i\bar{x}_{n - 1}B_i \vee \bigvee_ix_nC_i \vee \bigvee_i\bar{x}_nD_i \vee \ldots
		$$
		$$
		\ldots \vee \bigvee_ix_{n - 1}x_nE_i \vee \bigvee_ix_{n - 1}\bar{x}_nF_i \vee \bigvee_i\bar{x}_{n - 1}x_nG_i \vee \bigvee_i\bar{x}_{n - 1}\bar{x}_nH_i \vee \bigvee_iI_i
		$$
		\begin{minipage}[t]{0.5\textwidth}
		Условия для образования конъюнкции $x_{n - 1}$:
		\begin{enumerate}
			\item
			$\exists A_i = 1$
			\item
			$\exists C_i = 1$ и $\exists F_i = 1$
			\item
			$\exists D_i = 1$ и $\exists E_i = 1$
			\item
			$\exists E_i = 1$ и $\exists F_i = 1$\vspace{10pt}
		\end{enumerate}
		\end{minipage}
		\hfill
		\begin{minipage}[t]{0.4\textwidth}
		Аналогично для $\bar{x}_{n - 1}$:
		\begin{enumerate}
			\item
			$\exists B_i = 1$
			\item
			$\exists C_i = 1$ и $\exists H_i = 1$
			\item
			$\exists D_i = 1$ и $\exists G_i = 1$
			\item
			$\exists G_i = 1$ и $\exists H_i = 1$
		\end{enumerate}
		\end{minipage}
		Остальные условия (т.е. случай образования $x_{n - 1}$ или $\bar{x}_{n - 1}$ после цепочки из 2-х применений правила Блейка) избыточны, т.к. включают в себя хотя бы одно из перечисленных выше.\par
		Таким образом достаточно провести только первые 2 этапа метода Блейка, чтобы понять, будет ли образована единичная конъюнкция.
\newpage

			%\section{Новые подходы и~результаты}
%Название этого раздела обязательно надо заменить на~содержательное. 
%В~этом разделе, как правило, много подразделов. 

%В~дипломной работе не~стоит делать более двух уровней, достаточно разделов и~подразделов.
%Будете писать диссертацию или монографию~--- сделаете три уровня. 
  
			%\section{Вычислительные эксперименты}

%Цель данного раздела: продемонстрировать, что предложенная теория работает на практике; показать границы её применимости; рассказать о~новых экспериментальных фактах.
%Чисто теоретические работы могут вообще не~содержать раздела экспериментов (не~работает, ну и~не~надо~--- зато теория красивая).
%Кстати, теоретики имеют право не~догадываться, где, кому и~когда их теории пригодятся.

			%\subsection{Исходные данные и~условия эксперимента}
%Описывается прикладная задача, параметры анализируемых данных (например, сколько объектов, сколько признаков, каких они типов),  параметры эксперимента  (например, как производился скользящий контроль). 

			%\subsection{Результаты эксперимента}
%Результаты экспериментов представляются в~виде таблиц и~графиков. 
%Объясняется точный смысл всех обозначений на графиках, строк и~столбцов в~таблицах. 

			%\subsection{Обсуждение и~выводы}
%Приводятся выводы: 
%в~какой степени результаты экспериментов согласуются с~теорией? 
%Достигнут ли желаемый результат? 
%Обнаружены ли какие-либо факты, не~нашедшие объяснения, и~которые нельзя списать на «грязный» эксперимент?

%Обсуждаются основные отличия предложенных методов от известных ранее. 
%В~чем их преимущества? 
%Каковы границы их применимости? 
%Какие проблемы удалось решить, а~какие остались открытыми? 
%Какие возникли новые постановки задач?

\section*{Заключение}
\addcontentsline{toc}{section}{Заключение}

В~квалификационных работах последний раздел нужен для того, чтобы 
конспективно перечислить основные результаты, полученные лично автором. 

Результатами, в~частности, являются:
\begin{itemize}
\item 
    Предложен новый подход к\dots
\item 
    Разработан новый метод\dots, позволяющий\dots
\item 
    Доказан ряд теорем, подтверждающих (опровергающих), что\dots
\item 
    Проведены вычислительные эксперименты\dots,
    которые подтвердили / опровергли / привели к~новым постановкам задач.\cite{zhuravlev99pria-eng}
\end{itemize}
    
%Цель данного раздела: доказать квалификацию автора.  Даже беглого взгляда на заключение должно быть достаточно, чтобы стало ясно: автору удалось решить актуальную, трудную, ранее не~решённую задачу, предложенные автором решения обоснованы и~проверены.

%Иногда в~Заключении приводится список направлений дальнейших исследований.

\newpage

\renewcommand{\bibname}{Список литературы}
\addcontentsline{toc}{section}{\bibname}

\nocite{hastie09elements,bishop06pattern,zhuravlev06recognition,zhuravlev78prob33,shlezinger04ten,boucheron05theory}

\def\BibUrl#1.{}\def\BibAnnote#1.{}
%\def\BibUrl#1{\\{\footnotesize\tt\def~{\char126} http://#1}}
\bibliographystyle{gost71s}
\bibliography{MachLearn}

\begin{thebibliography}{99}

\bibitem{Marsaglia03}
  George Marsaglia. 2003. Xorshift RNGs. Journal of Statistical Software 8, 14 (2003), 1--6.

\bibitem{Vigna14}
  Sebastiano Vigna. 2014. Further scramblings of Marsaglia's xorshift generators. CoRR, abs/1404.0390.

\bibitem{TestU01}
  P. L'Ecuyer and R. Simard. 2007. TestU01: A C Library for Empirical Testing of Random Number Generators ACM Transactions on Mathematical Software, Vol. 33, article 22.

\end{thebibliography}

\end{document}
