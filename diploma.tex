 % !TEX encoding = UTF-8 Unicode
\documentclass[12pt,a4paper,oneside,fleqn,leqno]{article}
\usepackage[utf8]{inputenc}
%\usepackage[T1]{fontenc}
\usepackage{geometry}
\usepackage{graphicx}
\usepackage{amssymb}
\usepackage{amsmath}
\usepackage{wrapfig}
\usepackage{float}
\usepackage{caption}
\usepackage{subcaption}
\usepackage{multirow}
\usepackage{amsmath}
\usepackage{listings}
\usepackage{epstopdf}
\usepackage{url}
\usepackage{multirow}
\usepackage{amsthm}
\usepackage{longtable}
\usepackage{indentfirst}
\usepackage{array}
\usepackage{titletoc}
\usepackage{tocloft,calc}
\usepackage{tikz}
\usetikzlibrary{arrows,matrix,positioning}
\usetikzlibrary{matrix}
%\usepackage{algorithm}
%\usepackage[center]{caption}
\geometry{a4paper}
\usepackage[russian]{babel}

\setlength{\topmargin}{-1.0in}
\setlength{\textheight}{10.5in}
\setlength{\oddsidemargin}{-.125in}
\setlength{\textwidth}{6.5in}

\newtheorem{theorem}{Теорема}[section]
\newtheorem{statement}{Утверждение}
\newtheorem{lemma}{Лемма}
\theoremstyle{definition}
\newtheorem{example}{Пример}%[section] Figure a way for proper example numbering

\renewcommand{\thesubsection}{\S\arabic{section}.\arabic{subsection}}
\renewcommand{\thesection}{Глава \arabic{section}.}
%\renewcommand{\cftsubsecpresnum}{Глава }
\AtBeginDocument{\addtolength\cftsecnumwidth{\widthof{Глава }}}
%\titlecontents*{subsection}% <section-type>
%  [0pt]% <left>
%  {}% <above-code>
%  {\bfseries \thecontentslabel \quad}% <numbered-entry-format>
%  {}% <numberless-entry-format>
%  {\bfseries23\hfill42\contentspage}% <filler-page-format>

\newcolumntype{x}[1]{>{\centering\hspace{0pt}}p{#1}}

\newcommand\SmallMatrix[1]{{%
  \small\arraycolsep=0.6\arraycolsep\ensuremath{\begin{matrix}#1\end{matrix}}}}

\def\VR{\kern-\arraycolsep\strut\vrule &\kern-\arraycolsep}
\def\vr{\kern-\arraycolsep & \kern-\arraycolsep}

\setcounter{MaxMatrixCols}{20}

%\documentclass[12pt,fleqn]{article}
%\usepackage[cp1251]{inputenc}
\usepackage[T2A]{fontenc}
\usepackage{amssymb,amsmath,mathrsfs,amsthm}
\usepackage[russian]{babel}
\usepackage{graphicx}
%\usepackage[footnotesize]{caption2}
\usepackage{indentfirst}
%\usepackage[ruled,section]{algorithm}
\usepackage[ruled, linesnumbered]{algorithm2e}
\usepackage[noend]{algorithmic}
%\usepackage[all]{xy}

%\renewcommand{\algorithmicrequire}{\textbf{Вход:}}
%\renewcommand{\algorithmicensure}{\textbf{Выход:}}
%\renewcommand{\algorithmicfor}{\textbf{Для}}
\SetKwInOut{Parameter}{Параметры}
\SetKwInput{KwData}{Вход}
\SetKwInput{KwResult}{Выход}
\SetKwInput{KwIn}{Входные данные}
\SetKwInput{KwOut}{Выходные данные}
\SetKwIF{If}{ElseIf}{Else}{Если}{то}{иначе\ если}{иначе}{}%{конец\ условия}
\SetKwFor{While}{До\ тех\ пор,\ пока}{выполнять}{}%{конец\ цикла}
\SetKw{KwTo}{от}
\SetKw{KwRet}{возвратить}
\SetKw{Return}{возвратить}
\SetKwBlock{Begin}{начало\ блока}{конец\ блока}
\SetKwSwitch{Switch}{Case}{Other}{Проверить\ значение}{и\ выполнить}{вариант}{в\ противном\ случае}{конец\ варианта}{конец\ проверки\ значений}
\SetKwFor{For}{Для}{выполнять}{}%{конец\ цикла}
\SetKwFor{ForEach}{для\ каждого}{выполнять}{конец\ цикла}
\SetKwRepeat{Repeat}{повторять}{до\ тех\ пор,\ пока}
\SetAlgorithmName{Алгоритм}{алгоритм}{Список алгоритмов}

% Параметры страницы
%\textheight=24cm
%\textwidth=16cm
%\oddsidemargin=5mm
%\evensidemargin=-5mm
%\marginparwidth=36pt
%\topmargin=-1cm
%\footnotesep=3ex
%\flushbottom
\raggedbottom
\tolerance 3000
% подавить эффект "висячих стpок"
\clubpenalty=10000
\widowpenalty=10000
\renewcommand{\baselinestretch}{1.1}
\renewcommand{\baselinestretch}{1.5} %для печати с большим интервалом

\begin{document}

\begin{titlepage}
\begin{center}
    Московский государственный университет имени М. В. Ломоносова

    \bigskip
    \includegraphics[width=50mm]{msu.eps}

    \bigskip
    Факультет Вычислительной Математики и Кибернетики\\
    Кафедра Математических Методов Прогнозирования\\[10mm]

    \textsf{\large\bfseries
        ДИПЛОМНАЯ РАБОТА СТУДЕНТА 517 ГРУППЫ\\[10mm]
        <<Построение дизъюнктивных нормальных форм для специальных классов булевых функций>>
    }\\[10mm]

    \begin{flushright}
        \parbox{0.5\textwidth}{
            Выполнил:\\
            студент 5 курса 517 группы\\
            \emph{Арбузова Дарья Андреевна}\\[5mm]
            Научный руководитель:\\
            д.ф-м.н., академик РАН\\
            \emph{Журавлёв Юрий Иванович}
        }
    \end{flushright}

    \begin{tabular}{p{0.45\textwidth}p{0.45\textwidth}}
        Заведующий кафедрой\newline
        Математических Методов\newline
        Прогнозирования, академик РАН
        &
        ~\newline~\newline
        \hfill\hbox to 0.45\textwidth{\hrulefill~Ю. И. Журавлёв}
    \\[20mm]
        К защите допускаю\newline
        \hbox to 0.4\textwidth{<<\hbox to 12mm{\hrulefill}>> \hrulefill~2015 г.}
        &
        К защите рекомендую\newline
        \hbox to 0.45\textwidth{<<\hbox to 12mm{\hrulefill}>> \hrulefill~2015 г.}
    \end{tabular}

    \vspace{\fill}
    Москва, 2015
\end{center}
\end{titlepage}

\newpage
\renewcommand{\contentsname}{Содержание}
\tableofcontents

%\newpage
%\begin{abstract}    
%\end{abstract}

\newpage
\section*{Введение}
\addcontentsline{toc}{section}{Введение}
	В работе изучаются методы построения дизъюнктивных нормальных форм булевых функций и изучаются проблемы, связанные с трудностью построения достаточно простых ДНФ. Алгоритм построения таких ДНФ принято называть алгоритмом минимизации булевых функций. При этом не требуется, как правило, построения формулы наименьшей сложности, что, по-видимому, невыполнимо в классе достаточно простых алгоритмов [4]. \par
	Весь класс алгоритмов для построения таких форм условно называется проблемой минимизации булевых функций. При такой минимизации типичным является метод, когда на первом этапе строится сокращённая ДНФ, а затем из неё удаляются <<лишние>> конъюнкции. Под лишними понимаются такие конъюнкции, соответствующие интервалы которых покрываются интервалами других конъюнкций, входящих в ДНФ.\par
	Задача удаления или оставления конъюнкции может решаться прямым перебором по <<точкам испытуемой конъюнкции>> или с помощью критерия поглощения. Известно, что в общем случае проверка на то, равна ли функция тождественной единице, является сложной задачей. Однако при проверке поглощения мы рассматриваем относительно простой частный случай. При этом есть надежда, что иногда такую проверку можно осуществить достаточно быстро.\par
	Одна из возможностей состоит в применении проверки ДНФ с уже удалёнными буквами испытуемой конъюнкции на тождественное равенство единице. Для этого можно применить метод Блейка построения сокращённой ДНФ, и если рассматриваемая функция равна единице, то и её сокращённая ДНФ тождественно равна единице, что покажет метод Блейка. Если в процессе применения метода Блейка получится любая отличная от единицы ДНФ, то построенная функция не равна единице, и, следовательно, испытуемая конъюнкция не поглощается совокупностью конъюнкций, входящих в окрестность первого порядка (или их частью).\par
	В данной работе выделяются случаи, когда метод Блейка быстро приводит к ответу и, наоборот, когда применения этого метода связано с большими, невыполнимыми объёмами вычислений.\par
	Предлагается синтез, обеспечивающий реальную применимость при решении прикладных задач. В работе последовательно конструируются и изучаются методы построения <<относительно несложной>> ДНФ, а затем решается задача, кратко сформулированная выше.\par
%Во введении рассказывается, где возникает данная задача, и~почему её решение так важно.
%Вводится на неформальном уровне минимум терминов, необходимый для понимания постановки задачи.
%Приводится краткий анализ источников информации (литературный обзор):
%как эту задачу решали до сих пор, в~чем недостаток этих решений, и~что нового предлагает автор.
%Формулируются цели исследования. 
%В~конце введения даётся краткое содержание работы по разделам; 
%при этом отмечается, какие подходы, методы, алгоритмы предлагаются автором впервые. 
%При~упоминании ключевых разделов кратко формулируются основные результаты и~наиболее важные выводы.

%Цель введения: дать достаточно полное представление о~выполненном исследовании 
%и~полученных результатах, понятное широкому кругу специалистов. 
%Большинство читателей прочтут именно введение и, быть может, заключение. 
%Во~введении автор решает сложную оптимизационную проблему: 
%как сообщить только самое важное, потратив минимум времени читателя,
%да~так, чтобы максимум читателей поняли, о~чём вообще идёт речь.

%Введение лучше писать напоследок, так как в~ходе работы обычно происходит переосмысление постановки задачи.
%Если же введение писать, когда работа еще не~готова, задача усложняется вдвойне.
%В~конце обычно приходит понимание, что всё получилось совсем не~так, как планировалось в~начале,
%и~исходный вариант введения всё равно придётся переписывать. 
%Кстати, к~таким «потерям» надо относиться спокойно~--- в~хорошей работе почти каждый абзац многократно переделывается до~неузнаваемости.
 
%Введение имеет много общего с~текстом доклада на~защите, поэтому имеет смысл готовить их одновременно.

			%\section{Определения и~обозначения}
%Формальная постановка задачи. 
%Для известных понятий желательно придерживаться стандартных обозначений. 
%Общепринятые термины вводятся словом «называется». 
%Термины, придуманные самим автором, вводятся словами «назовём» или «будем называть». 
%Обычно этот раздел заканчивается формальной постановкой задачи.
%Именно с~этого раздела стоит начинать писать работу.

\newpage
\section*{Построение дизъюнктивных нормальных форм}
\addcontentsline{toc}{section}{Построение дизъюнктивных нормальных форм}

%Лучше, чтобы название этого подраздела было содержательным, 
%например, общепринятым названием задачи, проблемы или метода,
%рассматриваемого в~данной работе. 
%Перечисляются подходы, методы, факты, на которые существенно опирается данная работа, 
%но~которые могут быть не~известны широкому кругу читателей.
%Здесь ссылки на литературу обязательны. 
%Теоремы только формулируются, но не~доказываются.
%Данный раздел преследует две цели. 
%Во-первых, сделать работу самодостаточной~--- дать необходимый минимум информации тем читателям,
%которые не~очень хорошо ориентируются в~теме, но желают поближе познакомиться именно с~данной работой.
%Во-вторых, облегчить сопоставление полученных автором результатов с~ранее известными.
	Пусть $n$ --- число переменных и булева функция $f(\tilde x^n)$ задана набором $k$ своих нулей:
	$$
		(\alpha_{i1},\ldots,\alpha_{in}),\quad i=\overline{1,\,k}.
	$$\par
	Известно, что такая функция представима в виде совершенной конъюнктивной формы следующим образом:
	$$
		f(\tilde x^n) = \bigwedge_{i = 1}^k (x_1^{\alpha_{i1}} \vee \ldots \vee x_n^{\alpha_{in}}).
	$$\par
	Для того, чтобы преобразовать это выражение в дизъюнктивную нормальную форму, необходимо раскрыть скобки, используя закон дистрибутивности, и применить правила идемпотентности ($x\wedge x = x$) и поглощения ($K_1 \vee K_1K_2 = K_1$). %(алгоритм Нельсона?).
Нетрудно видеть, что число образованных слагаемых до поглощений составляет $n^k.$ Оно быстро растёт с увеличением числа переменных, и даже при относительно малом числе нулей большое $n$ может создать практически неразрешимую задачу.\par
	Возникает вопрос о нахождении <<непрямых>> методов получения ДНФ функции, заданной набором своих нулей.
	%\newpage
	\section{Обзор существующих подходов}
		%Исследования в этом направлении проводились ещё в середине прошлого столетия.
		Задача реализации булевых функций с небольшим числом нулей возникает во многих прикладных задачах и потому активно исследовалась. Коснёмся проведённых исследований в этом направлении.
		\subsection{Построение ДНФ по формуле С.\,В. Яблонского}\label{jab}
			Рассматривается частный пример функции $f(\tilde x^n)$, которая обращается в ноль только на нулевом и единичном наборах: $f(0,\ldots,0) = f(1,\ldots,1) = 0.$\par
			Соответствующая ей КНФ имеет вид
			$$
				f(\tilde x^n) = (x_1 \vee x_2 \vee \ldots \vee x_n)\wedge(\bar{x}_1 \vee \bar{x}_2 \vee \ldots \vee \bar{x}_n).
			$$\par
			После прямого раскрытия скобок образуется $n^2 - n$ конъюнкций (за исключением противоречивых конъюнкций вида $x_i\wedge\bar{x}_i,\,\,i = \overline{1,\,n}$), однако С.\,В. Яблонский заметил, что существует короткая форма:
			$$
				f(\tilde x^n) = x_1\bar{x}_2 \vee x_2\bar{x}_3 \vee \ldots \vee x_n \bar{x}_1.
			$$\par
			Длина этой ДНФ равна $n.$
		\subsection{Метод Журавлёва -- Когана} \label{zhuravlev_kogan}
			Опишем алгоритм, предложенный в работе Ю.\,И. Журавлёва и А.\,Ю. Когана [2].\par
			Пусть задана матрица нулей функции $f:$
			$$
				M_{k \times n} = \{\bar{\alpha}_i = (\alpha_{i1},\ldots,\alpha_{in}) \}_{i = 1}^k.
			$$\par
			Будем считать, что
			\begin{enumerate}
				\item
				в матрице $M$ отсутствуют нулевые и единичные столбцы (поскольку в результирующую ДНФ сопоставленные им переменные в соответствующих степенях войдут в качестве конъюнкции ранга 1);
				\item
				в матрице $M$ присутствует не более одного из двойственных столбцов (т.е. таких, один из которых является отрицанием второго). Этого можно добиться заменой переменных
				$$
					x_i \rightarrow x_i^{\bar{\alpha}_{1i}},\,\,i = \overline{1,\,n},
				$$
				при этом первая строка матрицы $M$ окажется нулевой.
				\item
				столбцы матрицы $M$ упорядочены так, чтобы одинаковые столбцы образовывали последовательные блоки. Это может быть получено заменой
					$$
					x_i \rightarrow x_{\pi(i)},\,\,i = \overline{1,\,n},
				$$
				где $\pi$ --- перестановка на множестве из $n$ элементов.\\
				Пусть было образовано $m$ групп, в каждой по $n_i,\,\,i = \overline{1,\,m},$ столбцов.
			\end{enumerate}\par
			Составим новую матрицу $M'_{k \times m},$ в которую столбцы из каждой группы входят ровно один раз, и пусть она является матрицей нулей некоторой функции $\varphi(\tilde z^m).$ Пусть $\mathcal{D}_{\varphi}$ --- ДНФ функции $\varphi$ в исходных переменных, что можно образовать c помощью замены
			$$
				z_i \rightarrow x_{n_1 + \ldots + n_{i - 1} + 1},\,\,i = \overline{1,\,m}.
			$$\par
			Тогда
			$$
				\mathcal{D}_f = x_1\bar{x}_2 \vee x_2\bar{x}_3 \vee \ldots \vee x_{n_1}\bar{x}_1 \vee x_{n_1 + 1}\bar{x}_{n_1 + 2} \vee x_{n_1 + 2}\bar{x}_{n_1 + 3} \vee \ldots \vee x_{n_2}\bar{x}_{n_1 + 1} \vee \ldots
			$$
			$$
				\ldots \vee x_{n_{m - 1} + 1}\bar{x}_{n_{m - 1} + 2} \vee x_{n_{m - 1} + 2}\bar{x}_{n_{m - 1} + 3} \vee \ldots \vee x_{n}\bar{x}_{n_{m - 1} + 1} \vee \mathcal{D}_{\varphi}.
			$$\par
			Её длина $L(\mathcal{D}_f) = L(\mathcal{D}_{\varphi}) + \sum\limits_{i = 1}^m n_i[n_i > 1].$\par
			\underline{Определение}. Булева функция $g$ называется {\it приведённой}, если матрица $M_g$ её нулевых точек состоит из различных столбцов, исключая нулевой и единичный, причём из каждых двух двойственных столбцов в $M_g$ содержится ровно один.\par
			Таким образом выше описано построение приведённой функции $\varphi,$ соответствующей исходной функции $f.$
			\begin{example}
				Пусть задана матрица нулей функции $f(\tilde x^9)$:
				$$
					\bordermatrix{
						& x_1& x_2 & x_3 & x_4 & x_5 & x_6 & x_7 & x_8 & x_9\cr
						&1 &0 &1 &0& 1& 1& 0& 1& 1\cr
						&1 &1& 0& 0& 1& 0& 1& 1& 0\cr
						&0 &1 &1& 0& 1& 1& 1& 1& 0\cr
						&0 &0 &0 &1& 1& 0& 0& 0& 1
					}
				$$\par
				Описанными выше заменами переменных она приводится к виду
				$$
					M = \bordermatrix{
						& \bar{x}_1& \vr x_2 & x_7 & \bar{x}_9 & \vr \bar{x}_3 & \bar{x}_6 & \vr x_4 & \bar{x}_8\cr
						&0 & \VR 0 &0 &0 & \VR 0& 0& \VR 0& 0\cr
						&0 & \VR 1 &1 &1 & \VR 1& 1& \VR 0& 0\cr
						&1 & \VR 1 &1 &1 & \VR 0& 0& \VR 0& 0\cr
						&1 & \VR 0 &0 &0 & \VR 1& 1& \VR 1& 1
					}
				$$\par
				Матрица соответствующей приведённой функции $\varphi(\tilde z^4)$:
				$$
					M' = \bordermatrix{
						& z_1& z_2 & z_3 & z_4\cr
						&0 &0 &0 &0\cr
						&0 &1 &1 & 0\cr
						&1 &1 &0 & 0\cr
						&1 &0 &1 &1
					}
				$$\par
				Минимальная ДНФ функции $\varphi$:
				$$
					\mathcal{D}_{\varphi}^z = \bar{z}_1z_4 \vee \bar{z}_3z_4 \vee z_1z_2z_3 \vee z_1\bar{z}_2\bar{z}_4 \vee \bar{z}_1z_2\bar{z}_3 \vee \bar{z}_1\bar{z}_2z_3.
				$$\par
				В исходных переменных $x$:
				$$
					\mathcal{D}_{\varphi}^x = x_1x_4 \vee x_3x_4 \vee \bar{x}_1x_2\bar{x}_3 \vee \bar{x}_1\bar{x}_2\bar{x}_3 \vee x_1x_2x_3 \vee x_1\bar{x}_2\bar{x}_3.
				$$\par
				ДНФ функции $f$:
				$$
					\mathcal{D}_f = (x_2\bar{x}_7 \vee x_7x_9 \vee \bar{x}_9\bar{x}_2) \vee (\bar{x}_3x_6 \vee \bar{x}_6x_3) \vee (x_4x_8 \vee \bar{x}_8\bar{x}_4) \vee \bar{x}_5 \vee \mathcal{D}_{\varphi}^x = $$ $$
					= x_2\bar{x}_7 \vee x_7x_9 \vee \bar{x}_9\bar{x}_2 \vee \bar{x}_3x_6 \vee \bar{x}_6x_3 \vee x_4x_8 \vee \bar{x}_8\bar{x}_4 \vee \bar{x}_5 \vee x_1x_4 \vee x_3x_4 \vee \bar{x}_1x_2\bar{x}_3 \vee \bar{x}_1\bar{x}_2\bar{x}_3 \vee x_1x_2x_3 \vee x_1\bar{x}_2\bar{x}_3.
				$$
			\end{example}
		\subsection{Метод Дьяконова} \label{dj}
			В статье [6] А.\,Г. Дьяконова описывается метод построения ДНФ булевой функции, заданной матрицей своих нулей, в которой существует единичная подматрица (с точностью до перестановки столбцов и инвертирования их значений).\par
			Опишем полученные А.\,Г. Дьяконовым результаты построения ДНФ приведённой функции. Связь ДНФ исходной функции с ДНФ соответствующей ей приведённой подробно описана в \ref{zhuravlev_kogan}.\par
			Напомним используемые в статье обозначения.\par
			Задана приведённая функция $\varphi(\tilde x^n)$ и её матрица нулей $||\alpha_{ij}||_{k \times n}.$\par
			Столбцу $(\alpha_{1j}, \ldots, \alpha_{kj})^{\text{T}}$ соответствует переменная $x_{z(j)}, z(j) = \sum\limits_{i = 1}^k 2^{i - 1}\alpha_{ij},$ переменной $x_t, t\in \{z(1),\ldots,z(n)\}$, --- столбец $(\alpha_{1y(t)}, \ldots, \alpha_{ky(t)})^{\text{T}}.$\par
			Множества индексов нулевых/ненулевых элементов столбцов:
			$$
				Z(t) = \{j | \alpha_{jy(t)} = 0\},\quad E(t) = \{j | \alpha_{jy(t)} = 1\}.
			$$\par
			Множество индексов переменных, соответствующие столбцы которых не входят в единичную подматрицу:
			$$
				V = \{z(1), \ldots, z(n)\} \backslash \left\{2^i | i = \overline{0,\,k - 1}\right\}.
			$$\par
			Разбиение множества столбцов длины $(k - 1)$
			$$
				\Psi = \{\psi(t) = (\alpha_{1t}, \ldots, \alpha_{(k - 1) t})^{\text{T}} | t \in V\} \subset E^{k - 1}\backslash E^{k - 1}_0\backslash E^{k - 1}_1 \backslash E^{k - 1}_{k - 1}
			$$
			 на непересекающиеся цепи:
			$$
				V = \bigcup_{i = 1}^qB^i,\quad B^i\cap B^j = \varnothing,\,\,i \neq j,\,\,i,j \in \{1,\ldots,q\},\quad B^i = \{b^i_1,\ldots, b^i_{\beta(i)} \},
			$$
			при этом $(\psi(y(b^i_1)), \ldots, \psi(y(b^i_{\beta(i)}))$ --- цепь в $\Psi.$\par
			ДНФ приведённой функции:
			$$
				\mathcal{D}^{\gamma} = \bigvee_{t  \in V} \left(x_t \wedge \bigwedge_{i \in E(t)} \bar{x}_{2^{i - 1}} \vee \bar{x}_t \wedge \bigwedge_{i \in Z(t)} \bar{x}_{2^{i - 1}}\right) \vee \bigvee_{1 \leqslant i < j \leqslant k} x_{2^{i - 1}} x_{2^{j - 1}}
			$$\par
			Её длина $L(\mathcal{D}^{\gamma}) = 2n + \frac{1}{2}(k^2 - 5k).$\par
			
			$$
				\mathcal{D}^{\Gamma} = \bigvee_{i = 1}^q \left(\bigvee_{j = 1}^{\beta(i) - 1} \bar{x}_{b_j^i} \bar{x}_{b_{j + 1}^i - b_j^i} \bar{x}_{b_{j + 1}^i} \vee x_{b_1^i} \wedge \bigwedge_{j \in E(b_1^i)}\bar{x}_{2^{j - 1}} \vee \bar{x}_{b^i_{\beta(i)}} \wedge \bigwedge_{j \in Z(b^i_{\beta(i)})}\bar{x}_{2^{j - 1}} \right)
				 \vee \bigvee_{1 \leqslant i < j \leqslant k} x_{2^{i - 1}} x_{2^{j - 1}}
			$$\par
			Её длина $L(\mathcal{D}^{\Gamma}) = n + \frac{1}{2}(k^2 - 5k) + q.$\par
			Обе ДНФ $\mathcal{D}^{\gamma}$ и $\mathcal{D}^{\Gamma}$ реализуют приведённую функцию $\varphi$ и являются тупиковыми. $\mathcal{D}^{\gamma}$ является частным случаем $\mathcal{D}^{\Gamma},$ когда разбиение булева куба на цепи представляет собой набор цепей по одному элементу. Поэтому для $\mathcal{D}^{\gamma}$ формула выписывается явно, а для $\mathcal{D}^{\Gamma}$ необходимо произвести предварительное разбиение булева куба $E^{k - 1}$ на непересекающиеся цепи.\par
	\newpage
	\section{Метод разбиения на полосы}
		Пусть функция $f(\tilde x^n)$ задана своей матрицей нулей $M_{k \times n}$ и пусть требуется построить её ДНФ по возможности меньшей сложности. Предложим алгоритм, который заключается в построении КНФ этой функции, состоящей из скобок небольшой сложности, с последующим их раскрытием.\par
		Выберем некоторый способ разбиения и поделим матрицу $M$ на $r$ полос.\par
		%Выберем некоторым образом параметр $p$ и поделим матрицу $M$ на $r = \left[\frac{k}{p}\right]$ полос по $p$ строк в каждой.\par
		Для каждой из подматриц $M_i, i = \overline{1,\,r},$ построим ДНФ $\mathcal{D}_i$, соответствующую функции с такой матрицей нулей.\par
		Исходная функция $f$ представима в виде
		$$
			f(\tilde x^n) = \bigwedge_{i = 1}^r \mathcal{D}_i,
		$$
		и её ДНФ $\mathcal{D}_f$ может быть получена раскрытием скобок в данном выражении и применением правила поглощения.\par
		Видно, что чем меньше длина каждой ДНФ $\mathcal{D}_i,$ тем меньше длина и искомой результирующей ДНФ $\mathcal{D}_f.$\par
		Для определения $\mathcal{D}_i$ воспользуемся результатами А.\,Г. Дьяконова (см. \ref{dj}) построения ДНФ булевых функций, матрица нулей которых содержит единичную подматрицу. По заданной матрице нулей $M_i$ ДНФ $\mathcal{D}_i$ можно строить любым из описанных в \ref{dj} двух способов.\par
		Этим направлением разбиения матрицы на полосы занимались и ранее, например, проведено исследование в дипломной работе Д.\,Ю. Морозовой (Москва, факультет ВМК, 2014).\par
		\subsection{Cпособы разбиения строк матрицы нулей}
			Возможны различные подходы к выбору ширины полос, на которые разбивается матрица нулей $M.$\par
			$1^{\circ}.$ Пусть фиксируется параметр $p$ и исходная матрица нулей $M$ делится на $r = \left[\frac{k}{p}\right]$ полос одинаковой ширины.\par
			При $p = 1$ описанный выше алгоритм сведётся к построению ДНФ по каждому отдельному нулю функции $f$ и записи совершенной конъюнктивной нормальной формы. При $p = k$ мы ничем не упростили задачу построения ДНФ функции с $k$ нулями. Покажем, что выбором $1 \leqslant p \leqslant k$ можно добиться улучшения результата.\par
			Исходя из того, что
			\begin{itemize}\itemsep=0pt
				\item
				в каждой образованной $M_i$ должна быть единичная подматрица,
				\item
					чем меньше различных столбцов в матрице $M_i,$ тем меньше длина образуемой ДНФ,
			\end{itemize}
			предложим следующий жадный алгоритм.\par
			Предполагая, что выбранный параметр $p$ достаточно мал, мы можем позволить полный перебор $C_k^p$ подножеств строк (например, при $p = 5$ и $k = 25$ необходимо рассмотреть $C_{25}^5 = 53130$ вариантов).\par
			Для первой группы строк перебираются все подмножества по $p$ из $k$ строк исходной матрицы $M$ и выбирается такое, в котором присутствует единичная подматрица и содержится как можно больше одинаковых столбцов. Для второй группы процедура повторяется выбором подножества по $p$ из оставшихся $k - p$ строк и так далее.\par
			Если в какой-то момент оказалось, что ни одно подмножество строк не образует единичную подматрицу, то параметр $p$ уменьшается, т.е. последующие полосы будут иметь меньшую <<ширину>>, и процесс для них повторяется. Заметим, что уменьшением $p$ всегда можно добиться получения единичной матрицы в выбранной подматрице: для $p = 1$ это выполнено всегда.\par %В случае приведённой функции это верно и для $p = 2,$ поскольку отсутствуют константные и одинаковые столбцы. (нужно получше доказать)\par А ЧТО В СЛУЧАЕ НУЛЕВОЙ СТРОКИ???
		\begin{algorithm}[H]
			\SetAlgoLined
			\KwData{$M \in \{0,\,1\}^{k \times n}$ --- бинарная матрица}
			\Parameter{$p$ --- число строк в образованных подматрицах}
		 	\KwResult{$\mathcal{M} = \{M_i\}_{i = 1}^{r}$ --- разбиение матрицы $M$ по строкам, где $r = \left [ \frac{k}{p}\right ]$}
			$Inds \mathbin{:=} \{1,\ldots,k\}$ --- множество ещё не использованных индексов строк матрицы $M$\;
			\For{$i = \overline{1,\,r}$}{
				\For{\textnormal{каждого подмножества $subset$ по $p$ индексов из $Inds$}}{
					\If{\textnormal{в матрице, образованной выбранными строками $subset$, существует единичная подматрица и число различных столбцов меньше, чем в текущем выбранном наборе $subset_i,$}}{
						$subset_i \mathbin{:=} subset$
					}
				}
				$M_i \mathbin{:=} M(subset_i)$\;
				$Inds \mathbin{:=} Inds \backslash subset_i$\;
			}
			\caption{Разбиение матрицы нулей $M$}
		\label{slice}
		\end{algorithm}\par
			Отсутствие единичной подматрицы размера $p \times p$ с точностью до перестановки столбцов и их инвертирования означает, что в хотя бы $n - p$ столбцах присутствует хотя бы 2 нуля и 2 единицы.\par
			$2^{\circ}.$ Метод А.\,Г. Дьяконова строит ДНФ, длины которых показали себя достаточно близкими к оптимальным в ряде случаев. Если бы исходная матрица $M$ содержала единичную подматрицу, то разбивать её и вовсе не потребовалось бы.\par
			Будем искать наибольшее множество строк матрицы $M$, содержащее единичную подматрицу. После их удаления повторим процедуру для оставшихся строк и так далее.\par
			Этот способ может оказаться более вычислительно затратным, чем предыдущий, поскольку размер $p$ наибольшей единичной подматрицы заранее неизвестнен, и на первом шаге необходимо последовательно перебирать все возможные размеры от $k$ до 1. При этом число перебираемых вариантов подмножеств строк, образующих подматрицы, равно $C_k^k + C_k^{k - 1} + \ldots + C_k^p$.\par
			\begin{algorithm}[H]
			\SetAlgoLined
			\KwData{$M \in \{0,\,1\}^{k \times n}$ --- бинарная матрица}
		 	\KwResult{$\mathcal{M} = \{M_i\}_{i = 1}^{r}$ --- разбиение матрицы $M$ по строкам}
			$Inds \mathbin{:=} \{1,\ldots,k\}$ --- множество ещё не использованных индексов строк матрицы $M$\;
			\For{$p = k,\ldots, 1$}{
				\While{\textnormal{в $M$ существует единичная подматрица размера $p \times p,$}}{
					\For{\textnormal{каждого подмножества $subset$ по $p$ индексов из $Inds$}}{
						\If{\textnormal{в матрице, образованной выбранными строками $subset$, существует единичная подматрица и число различных столбцов меньше, чем в текущем выбранном наборе $subset_i,$}}{
							$subset_i \mathbin{:=} subset$
						}
					}
					$\mathcal{M} \mathbin{:=} \mathcal{M} \cup M(subset_i)$\;
					$Inds \mathbin{:=} Inds \backslash subset_i$\;
				}
			}
			\caption{Разбиение матрицы нулей $M$}
		\label{slice_max_id}
		\end{algorithm}\par
		\subsection{Оценка длины ДНФ}
			Оценим число конъюнкций в результирующей ДНФ $\mathcal{D}_f.$\par
			Напомним, что в случае фиксированного параметра $p$ ДНФ $\mathcal{D}_f$ получается перемножением $r = \left [ \frac{k}{p}\right ]$ скобок, каждая из которых представляет собой ДНФ $\mathcal{D}_i$, полученную по способу, описанному в \ref{dj}. 
			$$L(\mathcal{D}_i) = 2m_i + \frac{1}{2}(p^2 - 5p) + L(\mathcal{D}_i'),$$
			где $\mathcal{D}_i'$ соответствует переводу ДНФ приведённой функции к исходной, $L(\mathcal{D}_i') \leqslant n$.\par
			Тогда длина ДНФ $\mathcal{D}_f$ до применения поглощений после раскрытия скобок оценивается как
			$$
				L(\mathcal{D}_f) = \prod_{i = 1}^{r} \left(2m_i + \frac{1}{2}(p^2 - 5p) + L(D_i') \right).
			$$\par
			Как функция от параметра $p,\,\,1 \leqslant p \leqslant k,$ она выпукла:
			$$
				L(p; n, k) = \left( \underline{O}(n) + \frac{1}{2}(p^2 - 5p)\right)^{\frac{k}{p}}.
			$$
			\begin{example}
				Рассмотрим пример построения ДНФ функции описанным выше методом.\par
				Пусть задана матрица 15-ти нулей функции $f(\tilde x^{15})$:
				$$
					\bordermatrix{
					& x_1 & x_2 & x_3 & x_4 & x_5 & x_6 & x_7 & x_8 & x_9 & x_{10} & x_{11} & x_{12} & x_{13} & x_{14} & x_{15} \cr
& 0 & 0 & 0 & 1 & 0 & 0 & 1 & 1 & 0 & 0 & 1 & 0 & 0 & 1 & 0 \cr
& 0 & 0 & 1 & 1 & 1 & 0 & 0 & 1 & 1 & 1 & 1 & 0 & 1 & 1 & 1 \cr
& 0 & 1 & 0 & 1 & 0 & 1 & 0 & 0 & 1 & 1 & 1 & 0 & 0 & 0 & 1 \cr
& 1 & 0 & 0 & 0 & 0 & 1 & 0 & 1 & 0 & 1 & 1 & 0 & 0 & 1 & 1 \cr
& 1 & 1 & 0 & 0 & 1 & 1 & 0 & 1 & 0 & 1 & 1 & 0 & 1 & 1 & 0 \cr
& 0 & 0 & 1 & 1 & 0 & 1 & 0 & 0 & 1 & 1 & 0 & 0 & 0 & 0 & 1 \cr
& 0 & 0 & 0 & 1 & 1 & 1 & 0 & 1 & 1 & 0 & 1 & 0 & 0 & 1 & 0 \cr
& 0 & 1 & 1 & 0 & 1 & 0 & 1 & 0 & 1 & 0 & 1 & 1 & 1 & 0 & 1 \cr
& 1 & 0 & 0 & 0 & 0 & 1 & 0 & 0 & 0 & 0 & 0 & 0 & 0 & 1 & 0 \cr
& 1 & 1 & 0 & 1 & 0 & 0 & 0 & 0 & 1 & 0 & 0 & 0 & 1 & 1 & 1 \cr
& 1 & 0 & 1 & 1 & 0 & 0 & 0 & 0 & 1 & 1 & 1 & 0 & 1 & 1 & 1 \cr
& 1 & 1 & 0 & 1 & 0 & 0 & 1 & 1 & 0 & 0 & 1 & 0 & 1 & 1 & 1 \cr
& 1 & 1 & 1 & 0 & 1 & 0 & 1 & 1 & 0 & 0 & 0 & 0 & 1 & 1 & 0 \cr
& 0 & 0 & 0 & 0 & 1 & 0 & 1 & 0 & 1 & 0 & 1 & 0 & 1 & 0 & 0 \cr
& 1 & 0 & 0 & 1 & 0 & 1 & 1 & 0 & 0 & 1 & 1 & 1 & 1 & 1 & 1 \cr
}
				$$\par
				Она была разбита на 3 полосы по 6, 5 и 4 нуля следующим образом:
				$$
					M_1 = \begin{pmatrix}
0 & 0 & 0 & 1 & 0 & 0 & 1 & 1 & 0 & 0 & 1 & 0 & 0 & 1 & 0 \\
0 & 0 & 1 & 1 & 1 & 0 & 0 & 1 & 1 & 1 & 1 & 0 & 1 & 1 & 1 \\
0 & 1 & 0 & 1 & 0 & 1 & 0 & 0 & 1 & 1 & 1 & 0 & 0 & 0 & 1 \\
1 & 0 & 0 & 0 & 0 & 1 & 0 & 1 & 0 & 1 & 1 & 0 & 0 & 1 & 1 \\
0 & 0 & 1 & 1 & 0 & 1 & 0 & 0 & 1 & 1 & 0 & 0 & 0 & 0 & 1 \\
1 & 0 & 0 & 1 & 0 & 1 & 1 & 0 & 0 & 1 & 1 & 1 & 1 & 1 & 1 \\
\end{pmatrix}
				$$
				$$
					M_2 = \begin{pmatrix}
 1 & 1 & 0 & 0 & 1 & 1 & 0 & 1 & 0 & 1 & 1 & 0 & 1 & 1 & 0 \cr
 0 & 0 & 0 & 1 & 1 & 1 & 0 & 1 & 1 & 0 & 1 & 0 & 0 & 1 & 0 \cr
 0 & 1 & 1 & 0 & 1 & 0 & 1 & 0 & 1 & 0 & 1 & 1 & 1 & 0 & 1 \cr
 1 & 1 & 0 & 1 & 0 & 0 & 1 & 1 & 0 & 0 & 1 & 0 & 1 & 1 & 1 \cr
 1 & 1 & 1 & 0 & 1 & 0 & 1 & 1 & 0 & 0 & 0 & 0 & 1 & 1 & 0 \cr
\end{pmatrix}
				$$
				$$
					M_3 = \begin{pmatrix}
 1 & 0 & 0 & 0 & 0 & 1 & 0 & 0 & 0 & 0 & 0 & 0 & 0 & 1 & 0 \cr
 1 & 1 & 0 & 1 & 0 & 0 & 0 & 0 & 1 & 0 & 0 & 0 & 1 & 1 & 1 \cr
 1 & 0 & 1 & 1 & 0 & 0 & 0 & 0 & 1 & 1 & 1 & 0 & 1 & 1 & 1 \cr
 0 & 0 & 0 & 0 & 1 & 0 & 1 & 0 & 1 & 0 & 1 & 0 & 1 & 0 & 0 \cr
\end{pmatrix}
				$$\par
				Матрицы соответствующих приведённых функций с выделенными единичными подматрицами:
				$$ M'_1 =
				\begin{tikzpicture}[baseline={(m.center)}]
				   \matrix [matrix of math nodes,left delimiter=(,right delimiter=)] (m)
    					    {
            0 & 0 & 0 & 0 & 0 & 0 & 0 & 0 & 0 & 1 & 1 & 1 & 1 & 1 \\
 0 & 0 & 0 & 0 & 0 & 1 & 1 & 1 & 1 & 0 & 0 & 1 & 1 & 1 \\
 0 & 0 & 0 & 1 & 1 & 0 & 0 & 1 & 1 & 0 & 1 & 0 & 0 & 1 \\
 0 & 0 & 1 & 0 & 0 & 0 & 0 & 0 & 1 & 0 & 1 & 0 & 1 & 0 \\
 0 & 1 & 0 & 0 & 1 & 0 & 1 & 1 & 1 & 0 & 1 & 0 & 0 & 1 \\
 1 & 0 & 0 & 0 & 0 & 0 & 0 & 0 & 0 & 0 & 0 & 0 & 0 & 0 \\
        };  
       				 \draw[color=black] (m-1-1.north west) -- (m-1-4.north east) -- (m-6-4.south east) -- (m-6-1.south west) -- (m-1-1.north west);
				 \draw[color=black] (m-1-6.north west) -- (m-1-6.north east) -- (m-6-6.south east) -- (m-6-6.south west) -- (m-1-6.north west);
				 \draw[color=black] (m-1-10.north west) -- (m-1-10.north east) -- (m-6-10.south east) -- (m-6-10.south west) -- (m-1-10.north west);
       				 %\draw[color=black,double,implies-](m-1-2.north) -- +(0,0.3);
  			  \end{tikzpicture}_{6\times 14}$$
			  $$ M'_2 =
				\begin{tikzpicture}[baseline={(m.center)}]
				   \matrix [matrix of math nodes,left delimiter=(,right delimiter=)] (m)
    					    {
             0 & 0 & 0 & 0 & 0 & 0 & 0 & 1 & 1 & 1 \\
 0 & 0 & 0 & 0 & 1 & 1 & 1 & 0 & 1 & 1 \\
 0 & 0 & 1 & 1 & 0 & 0 & 1 & 0 & 0 & 0 \\
 0 & 1 & 0 & 1 & 0 & 1 & 0 & 0 & 0 & 1 \\
 1 & 0 & 0 & 0 & 0 & 0 & 0 & 0 & 0 & 0 \\
        };  
       				 \draw[color=black] (m-1-1.north west) -- (m-1-3.north east) -- (m-5-3.south east) -- (m-5-1.south west) -- (m-1-1.north west);
				 \draw[color=black] (m-1-5.north west) -- (m-1-5.north east) -- (m-5-5.south east) -- (m-5-5.south west) -- (m-1-5.north west);
				 \draw[color=black] (m-1-8.north west) -- (m-1-8.north east) -- (m-5-8.south east) -- (m-5-8.south west) -- (m-1-8.north west);
       				 %\draw[color=black,double,implies-](m-1-2.north) -- +(0,0.3);
  			  \end{tikzpicture}_{5\times 10}$$
			  $$ M'_3 =
				\begin{tikzpicture}[baseline={(m.center)}]
				   \matrix [matrix of math nodes,left delimiter=(,right delimiter=)] (m)
    					    {
 0 & 0 & 0 & 0 & 1 & 1 \\
 0 & 0 & 1 & 1 & 0 & 1 \\
 0 & 1 & 0 & 1 & 0 & 0 \\
 1 & 0 & 0 & 0 & 0 & 0 \\
        };  
       				 \draw[color=black] (m-1-1.north west) -- (m-1-3.north east) -- (m-4-3.south east) -- (m-4-1.south west) -- (m-1-1.north west);
				 \draw[color=black] (m-1-5.north west) -- (m-1-5.north east) -- (m-4-5.south east) -- (m-4-5.south west) -- (m-1-5.north west);
       				 %\draw[color=black,double,implies-](m-1-2.north) -- +(0,0.3);
  			  \end{tikzpicture}_{4\times 6}$$\par
			Длины ДНФ приведённых функций, построенных по методу Дьяконова:
			$$
				L(\mathcal{D}'_1) = 2\cdot14 + \frac{1}{2}(6^2 - 5\cdot6) = 31$$ $$ L(\mathcal{D}'_2) = 2\cdot10 + \frac{1}{2}(5^2 - 5\cdot5) = 20$$ $$L(\mathcal{D}'_3 )= 2\cdot6 + \frac{1}{2}(4^2 - 5\cdot4) = 10
			$$\par
			Длины ДНФ исходных функций, задаваемых матрицами нулей $M_1, M_2, M_3,$ где добавляются конъюнкции, соответствующие константым столбцам и группам переменных с одинаковыми столбцами в матрице нулей:
			$$
				L(\mathcal{D}_1) = 2 + L(\mathcal{D}'_1) = 33
			$$
			$$
				L(\mathcal{D}_2) = 3 + 2 + 2 + 2 + L(\mathcal{D}'_2) = 29
			$$
			$$
				L(\mathcal{D}_3) = 2 + 4 + 2 + 2 + 3 + L(\mathcal{D}'_3) = 23
			$$\par
			Длина ДНФ $\mathcal{D}$ после раскрытия скобок и применения правила поглощения:
			$$
				L(\mathcal{D}) = 11647
			$$\par
			\end{example}
	\newpage
	\section{Построение тупиковой ДНФ}
		Напомним, что тупиковой ДНФ функции $f$ называется такая ДНФ её простых импликант, из которых нельзя выбросить ни одного импликанта, не изменив функции $f$.\par
		Известно, что любая тупиковая ДНФ, в частности, минимальная, может быть получена из сокращённой путём удаления некоторых импликант. В общем случае поиск сводится к полному перебору всех возможных комбинаций. Существуют различные способы повышения эффективности алгоритма синтеза минимальных или достаточно простых ДНФ по сравнению с сокращённой нормальной формой.\par
			%Рассмотрим один из подходов \emph{приближённого} решения описанной задачи.
		Пусть дана сокращённая ДНФ $\mathcal{D}_{f}$ функции $f$:
		$$
			\mathcal{D}_{f} = \bigvee_{i = 1}^{n} K_{i}.
		$$ \par
		Распространена, по-видимому, правильная гипотеза о том, что построение минимальной ДНФ этой функции является $NP$-полной задачей, поэтому предложим вариант построения некоторой тупиковой ДНФ $f$, которая в некоторых случаях, вероятно, достаточно близка по сложности к минимальной.
		\subsection{Проверка поглощения конъюнкции}
			Пусть для конъюнкции $K$ требуется проверить, поглощается ли она остальными конъюнкциями ДНФ $\mathcal{D}.$ Рассмотрим прямой алгоритм поточечной проверки интервала, соответствующего конъюнкции $K.$\par
			Определим окрестность 1-го порядка $S_1(K)$ конъюнкции $K.$ Будем по очереди перебирать конъюнкции $K' \in S_1(K)$ и пересекать соответствующий им интервал $N_{K'}$ c интервалом $N_K.$ Если в результате все точки интервала $N_K$ оказались покрытыми, то $K$ поглощается конъюнкциями из своей окрестности 1-го порядка.\par
			Пусть $K = x_{i_1}^{\sigma_{i_1}} \ldots x_{i_k}^{\sigma_{i_k}},\,\, \text{rg}K = k,$ и $K' = x_{j_1}^{\sigma_{j_1}} \ldots x_{j_m}^{\sigma_{j_m}},\,\, \text{rg}K' = m.$ Определим конъюнкцию, соответствующую пересечению интервалов $N_K$ и $N_{K'}.$ Она будет иметь вид $K'' = x_{i_1}^{\sigma_{i_1}} \ldots x_{i_k}^{\sigma_{i_k}} x_{j_1}^{\sigma_{j_1}} \ldots x_{j_m}^{\sigma_{j_m}},$ при этом некоторые буквы, входящие в запись $K$ и $K'$ могут совпадать, т.е. $\max\{k, m\} \leqslant \text{rg}K'' \leqslant k + m.$ Тогда сложность построения интервала пересечения составляет $\underline{O}(k + m) = \underline{O}(n),$ где $n$ --- число переменных.\par
			Пусть $|S_1(K)| = t,$ тогда построение всех пересечений займёт $\underline{O}(tn)$ операций.\par
			Упорядочим полученные конъюнкции-пересечения по убыванию их размерности, таким образом, при проверке покрытия точек конъюнкции $K$ мы начнём с тех, которые накроют максимальное число точек. Поскольку ранги конъюнкций --- целые числа, непревосходящие $n,$ то для их упорядочения можно применить алгоритм сортировки подсчётом, временная сложность которого линейна $\underline{O}(t + n).$ При этом можно удалить повторы, которые в отсортированном списке конъюнкций будут идти подряд.\par
			Для каждой из $t$ конъюнкций $K''$ необходимо отметить покрываемые ею точки, число которых $2^{n - \text{rg}K''}.$\par
		\begin{algorithm}[H]
			\SetAlgoLined
			\KwData{$K$ --- испытуемая конъюнкция, $S_1(K)$ --- её окрестность 1-го порядка}
		 	\KwResult{ответ, покрывается ли $K$ её окрестностью 1-го порядка}
			$X \mathbin{:=} \varnothing$ --- множество конъюнкций, соответствующих пересечениям\;
			\For{\textnormal{каждой $K' \in S_1(K)$}}{
				$X \mathbin{:=} X \cup \{K \wedge K'\}$\;
			}
			Сортировка $X$ в соответствии с размерностями интервалов конъюнкций\;
			Интервал $N_K$ конъюнкции $K$ заполняется нулями\;
			$c \mathbin{:=} 0$ --- обнуляется счётчик покрытых точек\;
			\For{\textnormal{каждой $K'' \in X$}}{
				$c \mathbin{:=} c + 2^{n - \text{rg}K''} - (\#\text{ число отмеченных единиц в $N_{K''}$})$\;
				Подынтервал $N_{K''}$ интервала $N_K$ заполняется единицами\;
				\If{$c = 2^{n - \text{rg}K},$}{
						Возвращается ответ, что конъюнкция $K$ поглощается, и выход\;
					}
				}
			Возвращается ответ, что конъюнкция $K$ не поглощается, и выход\;
			\caption{Проверка факта поглощения конъюнкции}
			\label{check_cover}
		\end{algorithm}
		\subsection{Очередь проверки}
			При построении тупиковой ДНФ по данной сокращённой требуется вычеркнуть из неё некоторые конъюнкции так, чтобы оставшиеся интервалы образовывали неприводимое покрытие. Опишем метод, предложенный в нашей предыдущей курсовой работе, который показал для некоторых ДНФ неплохие результаты.\par
			Итак, пусть $\mathcal{D}_f = \bigvee\limits_{i = 1}^r K_i$ --- сокращённая ДНФ функции $f$. Будем по очереди выбрасывать из $\mathcal{D}_f$ те конъюнкции, соответствующие интервалы которых покрываются минимальным числом других интервалов из своей окрестности 1-го порядка, пока не останется неприводимое покрытие.\par
			\begin{algorithm}[H]
				\SetAlgoLined
				\KwData{$\mathcal{D}_f$ --- сокращённая ДНФ функции $f$}
		 		\KwResult{$\mathcal{D}'$ --- тупиковая ДНФ, реализующая функцию $f$}
				$\mathcal{D}' \mathbin{:=} \mathcal{D}_f$ --- искомая ДНФ\;	
				\For{\textnormal{каждой $K \in \mathcal{D}$}}{
					Определяем её окрестность 1-го порядка $S_1(K)$\;
				}
				\While{\textnormal{$\exists K^{*} \in \mathcal{D}',$ которая поглощается конъюнкциями из своей окрестности 1-го порядка,}}{
					\For{\textnormal{каждой такой $K^{*}$}}{
						Вычисляем mincover($K^{*}$) --- минимальное число конъюнкций из $S_1(K^{*}),$ которыми поглощается $K^{*}$\;
						\If{\textnormal{$\text{mincover}(K^{*}) < \text{mincover}(K_{\min}),$}}{
							$K_{\min} \mathbin{:=} K^{*}$ --- обновляется текущая лучшая найденная конъюнкция\;
						}
					}
					\For{\textnormal{каждой $K \in S_1(K_{\min})$}}{
						$S_1(K) \mathbin{:=} S_1(K) \backslash K_{\min}$\;
					}
					$\mathcal{D}' \mathbin{:=} \mathcal{D}' \backslash K_{\min}$\;
				}
			\caption{Построение тупиковой ДНФ}
			\label{get_irred}
		\end{algorithm}\par
		Основная сложность заключается в определении mincover --- минимального числа конъюнкций, которыми покрывается какая-то данная. Решение подробно описано в предыдущей курсовой работе.
	\newpage
	\section{Критерий поглощения и алгоритм Блейка}
		В ходе применения критерия поглощения возникает, вообще говоря, NP-полная задача определить, является ли полученная ДНФ $\mathcal{D} = \bigvee\limits_{i = 1}^rK_i$ тождественной единицей. Тем не менее, в некоторых случаях проверка выполнимости может быть достаточно эффективно решена с помощью алгоритма Блейка.\par
		Для выполнения этой проверки воспользуемся методом Блейка построения сокращённой ДНФ булевой функции, задаваемой $\mathcal{D}.$\par
		Правило Блейка:
		$$
			Ax \vee B\bar{x} = Ax \vee B\bar{x} \vee AB,
		$$
		$$
			x \vee A\bar{x} = x \vee A.
		$$\par
		Если обобщить правило Блейка на случай, когда и $A$, и $B$ состоят из пустого множества букв, т.е. $A = B = 1,$ то по этому правилу можно было бы получить конъюнкцию, соответствующую 1:
		$$
			x \vee \bar{x} = 1\& x \vee 1\&\bar{x} = 1 \& x \vee 1\&\bar{x} \vee 1 \& 1 = x \vee \bar{x} \vee 1 = 1.
		$$\par
		Соответствующий этой единичной конъюнкции интервал будет максимальным. Т.к. по теореме результат работы метода Блейка --- сокращённая ДНФ, то если изначальная $\mathcal{D} \equiv 1$, мы получим сокращённую ДНФ из одной конъюнкции: 1.\par
		Однако если считать элементарную конъюнкцию из пустого множества букв недопустимой, то в результате такая конъюнкция образована не будет.
		\begin{example}
			$$
				\mathcal{D} = x_3 \vee \bar{x}_1 \vee x_1x_2 \vee x_1\bar{x}_2\bar{x}_3
			$$\par
			Применим правило Блейка и простое поглощение:
			$$
				x_3 \vee \bar{x}_1 \vee x_1x_2 \vee x_1\bar{x}_2\bar{x}_3 \vee x_2 \vee \bar{x}_2\bar{x}_3 \vee x_1\bar{x}_2 \vee x_1 \vee \bar{x}_2 \vee \bar{x}_3 \vee x_1\bar{x}_3 =
			$$
			$$
				= x_1 \vee \bar{x}_1 \vee x_2 \vee \bar{x}_2 \vee x_3 \vee \bar{x}_3
			$$\par
			Для каждой переменной $x_i$ в сокращённую ДНФ вошло $x_i \vee \bar{x}_i.$ Заметим, что и в этом случае можно получить конъюнкцию 1, воспользовавшись правилом $x \vee \bar{x} = 1.$
		\end{example}
		\begin{example}
			$$
				\mathcal{D} = \bar{x}_1 \vee \bar{x}_2 \vee x_1\bar{x}_3 \vee x_1x_2
			$$\par
			Применим правило Блейка и простое поглощение:
			$$
				\bar{x}_1 \vee \bar{x}_2 \vee x_1\bar{x}_3 \vee x_1x_2 \vee \bar{x}_3 \vee x_1 \vee x_2 =
			$$
			$$
				= x_1 \vee \bar{x}_1  \vee x_2 \vee \bar{x}_2 \vee \bar{x}_3
			$$\par
			Не для всех $x_i$ в сокращённую ДНФ вошло выражение $x_i \vee \bar{x}_i.$
		\end{example}
		Если $\mathcal{D} \equiv 1,$ то, вообще говоря, не для всех переменных $x_i$ из её формулы в сокращённой ДНФ будет присутствовать $x_i \vee \bar{x}_i$, это зависит от формы записи исходной ДНФ $\mathcal{D}.$\par
		\begin{statement}
			Пусть задана ДНФ $\mathcal{D}$ и в результате применения метода Блейка получена ДНФ $\mathcal{D}'.$ $\mathcal{D} \equiv 1$ тогда и только тогда, когда $\exists i: x_i \vee \bar{x}_i$ входит в $\mathcal{D}'.$
		\end{statement}
		\begin{proof}
			\textit{Необходимость.} Рассмотрим функцию $f(\tilde x^{n+1}) = x_{n + 1} \wedge \mathcal{D}.$ Т.к. $\mathcal{D} \equiv 1,$ то сокращённая ДНФ $\mathcal{D}_f^{\text{сокр}} = x_{n + 1}.$\par
			По теореме применение метода Блейка к ДНФ $\bigvee\limits_{i = 1}^r x_{n + 1} \wedge K_i$ функции $f$ даст единственную конъюнкцию $x_{n + 1},$ которая могла быть получена из преобразования
			$$
				x_{n + 1} x_i \vee x_{n + 1} \bar{x}_i = x_{n + 1} x_i \vee x_{n + 1} \bar{x}_i \vee x_{n + 1}.
			$$\par
			(При этом правило $x_i \vee x_{n + 1} \bar{x}_i = x_i \vee x_{n + 1}$ не могло быть применено, т.к. все конъюнкции на любом этапе содержат $x_{n + 1},$ а $\bar{x}_{n + 1}$ нигде не встречается и образовано быть не может.)\par
			Значит, данные $x_i$ и $\bar{x}_i$ встретятся в $\mathcal{D}'.$\\
		\textit{Достаточность.} Т.к. $x_i \vee \bar{x}_i \equiv 1$, то $\mathcal{D}' \equiv 1$. ДНФ $\mathcal{D}'$ и $\mathcal{D}$ задают одну и ту же функцию, поэтому из того, что $\mathcal{D}' \equiv 1$, следует, что и $\mathcal{D} \equiv 1.$ Утверждение доказано.
		\end{proof}\par
		Итак, пусть заданы $n$ --- число переменных, конъюнкция $K = x_1\ldots x_q$ ранга $q$ (с помощью замены переменных всегда можно добиться представления в таком виде) и ДНФ $\bigvee\limits_iK'_i.$\par
		В ходе применения критерия поглощения после удаления букв, входящих в конъюнкцию $K$, образована ДНФ $\mathcal{D} = \bigvee\limits_{i = 1}^rK_i,$ в которую входят $2n - q$ различных букв: $x_{q + 1},\ldots, x_n, \bar{x}_1,\ldots,\bar{x}_n$.\par
		В ходе метода Блейка для ДНФ $\mathcal{D}$ правило Блейка может быть применено к парам конъюнкций, содержащим только переменные $x_{q+1}$ и $\bar{x}_{q+1}$, $\ldots$, $x_{n}$ и $\bar{x}_{n},$ остальные же могут встречаться в $\mathcal{D}$ только в единственной степени. Т.е. не более $n - q$ переменных встречаются в $\mathcal{D}$ в обоих степенях.\par
		%Пусть $q = n - 2$. В этом случае правило Блейка применимо к конъюнкциям $x_{n - 1}A \vee \bar{x}_{n - 1}B$ и $x_{n}A \vee \bar{x}_{n}B.$ Новые образованные конъюнкции содержат только отрицания переменных $x_1,\ldots,x_q,$ и тождественная единица может быть получена только в виде $x_{n - 1} \vee \bar{x}_{n - 1}$ или $x_{n} \vee \bar{x}_{n}.$\par
		%Итак, пусть в ходе применения критерия поглощения образована ДНФ $\mathcal{D} = \bigvee\limits_iK_i.$\par
		Будем исследовать поведение метода Блейка, применённого к полученной ДНФ, в зависимости от числа различных переменных, входящих в её запись в обеих степенях.\par
		\subsection{Тривиальный случай}
			Пусть все переменные, встречавшиеся в исходной ДНФ в обоих степенях, входят в испытуемую конъюнкцию $K.$ Тогда после удаления букв ДНФ $\mathcal{D}$ содержит все переменные в одинаковых степенях.\par
			\begin{statement}
				Пусть каждая из переменных $x_i,\,\,i = \overline{1,\,n}$, встречается в любой из конъюнкций ДНФ $\mathcal{D}$ только в степени $\sigma_i \in \{ 0, 1\}.$ Тогда $\mathcal{D} \not\equiv 1$.
			\end{statement}
			\begin{proof}
				Рассмотрим набор $(\bar{\sigma}_1, \ldots, \bar{\sigma}_n).$ Функция, задаваемая ДНФ $\mathcal{D},$ принимает на нём значение 0, т.к. $\bar{\sigma}^{\sigma} = 0.$ Значит, существует точка, не покрываемая $\mathcal{D},$ и $\mathcal{D} \not\equiv 1$.
			\end{proof}
		\subsection{Случай одной переменной}
			Пусть после удаления букв осталась единственная переменная $x$, которая встречается в записи $\mathcal{D}$ в обеих степенях: $x$ и $\bar{x}$.\par
			Пусть $x$ встречается в $r_1$ конъюнкциях, а $\bar{x}$ --- в $r_0.$ В записи одной и той же конъюнкции и $x$, и $\bar{x}$ встречаться не могут, поскольку иначе она была бы противоречивой.\par
			Правило Блейка применимо $r_1\times r_0$ раз, по числу различных пар конъюнкций вида $xA \vee \bar{x}B.$ Образованные в ходе метода Блейка конъюнкции имеют вид $AB$ и содержать ни одну из букв $x$ и $\bar{x}$ не могут.\par
			Таким образом будет образовано не более $r_1\times r_0$ новых конъюнкций, и если длина исходной ДНФ равна $r$, то получаем оценку $\underline{O}(r^2).$

%\newpage
\subsection{Случай двух переменных}
			Пусть после удаления букв существуют две переменные, не ограничивая общности, $x_1$ и $x_2$, которые встречаются в записи $\mathcal{D}$ в обеих степенях.\par
			Рассмотрим общий вид ДНФ $\mathcal{D}$:
			$$
				\bigvee_ix_1A_i \vee \bigvee_i\bar{x}_1B_i \vee \bigvee_ix_2C_i \vee \bigvee_i\bar{x}_2D_i \vee \bigvee_ix_1x_2E_i \vee \bigvee_ix_1\bar{x}_2F_i \vee \bigvee_i\bar{x}_1x_2G_i \vee \bigvee_i\bar{x}_1\bar{x}_2H_i \vee \bigvee_iI_i,
			$$
			где $A_i, \ldots, I_i$ не содержат переменных $x_1, x_2.$\par
			Назовём набор конъюнкций $\{K_1, \ldots, K_p\}$ {\it независимыми}, если при перемножении двух различных их подмножеств результирующие конъюнкции не совпадают.\par
			В этом параграфе будет получена точная верхняя оценка на число новых образованных конъюнкций в ходе метода Блейка, применённого к ДНФ описанного выше вида, поэтому везде далее полагаем, что набор конъюнкций $\{A_i, \ldots, H_i\}$ независим. В противном случае при образовании новых конъюнкций, полученных по различным цепочкам применения правила Блейка, некоторые из них могут совпасть, что только уменьшит число новых конъюнкций, которые могли бы быть получены. \par
			Будем применять правило Блейка сначала ко всем парам конъюнкций вида\\$x_i^{\sigma_i}P \vee x_i^{1-\sigma_i}x_j^{\delta_j}Q,\, i,j\in\{1, 2\}$:
			$$
				x_i^{\sigma_i}P \vee x_i^{1-\sigma_i}x_j^{\sigma_j}Q = x_i^{\sigma_i}P \vee x_i^{1-\sigma_i}x_j^{\sigma_j}Q \vee x_j^{\sigma_j}PQ,\qquad(1)
			$$
			в результате чего образуются новые конъюнкции с участием переменной $x_j^{\sigma_j},$ которые впоследствии также могут участвовать в образовании новых конъюнкций.\par
			После того как правило (1) больше неприменимо, выполним все возможные преобразования вида:
			$$
				x_iP \vee \bar{x}_iQ = x_iP \vee \bar{x}_iQ \vee PQ,\qquad(2)
			$$
			в результате которых образуются конъюнкции, в которые переменные $x_1$ и $x_2$ не входят.\par
			Для начала опишем случай, когда в исходную ДНФ $\mathcal{D}$ входит одинаковое число $s$ конъюнкций вида $x_1^{\sigma_1}x_2^{\sigma_2}$ и по $t$ конъюнкций вида $x_i^{\sigma_i}$. Если их встречалось разное число, то положим $s$ или $t$ равным максимальному из них и получим оценку сверху, поскольку чем больше изначально было конъюнкций, тем большее число раз применимо правило Блейка и, следовательно, в ходе метода будет образовано не меньше конъюнкций.\par
			Назовём этапом метода Блейка применение правила (1) к образованным на предыдущем этапе конъюнкциям, содержащим $x_j^{\delta_j}$, в паре с $x_1^{\sigma_1}x_2^{\sigma_2}$ из исходной ДНФ.
			\begin{table}[H]
				\centering
				\begin{tabular}{|c|p{4in}|c|}
					\hline
					Этап & Вид образованных конъюнкций & Число различных конъюнкций\\
					\hline
					\multirow{2}{*}{1} & $x_1CF$ $\bar{x}_1CH$ $x_1DE$ $\bar{x}_1DG$ $x_1EF$ $\bar{x}_1GH$ \vspace{5pt}& \multirow{2}{*}{$2 \times 6$}\\
					&$x_2AG$ $\bar{x}_2AH$ $x_2BE$ $\bar{x}_2BF$ $x_2EG$ $\bar{x}_2FH$&\\
					\hline
					\multirow{2}{*}{2} & $x_1AFG$ $\bar{x}_1AGH$ $x_1AEH$ $x_1BEF$ $\bar{x}_1BEH$ $\bar{x}_1BFG$ $x_1EFG$ $\bar{x}_1EGH$ $x_1EFH$ $\bar{x}_1FGH$ \vspace{5pt} & \multirow{2}{*}{$2 \times 10$}\\
					&$x_2CFG$ $\bar{x}_2CFH$ $x_2CEH$ $x_2DEG$ $\bar{x}_2DEH$ $\bar{x}_2DFG$ $x_2EFG$ $\bar{x}_2EFH$ $x_2EGH$ $\bar{x}_2FGH$&\\
					\hline
					\multirow{2}{*}{3} & $x_1CFG$ $\bar{x}_1CFGH$ $x_1CEFH$ $\bar{x}_1CEH$ $x_1DEFG$ $\bar{x}_1DEGH$ $x_1DEH$ $\bar{x}_1DFG$ $\bar{x}_1EFGH$ $x_1EFGH$\vspace{5pt} & \multirow{2}{*}{$2 \times 10$}\\
					&$x_2AFG$ $\bar{x}_2AFGH$ $x_2AEGH$ $\bar{x}_2AEH$ $x_2BEFG$ $\bar{x}_2BEFH$ $x_2BEH$ $\bar{x}_2BFG$ $\bar{x}_2EFGH$ $x_2EFGH$&\\
					\hline
					\multirow{2}{*}{4} & $\bar{x}_1AFGH$ $x_1AEFGH$ $\bar{x}_1AEGH$ $x_1BEFG$ $\bar{x}_1BEFGH$ $x_1BEFH$ \vspace{5pt} & \multirow{2}{*}{$2 \times 6$}\\
					&$\bar{x}_2CFGH$ $x_2CEFGH$ $\bar{x}_2CEFH$ $x_2DEFG$ $\bar{x}_2DEFGH$ $x_2DEGH$&\\
					\hline
					\multirow{2}{*}{5} & $x_1CEFGH$ $\bar{x}_1CEFGH$ $\bar{x}_1DEFGH$ $x_1DEFGH$ \vspace{5pt} & \multirow{2}{*}{$2 \times 4$}\\
					&$x_2AEFGH$ $\bar{x}_2AEFGH$ $\bar{x}_2BEFGH$ $x_2BEFGH$&\\
					\hline
					\multirow{2}{*}{6} & $\bar{x}_1AEFGH$ $x_1BEFGH$ \vspace{5pt} & \multirow{2}{*}{$2 \times 2$}\\
					&$\bar{x}_2CEFGH$ $x_2DEFGH$ &\\
					\hline
				\end{tabular}
				\caption{$k = 1$}
			\end{table}\par
			Формулы симметричны по переменным $x_1, \bar{x}_1, x_2, \bar{x}_2.$\par
			Здесь $A, \ldots, H$ --- реализация одного из вариантов $A_i, \ldots, H_i$ из исходной ДНФ.\par
			Каждое из выражений $A, B, C$ или $D$ может входить в конъюнкцию с переменной $x_i^{\sigma_i}$ максимум один раз: применение правила Блейка с конъюнкциями вида $x_1A,\,\bar{x}_1B, \,x_2C, \,\bar{x}_2D$ осуществляется только 1-м этапе. \par
			Число этапов: $4s + 2.$\par
			Буква $E, F, G$ или $H$, вошедшая в описание вида конъюнкции, может удвоиться только через шаг: повторное применение правило Блейка с той же конъюнкцией невозможно.\par
			Таким образом, впервые в описание конъюнкции $k$ одинаковых букв входят на $(1 + 2(k - 1))$-м этапе.\par
			До шага $2s$ любое применение правила (1) давало увеличение числа букв, однако когда все варианты задействованы, добавлять больше нечего.\par
			Число видов конъюнкций на этапе $n:$ 
			$$
				\begin{cases}
					(1 + 2n)\times4, \text{если $n \leqslant 2s,$}\\
					(1 + 2(4s + 1 - n))\times4 \text{ иначе.}
				\end{cases}
			$$\par
			Число видов конъюнкций на этапе $n$, в которые входит выражение $A, B, C$ или $D:$ $(n + 1)\times4.$\par
			Число видов конъюнкций на этапе $n$, в которые не входит выражение $A, B, C$ или $D:$ $n\times4.$\par
			Максимальная длина конъюнкции на $n$-м этапе: $\min(n + 1, 1 + 4s)$.\par
			Максимальное число различных видов конъюнкций образуется на этапах $n = 2s$ и $n = 2s + 1$.\par
			Оценим сверху число конъюнкций, образованных на этапе $n \leqslant 2s:$ $4(1 + 2n) \times s^{n + 1}$, где $s^{n + 1}$ --- верхняя оценка на число различных вариантов реализаций конъюнкции определённого вида.\par
			%Максимальное разнообразие возникает при $n = 2k:$ $C_k^{\frac{k}{2}}.$\par
			При $n > 2s$ число симметрично убывает.\par
			%Оценка сверху:
			%$$
			%	\sum_{n = 1}^{2k}4(1 + 2n) k^{n + 1} + \sum_{n = 2k + 1}^{4k}4(1 + 2(4k + 1 - n)) k^{n + 1} =
			%$$
			%$$
			%	= \frac{4 k^2 (3-k-3 k^{2 k}-3 k^{1+2 k}+4 k^{2+2 k})}{(k - 1)^2} + \frac{4 k^{2k + 2} (-1+3 k-4 k^2+k^{2 k}+k^{2k + 1})}{(k - 1)^2} = O(k^{4k})
			%$$\par
			%После преобразований вида (2) получим окончательную оценку $O(k^{8k}).$\vspace{20pt}\par
			Покажем, что возможно образование любой конъюнкции вида:
			$$
			x_i^{\sigma_i} \underbrace{E_1\ldots E_{k_E}}_{1\leqslant k_E \leqslant k}\underbrace{F_1\ldots F_{k_F}}_{1\leqslant k_F \leqslant k}\underbrace{G_1\ldots G_{k_G}}_{1\leqslant k_G \leqslant k}\underbrace{H_1\ldots H_{k_H}}_{1\leqslant k_H \leqslant k}$$\par
			На первом этапе была образована конъюнкция $x_1E_1F_1.$
			\begin{enumerate}\itemsep=0pt
				\item
				$x_1E_1F_1 \vee \bar{x}_1x_2G_1 = x_1E_1F_1 \vee \bar{x}_1x_2G_1 \vee x_2E_1F_1G_1$
				\item
				$x_2E_1F_1G_1 \vee \bar{x}_1\bar{x}_2H_1 = x_2E_1F_1G_1 \vee \bar{x}_1\bar{x}_2H_1 \vee \bar{x}_1E_1F_1G_1H_1$
				\item
				$\bar{x}_1E_1F_1G_1H_1 \vee x_1x_2E_2 = \bar{x}_1E_1F_1G_1H_1 \vee x_1x_2E_2 \vee x_2E_1E_2F_1G_1H_1$
				\item
				$x_2E_1E_2F_1G_1H_1 \vee x_1\bar{x}_2F_2 = x_2E_1E_2F_1G_1H_1 \vee x_1\bar{x}_2F_2 \vee x_1E_1E_2F_1F_2G_1H_1$
			\end{enumerate}
			и т.д. необходимое число раз.\par
			Аналогичным образом могут быть получены конъюнкции, в которые входит одно из выражений $A, B, C$ или $D.$\par
			При добавлении случая, когда какие-либо из $k_E, k_F, k_G, k_H$ равны 0, такая запись охватывает все возможные реализации на всех этапах. Оценим число этих реализаций: $4 \times (4t + 1) \times 2^s2^s2^s2^s = \underline{O}(t2^{4s}).$\par
			После преобразований вида (2) получим следующую оценку на общее число образованных конъюнкций: $4\times (4t + 1)2^{4s} + 2\times t(t + 2^{4s}) + 2\times 2^{4s} = \underline{O}(t^2 + t2^{4s}) = \underline{O}(r2^r),$ cчитая, что длина исходной ДНФ $r$ и $t = \underline{O}(r)$ и $s = \underline{O}(r).$ \par
			Исходя из вида оценки, можно сделать вывод о том, что наибольшее число конъюнкций при фиксированной длине $r = 4t + 4s + q$ исходной ДНФ образуется при $t = 0$ и $s = \frac{r}{4},$ а минимальное в нетривиальном случае при $s = 1$ и $t = \frac{r}{4} - 1.$\par
			Таким образом показано, что наличие даже 2-х букв с отрицаниями делает метод Блейка весьма трудоёмким.\par
			Перейдём к рассмотрению случая образования наибольшего числа новых конъюнкций.\par
	\subsubsection*{Оценка числа образованных конъюнкций в ходе метода Блейка для ДНФ специального вида}
			Пусть в ходе критерия поглощения получена ДНФ
			$$
				\mathcal{D} = \bigvee_{i = 1}^{s_1}x_1x_2E_i \vee \bigvee_{i = 1}^{s_2}x_1\bar{x}_2F_i \vee \bigvee_{i = 1}^{s_3}\bar{x}_1x_2G_i \vee \bigvee_{i = 1}^{s_4}\bar{x}_1\bar{x}_2H_i \vee \bigvee_iI_i,
			$$
			$$
				s_1 + s_2 + s_3 + s_4 = 4s.
			$$\par
			Здесь каждая из конъюнкций $E_i, F_i, G_i, H_i$ содержит хотя бы одну уникальную переменную.\par
			Например,
			$$
				\mathcal{D} = \bigvee\limits_{i = 1}^{s_1}x_1x_2x_{i +2} \vee \bigvee\limits_{i = 1}^{s_2}x_1\bar{x}_2x_{i + s_1 + 2} \vee \bigvee\limits_{i = 1}^{s_3}\bar{x}_1x_2x_{i + s_1 + s_2 + 2} \vee \bigvee\limits_{i = 1}^{s_4}\bar{x}_1\bar{x}_2x_{i + s_1 + s_2 + s_3 + 2},
			$$
			где переменные $x_i, i = \overline{3,\,4s+2},$ встречаются только в одной степени. Её длина $L(\mathcal{D}) = 4s.$\par
			Покажем, что максимальное число новых конъюнкций в ходе метода Блейка, применённого к ДНФ $\mathcal{D}$, образуется в случае $s_1 = s_2 = s_3 = s_4 = s.$\par
			$1^{\circ}.$ Пусть хотя бы одно из $s_1, s_2, s_3$ или $s_4$ равно 0.\par
			$1.1^{\circ}.$ Пусть ровно один параметр равен нулю, для определённости, $s_1 = 0,$ т.е. в $\mathcal{D}$ нет конъюнкций вида $x_1x_2E,$ и
			$$\mathcal{D} = \bigvee_ix_1\bar{x}_2F_i \vee \bigvee_i\bar{x}_1\bar{x}_2G_i \vee \bigvee_i\bar{x}_1\bar{x}_2H_i,$$
			где $F_i \in \{x_{s_1 + 3}, \ldots, x_{s_1 + s_2 + 2}\}, G_i \in \{x_{s_1 + s_2 + 3}, \ldots, x_{s_1 + s_2 + s_3 + 2}\}, H_i \in \{x_{s_1 + s_2  + s_3 + 3}, \ldots, x_{4s + 2}\}.$\par
			Правило Блейка, применённое к исходным конъюнкциям, порождает следующие:\par
			а) $x_1\bar{x}_2F_i \vee \bar{x}_1\bar{x}_2H_i = x_1\bar{x}_2F_i \vee \bar{x}_1\bar{x}_2H_i \vee \bar{x}_2F_iH_i$\par
			б) $\bar{x}_1x_2G_i \vee \bar{x}_1\bar{x}_2H_i = \bar{x}_1x_2G_i \vee \bar{x}_1\bar{x}_2H_i \vee \bar{x}_1G_iH_i$\par
			Возможные варианты правила Блейка, применённого к ним в паре с конъюнкциями из исходной ДНФ $\mathcal{D}:$\par
			а) $\bar{x}_1G_iH_i \vee x_1\bar{x}_2F_i = \bar{x}_1G_iH_i \vee x_1\bar{x}_2F_i \vee \bar{x}_2F_iG_iH_i$\par
			б) $\bar{x}_2F_iH_i \vee \bar{x}_1x_2G_i = \bar{x}_2F_iH_i \vee \bar{x}_1x_2G_i \vee \bar{x}_1G_iG_iH_i$\par
			Образованные конъюнкции содержат переменные $x_1$ и $x_2$ только в отрицательной степени, и, следовательно, не могут больше образовывать новые конъюнкции в паре с конъюнкциями вида $\bar{x}_1\bar{x}_2D_i.$ Более того, конъюнкций без переменных $x_1$ и $x_2$ быть образовано не может: для этого понадобились бы конъюнкции вида $x_1P$ и $x_2Q.$\par
			Все возможные новые конъюнкции:
			$$
				\bar{x}_1GH \quad \bar{x}_1\underbrace{F_1\ldots F_{s_F}}_{1 \leqslant s_F \leqslant s_2}\underbrace{G_1\ldots G_{s_G}}_{1 \leqslant s_G \leqslant s_3}H$$
			$$
				 \bar{x}_2FH \quad\bar{x}_2\underbrace{F_1\ldots F_{s_F}}_{1 \leqslant s_F \leqslant s_2}\underbrace{G_1\ldots G_{s_G}}_{1 \leqslant s_G \leqslant s_3}H
			$$\par
			Подсчитаем их число: $\underbrace{s_3s_4 + (2^{s_2} - 1)(2^{s_3} - 1)s_4}_{\bar{x}_1} + \underbrace{s_2s_4 + (2^{s_2} - 1)(2^{s_3} - 1)s_4}_{\bar{x}_2} = \underline{O}(s2^{2s}),$ что уступает оценке из пункта $2^{\circ}.$\par
			$1.2^{\circ}.$ Пусть два параметра равны нулю.\par
			Если оставшиеся конъюнкции имеют вид $x_1^{\sigma_1}x_2^{\sigma_2}$ и $x_1^{1 - \sigma_1}x_2^{1 - \sigma_2},$ то новых конъюнкций образовано не будет.\par
			В противном случае, если только одна из переменных $x_1$ или $x_2$ встречается в обоих степенях, то число новых конъюнкций равно $s_is_j,$ где $i$ и $j$ --- индексы оставшихся групп.\par
			$1.3^{\circ}.$ Если какие-либо три параметра $s_i, i\in\{1,4\},$ равны нулю, то, очевидно, новых конъюнкций образовано не будет.\par
			$2^{\circ}.$ Пусть $s_1, s_2, s_3, s_4 > 0$ и
			$$\mathcal{D} =\bigvee_ix_1x_2E_i \vee \bigvee_ix_1\bar{x}_2F_i \vee \bigvee_i\bar{x}_1\bar{x}_2G_i \vee \bigvee_i\bar{x}_1\bar{x}_2H_i,$$
			где $E_i \in \{x_3,\ldots,x_{s_1 + 2} \}, F_i \in \{x_{s_1 + 3}, \ldots, x_{s_1 + s_2 + 2}\}, G_i \in \{x_{s_1 + s_2 + 3}, \ldots, x_{s_1 + s_2 + s_3 + 2}\}$,\\$H_i \in \{x_{s_1 + s_2  + s_3 + 3}, \ldots, x_{4s + 2}\}.$\par
%			По рассмотренному ранее все возможные новые образованные конъюнкции:
%			$$
%				\bar{x}_1CD, \bar{x}_1ACD, \bar{x}_1BCD
%			$$
%			$$
%				x_2AC, x_2ABC, x_2ACD
%			$$
%			$$
%				\bar{x}_2BD, \bar{x}_2ABD, \bar{x}_2BCD
%			$$
			
			Поскольку ранее было доказано, что образуются все конъюнкции вида
			\begin{equation}
				x_i^{\sigma} \underbrace{E_1\ldots E_{s_E}}_{1\leqslant s_E \leqslant s_1}\underbrace{F_1\ldots F_{s_F}}_{1\leqslant s_F \leqslant s_2}\underbrace{G_1\ldots G_{s_G}}_{1\leqslant s_G \leqslant s_3}\underbrace{H_1\ldots H_{s_H}}_{1\leqslant s_H \leqslant s_4}, i\in\{1,2\},\sigma\in\{0,1\},
			\end{equation}\\
			исследуем те, в которые не входят все 4 буквы $E, F, G, H,$ т.е. в цепочке из применений правила Блейка участвуют конъюнкции не из всех четырёх групп.
			\begin{table}[H]
				\centering
				\begin{tabular}{|c|p{4in}|}
					\hline
					Этап & Вид образованных конъюнкций\\
					\hline
					1 & $x_1EF$ $\bar{x}_1GH$ $x_2EG$ $\bar{x}_2FH$\\
					\hline
					\multirow{2}{*}{2} & $x_1EFG$ $x_1EFH$ $\bar{x}_1EGH$ $\bar{x}_1FGH$ \vspace{5pt}\\
					 & $x_2EFG$ $x_2EGH$ $\bar{x}_2EFH$ $\bar{x}_2FGH$\\
					\hline
					\multirow{2}{*}{3} & $x_1EFFG$ $x_1EEFH$ $\bar{x}_1EGHH$ $\bar{x}_1FGGH$ \vspace{5pt}\\
					 & $x_2EFGG$ $x_2EEGH$ $\bar{x}_2EFHH$ $\bar{x}_2FFGH$\\
					\hline
					\multirow{2}{*}{4} & $x_1EFFGG$ $x_1EEFHH$ $\bar{x}_1EEGHH$ $\bar{x}_1FFGGH$ \vspace{5pt}\\
					 & $x_2EFFGG$ $x_2EEGHH$ $\bar{x}_2EEFHH$ $\bar{x}_2FFGGH$\\
					\hline
					\multicolumn{2}{|c|}{и т.д.}\\
					\hline
				\end{tabular}
				\captionsetup{justification=centering}
				\caption{Вид конъюнкций, в которые не входят переменные из всех 4-х групп одновременно}
			\end{table}\par
			Таким образом, новые конъюнкции, в которые входят переменные $x_1, x_2,$ имеют вид:
			$$
				x_1EF \quad x_1E\underbrace{F_1\ldots F_{s_F}}_{1 \leqslant s_F \leqslant s_2}\underbrace{G_1\ldots G_{s_G}}_{1 \leqslant s_G \leqslant s_3} \quad x_1\underbrace{E_1\ldots E_{s_E}}_{1 \leqslant s_E \leqslant s_1}F\underbrace{H_1\ldots H_{s_H}}_{1 \leqslant s_H \leqslant s_4}
			$$
			$$
				\bar{x}_1GH \quad \bar{x}_1\underbrace{E_1\ldots E_{s_E}}_{1 \leqslant s_E \leqslant s_1}G\underbrace{H_1\ldots H_{s_H}}_{1 \leqslant s_H \leqslant s_4} \quad 
\bar{x}_1\underbrace{F_1\ldots F_{s_F}}_{1 \leqslant s_F \leqslant s_2}\underbrace{G_1\ldots G_{s_G}}_{1 \leqslant s_G \leqslant s_3}H
			$$
			$$
				x_2EG \quad x_2E\underbrace{F_1\ldots F_{s_F}}_{1 \leqslant s_F \leqslant s_2}\underbrace{G_1\ldots G_{s_G}}_{1 \leqslant s_G \leqslant s_3} \quad x_2\underbrace{E_1\ldots E_{s_E}}_{1 \leqslant s_E \leqslant s_1}G\underbrace{H_1\ldots H_{s_H}}_{1 \leqslant s_H \leqslant s_4}
			$$
			$$
				\bar{x}_2FH \quad \bar{x}_2\underbrace{E_1\ldots E_{s_E}}_{1 \leqslant s_E \leqslant s_1}F\underbrace{H_1\ldots H_{s_H}}_{1 \leqslant s_H \leqslant s_4} \quad \bar{x}_2\underbrace{F_1\ldots F_{s_F}}_{1 \leqslant s_F \leqslant s_2}\underbrace{G_1\ldots G_{s_G}}_{1 \leqslant s_G \leqslant s_3}H
			$$\\
			и (1): $\qquad\quad\,\, x_i^{\sigma} \underbrace{E_1\ldots E_{s_E}}_{1\leqslant s_E \leqslant s_1}\underbrace{F_1\ldots F_{s_F}}_{1\leqslant s_F \leqslant s_2}\underbrace{G_1\ldots G_{s_G}}_{1\leqslant s_G \leqslant s_3}\underbrace{H_1\ldots H_{s_H}}_{1\leqslant s_H \leqslant s_4}, i\in\{1,2\},\sigma\in\{0,1\}$.\par
			Заметим, что для любой пары конъюнкций вида $x_i^{\sigma}P \vee x_i^{1 - \sigma}Q, i\in\{1, 2\},\sigma\in\{0,1\},$ где $P$ и $Q$ не содержат букв $x_1, \bar{x}_1, x_2, \bar{x}_2,$ конъюнкция $PQ$ содержит буквы из всех четырёх групп переменных. В силу реализации всех конъюнкций (1) могут образоваться любые конъюнкции вида $\underbrace{E_1\ldots E_{s_E}}_{1\leqslant s_E \leqslant s_1}\underbrace{F_1\ldots F_{s_F}}_{1\leqslant s_F \leqslant s_2}\underbrace{G_1\ldots G_{s_G}}_{1\leqslant s_G \leqslant s_3}\underbrace{H_1\ldots H_{s_H}}_{1\leqslant s_H \leqslant s_4}$ и только они, поэтому можно точно подсчитать их число: $(2^{s_1} - 1) \times (2^{s_2} - 1) \times(2^{s_3} - 1) \times (2^{s_4} - 1).$\par
			Подсчитаем общее число новых образованных конъюнкций:
			$$
				\underbrace{s_1s_2 + s_1(2^{s_2} - 1)(2^{s_3} - 1) + (2^{s_1} - 1)s_2(2^{s_4} - 1)}_{x_1} + \underbrace{s_3s_4 + (2^{s_1} - 1)s_3(2^{s_4} - 1) + (2^{s_2} - 1)(2^{s_3} - 1)s_4}_{\bar{x}_1} +$$
			$$+ \underbrace{s_1s_3 + s_1(2^{s_2} - 1)(2^{s_3} - 1) + (2^{s_1} - 1)s_3(2^{s_4} - 1)}_{x_2} + \underbrace{s_2s_4 + (2^{s_1} - 1)s_2(2^{s_4} - 1) + (2^{s_2} - 1)(2^{s_3} - 1)s_4}_{\bar{x}_2} +$$
			$$+ \underbrace{4\times(2^{s_1} - 1)  (2^{s_2} - 1) (2^{s_3} - 1)  (2^{s_4} - 1)}_{(1)} + \underbrace{(2^{s_1} - 1)  (2^{s_2} - 1) (2^{s_3} - 1)  (2^{s_4} - 1)}_{\text{без } x_1, x_2}$$\par
			Максимизируя приведённое выражение при условии $s_1 + s_2 + s_3 + s_4 = 4s,$ получим, что наибольшее значение достигается при симметричном варианте $s_1 = s_2 = s_3 = s_4 = s.$
			\begin{table}[H]
				\centering
				\begin{tabular}{|c|c|c|}
					\hline
					$s$ & $n$ & Число новых конъюнкций\\
					\hline
					1 & 6 & 17\\
					\hline
					2 & 10 & 565\\
					\hline
					3 & 14 & 13217\\
					\hline
				\end{tabular}
				\caption*{Примеры для $s = 1, 2, 3$}
			\end{table}\par
			Обозначим
			$$
				f(\overline{s}) = s_1s_2 + s_1(2^{s_2} - 1)(2^{s_3} - 1) + (2^{s_1} - 1)s_2(2^{s_4} - 1) + s_3s_4 + (2^{s_1} - 1)s_3(2^{s_4} - 1) + (2^{s_2} - 1)(2^{s_3} - 1)s_4 + $$ $$ +s_1s_3 + s_1(2^{s_2} - 1)(2^{s_3} - 1) + (2^{s_1} - 1)s_3(2^{s_4} - 1) + s_2s_4 + (2^{s_1} - 1)s_2(2^{s_4} - 1) + (2^{s_2} - 1)(2^{s_3} - 1)s_4 +$$ $$ +5(2^{s_1} - 1)  (2^{s_2} - 1) (2^{s_3} - 1)  (2^{s_4} - 1).
			$$\par
			$f$ --- гладкая функция.\par
			Поставим задачу оптимизации:
			$$
				\begin{cases}
					f(\overline{s}) \rightarrow \max\\
					s_1 + s_2 + s_3 + s_4 = 4s
				\end{cases} (2)
			$$\par
			Функция Лагранжа имеет вид
			$$
				\mathcal{L}(\overline{s};\lambda) = f(\overline{s}) - \lambda(s_1 + s_2 + s_3 + s_4 - 4s).
			$$
			$$
				\begin{cases}
					\frac{\partial\mathcal{L}}{\partial s_1} = s_2 + s_3 + 2(2^{s_2} - 1)(2^{s_3} - 1) + \ln(2) 2^{s_1}(2^{s_4} - 1)(2(s_2 + s_3) + 5(2^{s_2} - 1)(2^{s_3} - 1)) - \lambda= 0\\
					\frac{\partial\mathcal{L}}{\partial s_2} = \frac{\partial\mathcal{L}}{\partial s_3} = \frac{\partial\mathcal{L}}{\partial s_4} = 0 \text{ (имеют аналогичный вид)}\\
					s_1 + s_2 + s_3 + s_4 = 4s
				\end{cases}(3)
			$$\par
			Если положить
			$$
				\begin{cases}
					s_1 = s_2 = s_3 = s_4 = s\\
					\lambda = 2s + 2(2^s - 1)^2 + \ln(2)2^s(2^s - 1)(4s + 5(2^s - 1)^2)
				\end{cases},
			$$
			то это решение удовлетворяет системе (3): все уравнения вида $\frac{\partial\mathcal{L}}{\partial s_i},i=\overline{1,4},$ при равенстве $s$ совпадают и дают выражение для $\lambda$. Значит, представленные $(\overline{s}, \lambda)$ --- седловая точка функции Лагранжа и, соответственно, $\overline{s}$ --- решение (2).
			$$
				f(s,s,s,s) = 4s^2 + 8s(2^s - 1)^2 + 5(2^s - 1)^4
			$$\par
			Перейдём к рассмотрению более общего случая $\mathcal{D}:$
			$$
				\bigvee_{i = 1}^{t_1}x_1A_i \vee \bigvee_{i = 1}^{t_2}\bar{x}_1B_i \vee \bigvee_{i = 1}^{t_3}x_2C_i \vee \bigvee_{i = 1}^{t_4}\bar{x}_2D_i \vee \bigvee_{i = 1}^{s_1}x_1x_2E_i \vee \bigvee_{i = 1}^{s_2}x_1\bar{x}_2F_i \vee \bigvee_{i = 1}^{s_3}\bar{x}_1x_2G_i \vee \bigvee_{i = 1}^{s_4}\bar{x}_1\bar{x}_2H_i \vee \bigvee_iI_i.
			$$\par
			Будем исследовать число новых конъюнкций, которые не исследованы в предыдущем случае, т.е. таких, в которые входят переменные из первых четырёх групп.\par
			Поскольку ранее было доказано, что образуются все конъюнкции вида
			$$%\begin{equation}
				x_i^{\sigma} P \underbrace{E_1\ldots E_{s_E}}_{1\leqslant s_E \leqslant s_1}\underbrace{F_1\ldots F_{s_F}}_{1\leqslant s_F \leqslant s_2}\underbrace{G_1\ldots G_{s_G}}_{1\leqslant s_G \leqslant s_3}\underbrace{H_1\ldots H_{s_H}}_{1\leqslant s_H \leqslant s_4}, i\in\{1,2\},\sigma\in\{0,1\}, P \in \{A, B, C, D\},
			$$\\%\end{equation}\\
			исследуем те, в которые не входят все 4 буквы $E, F, G, H.$\par
			Можно выписать вид таких конъюнкций:
% это ручками исходя из таблицы 1
			$$
			A\begin{cases}
\qquad\qquad\qquad\qquad x_2AG\quad \bar{x}_2AH\quad \bar{x}_1AGH\\
\begin{matrix}x_1\\x_2\end{matrix} A\underbrace{F_1\ldots F_{s_F}}_{1 \leqslant s_F \leqslant s_2}\underbrace{G_1\ldots G_{s_G}}_{1 \leqslant s_G \leqslant s_3} \quad
 \begin{matrix}x_1\\\bar{x}_2\end{matrix} A\underbrace{E_1\ldots E_{s_E}}_{1 \leqslant s_E \leqslant s_1}\underbrace{H_1\ldots H_{s_H}}_{1 \leqslant s_H \leqslant s_4} \\
 \begin{matrix}\bar{x}_1\\x_2\end{matrix} A\underbrace{E_1\ldots E_{s_E}}_{1 \leqslant s_E \leqslant s_1}G\underbrace{H_1\ldots H_{s_H}}_{1 \leqslant s_H \leqslant s_4} \quad
\begin{matrix}\bar{x}_1\\\bar{x}_2\end{matrix} A\underbrace{F_1\ldots F_{s_F}}_{1 \leqslant s_F \leqslant s_2}\underbrace{G_1\ldots G_{s_G}}_{1 \leqslant s_G \leqslant s_3}H
			\end{cases}
			$$
			$$
			B\begin{cases}
\qquad\qquad\qquad\qquad\quad x_2BE\quad \bar{x}_2BF\quad x_1BEF\\
\begin{matrix}x_1\\x_2\end{matrix} BE\underbrace{F_1\ldots F_{s_F}}_{1 \leqslant s_F \leqslant s_2}\underbrace{G_1\ldots G_{s_G}}_{1 \leqslant s_G \leqslant s_3} \quad
 \begin{matrix}x_1\\\bar{x}_2\end{matrix} B\underbrace{E_1\ldots E_{s_E}}_{1 \leqslant s_E \leqslant s_1}F\underbrace{H_1\ldots H_{s_H}}_{1 \leqslant s_H \leqslant s_4} \\
 \begin{matrix}\bar{x}_1\\x_2\end{matrix} B\underbrace{E_1\ldots E_{s_E}}_{1 \leqslant s_E \leqslant s_1}\underbrace{H_1\ldots H_{s_H}}_{1 \leqslant s_H \leqslant s_4} \quad
\begin{matrix}\bar{x}_1\\\bar{x}_2\end{matrix} B\underbrace{F_1\ldots F_{s_F}}_{1 \leqslant s_F \leqslant s_2}\underbrace{G_1\ldots G_{s_G}}_{1 \leqslant s_G \leqslant s_3}
			\end{cases}
			$$
			$$
			C\begin{cases}
\qquad\qquad\qquad\qquad x_1CF\quad \bar{x}_1CH\quad \bar{x}_2CFH\\
\begin{matrix}x_1\\x_2\end{matrix} C\underbrace{F_1\ldots F_{s_F}}_{1 \leqslant s_F \leqslant s_2}\underbrace{G_1\ldots G_{s_G}}_{1 \leqslant s_G \leqslant s_3} \quad
 \begin{matrix}x_1\\\bar{x}_2\end{matrix} C\underbrace{E_1\ldots E_{s_E}}_{1 \leqslant s_E \leqslant s_1}F\underbrace{H_1\ldots H_{s_H}}_{1 \leqslant s_H \leqslant s_4} \\
 \begin{matrix}\bar{x}_1\\x_2\end{matrix} C\underbrace{E_1\ldots E_{s_E}}_{1 \leqslant s_E \leqslant s_1}\underbrace{H_1\ldots H_{s_H}}_{1 \leqslant s_H \leqslant s_4} \quad
\begin{matrix}\bar{x}_1\\\bar{x}_2\end{matrix} C\underbrace{F_1\ldots F_{s_F}}_{1 \leqslant s_F \leqslant s_2}\underbrace{G_1\ldots G_{s_G}}_{1 \leqslant s_G \leqslant s_3}H
			\end{cases}
			$$
			$$
			D\begin{cases}
\qquad\qquad\qquad\qquad\quad x_1DE\quad \bar{x}_1DG\quad x_2DEG\\
\begin{matrix}x_1\\x_2\end{matrix} DE\underbrace{F_1\ldots F_{s_F}}_{1 \leqslant s_F \leqslant s_2}\underbrace{G_1\ldots G_{s_G}}_{1 \leqslant s_G \leqslant s_3} \quad
 \begin{matrix}x_1\\\bar{x}_2\end{matrix} D\underbrace{E_1\ldots E_{s_E}}_{1 \leqslant s_E \leqslant s_1}\underbrace{H_1\ldots H_{s_H}}_{1 \leqslant s_H \leqslant s_4} \\
 \begin{matrix}\bar{x}_1\\x_2\end{matrix} D\underbrace{E_1\ldots E_{s_E}}_{1 \leqslant s_E \leqslant s_1}G\underbrace{H_1\ldots H_{s_H}}_{1 \leqslant s_H \leqslant s_4} \quad
\begin{matrix}\bar{x}_1\\\bar{x}_2\end{matrix} D\underbrace{F_1\ldots F_{s_F}}_{1 \leqslant s_F \leqslant s_2}\underbrace{G_1\ldots G_{s_G}}_{1 \leqslant s_G \leqslant s_3}
			\end{cases}
			$$\par
			Подсчитаем число новых конъюнкций вместе с теми, которые содержат только $E, F, G, H,$ c участием переменных $x_1, x_2$:
			$$
				f(\overline{t}, \overline{s}) = t_1\big(s_3 + s_4 + s_3s_4 + 2((2^{s_2} - 1)(2^{s_3} - 1) + (2^{s_1} - 1)(2^{s_4} - 1) + (2^{s_1} - 1)s_3(2^{s_4} - 1) + (2^{s_2} - 1)(2^{s_3} - 1)s_4) + $$ $$ + 4(2^{s_1} - 1)(2^{s_2} - 1)(2^{s_3} - 1)(2^{s_4} - 1)\big) +
			$$
			$$
				+ t_2\big(s_1 + s_2 + s_1s_2 + 2(s_1(2^{s_2} - 1)(2^{s_3} - 1) + (2^{s_1} - 1)s_2(2^{s_4} - 1) + (2^{s_1} - 1)(2^{s_4} - 1) + (2^{s_2} - 1)(2^{s_3} - 1)) + $$ $$ + 4(2^{s_1} - 1)(2^{s_2} - 1)(2^{s_3} - 1)(2^{s_4} - 1)\big) +
			$$
			$$
				+ t_3\big(s_2 + s_4 + s_2s_4 + 2((2^{s_2} - 1)(2^{s_3} - 1) + (2^{s_1} - 1)s_2(2^{s_4} - 1) + (2^{s_1} - 1)(2^{s_4} - 1) + (2^{s_2} - 1)(2^{s_3} - 1)s_4) + $$ $$ + 4(2^{s_1} - 1)(2^{s_2} - 1)(2^{s_3} - 1)(2^{s_4} - 1)\big) +
			$$
			$$
				+ t_4\big(s_1 + s_3 + s_1s_3 + 2(s_1(2^{s_2} - 1)(2^{s_3} - 1) + (2^{s_1} - 1)(2^{s_4} - 1) + (2^{s_1} - 1)s_3(2^{s_4} - 1) + (2^{s_2} - 1)(2^{s_3} - 1)) + $$ $$ + 4(2^{s_1} - 1)(2^{s_2} - 1)(2^{s_3} - 1)(2^{s_4} - 1)\big) +
			$$
			$$
			+ \big(s_1s_2 + s_1(2^{s_2} - 1)(2^{s_3} - 1) + (2^{s_1} - 1)s_2(2^{s_4} - 1) + s_3s_4 + (2^{s_1} - 1)s_3(2^{s_4} - 1) + (2^{s_2} - 1)(2^{s_3} - 1)s_4 + $$ $$ +s_1s_3 + s_1(2^{s_2} - 1)(2^{s_3} - 1) + (2^{s_1} - 1)s_3(2^{s_4} - 1) + s_2s_4 + (2^{s_1} - 1)s_2(2^{s_4} - 1) + (2^{s_2} - 1)(2^{s_3} - 1)s_4 +$$ $$ +4(2^{s_1} - 1)  (2^{s_2} - 1) (2^{s_3} - 1)  (2^{s_4} - 1)\big)
$$\par
		Выпишем вид новых образованных конъюнкций с участием $A, B, C, D$, в которые не входят переменные $x_1, x_2:$
		%$$
		%	\underbrace{E_1\ldots E_{s_E}}_{1\leqslant s_E \leqslant s_1}\underbrace{F_1\ldots F_{s_F}}_{1\leqslant s_F \leqslant s_2}\underbrace{G_1\ldots G_{s_G}}_{1\leqslant s_G \leqslant s_3}\underbrace{H_1\ldots H_{s_H}}_{1\leqslant s_H \leqslant s_4}
		%$$
		$$
			PQ\underbrace{E_1\ldots E_{s_E}}_{1\leqslant s_E \leqslant s_1}\underbrace{F_1\ldots F_{s_F}}_{1\leqslant s_F \leqslant s_2}\underbrace{G_1\ldots G_{s_G}}_{1\leqslant s_G \leqslant s_3}\underbrace{H_1\ldots H_{s_H}}_{1\leqslant s_H \leqslant s_4},\,\,P,Q\in\{A, B, C, D \}
		$$
		%\vspace{1pt}
		$$
			AB\quad CD
		$$
		$$
			ACH\quad ADG\quad BCF \quad BDE
		$$
		$$
			AAGH\quad BBEF\quad CCFH\quad DDEG
		$$
		$$
			AA\underbrace{F_1\ldots F_{s_F}}_{1\leqslant s_F \leqslant s_2}\underbrace{G_1\ldots G_{s_G}}_{1\leqslant s_G \leqslant s_3}H\quad
AA\underbrace{E_1\ldots E_{s_E}}_{1\leqslant s_E \leqslant s_1}G\underbrace{H_1\ldots H_{s_H}}_{1\leqslant s_H \leqslant s_4}
		$$
		$$
			AB\underbrace{F_1\ldots F_{s_F}}_{1\leqslant s_F \leqslant s_2}\underbrace{G_1\ldots G_{s_G}}_{1\leqslant s_G \leqslant s_3}\quad
AB\underbrace{E_1\ldots E_{s_E}}_{1\leqslant s_E \leqslant s_1}\underbrace{H_1\ldots H_{s_H}}_{1\leqslant s_H \leqslant s_4}
		$$
		$$
			AC\underbrace{F_1\ldots F_{s_F}}_{1\leqslant s_F \leqslant s_2}\underbrace{G_1\ldots G_{s_G}}_{1\leqslant s_G \leqslant s_3}H\quad AC\underbrace{E_1\ldots E_{s_E}}_{1\leqslant s_E \leqslant s_1}\underbrace{H_1\ldots H_{s_H}}_{1\leqslant s_H \leqslant s_4}
		$$
		$$
			AD\underbrace{F_1\ldots F_{s_F}}_{1\leqslant s_F \leqslant s_2}\underbrace{G_1\ldots G_{s_G}}_{1\leqslant s_G \leqslant s_3}\quad AD\underbrace{E_1\ldots E_{s_E}}_{1\leqslant s_E \leqslant s_1}G\underbrace{H_1\ldots H_{s_H}}_{1\leqslant s_H \leqslant s_4}
		$$
		$$
			BBE\underbrace{F_1\ldots F_{s_F}}_{1\leqslant s_F \leqslant s_2}\underbrace{G_1\ldots G_{s_G}}_{1\leqslant s_G \leqslant s_3}\quad BB\underbrace{E_1\ldots E_{s_E}}_{1\leqslant s_E \leqslant s_1}F\underbrace{H_1\ldots H_{s_H}}_{1\leqslant s_H \leqslant s_4}
		$$
		$$
			BC\underbrace{F_1\ldots F_{s_F}}_{1\leqslant s_F \leqslant s_2}\underbrace{G_1\ldots G_{s_G}}_{1\leqslant s_G \leqslant s_3}\quad BC\underbrace{E_1\ldots E_{s_E}}_{1\leqslant s_E \leqslant s_1}F\underbrace{H_1\ldots H_{s_H}}_{1\leqslant s_H \leqslant s_4}
		$$
		$$
BDE\underbrace{F_1\ldots F_{s_F}}_{1\leqslant s_F \leqslant s_2}\underbrace{G_1\ldots G_{s_G}}_{1\leqslant s_G \leqslant s_3}\quad BD\underbrace{E_1\ldots E_{s_E}}_{1\leqslant s_E \leqslant s_1}\underbrace{H_1\ldots H_{s_H}}_{1\leqslant s_H \leqslant s_4}
		$$
		$$
			CC\underbrace{F_1\ldots F_{s_F}}_{1\leqslant s_F \leqslant s_2}\underbrace{G_1\ldots G_{s_G}}_{1\leqslant s_G \leqslant s_3}H\quad CC\underbrace{E_1\ldots E_{s_E}}_{1\leqslant s_E \leqslant s_1}F\underbrace{H_1\ldots H_{s_H}}_{1\leqslant s_H \leqslant s_4}
		$$
		$$
			CD\underbrace{F_1\ldots F_{s_F}}_{1\leqslant s_F \leqslant s_2}\underbrace{G_1\ldots G_{s_G}}_{1\leqslant s_G \leqslant s_3}\quad CD\underbrace{E_1\ldots E_{s_E}}_{1\leqslant s_E \leqslant s_1}\underbrace{H_1\ldots H_{s_H}}_{1\leqslant s_H \leqslant s_4}
		$$
		$$
			DDE\underbrace{F_1\ldots F_{s_F}}_{1\leqslant s_F \leqslant s_2}\underbrace{G_1\ldots G_{s_G}}_{1\leqslant s_G \leqslant s_3}\quad DD\underbrace{E_1\ldots E_{s_E}}_{1\leqslant s_E \leqslant s_1}G\underbrace{H_1\ldots H_{s_H}}_{1\leqslant s_H \leqslant s_4}
		$$\par
		Подсчитаем их число вместе с теми, которые содержат только $E, F, G, H$:
		$$
			\left(\frac{1}{2} (t_1 + t_2 + t_3 + t_4)(t_1 + t_2 + t_3 + t_4 + 1) + 1\right)(2^{s_1} - 1)(2^{s_2} - 1)(2^{s_3} - 1)(2^{s_4} - 1) + t_1t_2 + t_3t_4 + 
		$$
		$$
			+ t_1t_3s_4 + t_2t_4s_3 + t_2t_3s_2 + t_2t_4s_1 + \frac{1}{2}t_1(t_1 + 1)s_3s_4 + \frac{1}{2}t_2(t_2 + 1)s_1s_2 + \frac{1}{2}t_3(t_3 + 1)s_2s_4 +\frac{1}{2}t_4(t_4 + 1)s_1s_3 +
		$$
		$$
			+ (2^{s_2} - 1)(2^{s_3} - 1) \left (\frac{1}{2}t_1(t_1 + 1)s_4 + t_1t_2 + t_1t_3s_4 + t_1t_4 +\frac{1}{2}t_2(t_2 + 1)s_1 + \right.$$ $$ \left.+ t_2t_3 + t_2t_4s_1 + \frac{1}{2}t_3(t_3 + 1)s_4 + t_3t_4 + \frac{1}{2}t_4(t_4 + 1)s_1\right) +
		$$
		$$
			+ (2^{s_1} - 1)(2^{s_4} - 1) \left (\frac{1}{2}t_1(t_1 + 1)s_3 + t_1t_2 + t_1t_3 + t_1t_4s_3 + \frac{1}{2}t_2(t_2 + 1)s_2 + \right.$$ $$ \left.+ t_2t_3s_2 + t_2t_4 + \frac{1}{2}t_3(t_3 + 1)s_2 + t_3t_4 + \frac{1}{2}t_4(t_4 + 1)s_3\right)
		$$
		% здесь был случай трёх переменных
		\subsubsection*{Условия на образование единичной конъюнкции}
		В случае двух переменных в ходе метода Блейка для ДНФ $\mathcal{D}$ правило Блейка может быть применено к парам конъюнкций $x_1A \vee \bar{x}_1B$ и $x_2A \vee \bar{x}_2B.$ Выпишем некоторые достаточные условия образования пары конъюнкций $x_1 \vee \bar{x}_1.$\par
		Напомним общий вид $\mathcal{D}$:
			$$
				\bigvee_{i = 1}^{t_1}x_1A_i \vee \bigvee_{i = 1}^{t_2}\bar{x}_1B_i \vee \bigvee_{i = 1}^{t_3}x_2C_i \vee \bigvee_{i = 1}^{t_4}\bar{x}_2D_i \vee \bigvee_{i = 1}^{s_1}x_1x_2E_i \vee \bigvee_{i = 1}^{s_2}x_1\bar{x}_2F_i \vee \bigvee_{i = 1}^{s_3}\bar{x}_1x_2G_i \vee \bigvee_{i = 1}^{s_4}\bar{x}_1\bar{x}_2H_i \vee \bigvee_iI_i.
			$$\par
		\begin{minipage}[t]{0.5\textwidth}
		Условия для образования конъюнкции $x_1$:
		\begin{enumerate}
			\item
			$\exists A_i = 1$
			\item
			$\exists C_i = 1$ и $\exists F_i = 1$
			\item
			$\exists D_i = 1$ и $\exists E_i = 1$
			\item
			$\exists E_i = 1$ и $\exists F_i = 1$\vspace{10pt}
		\end{enumerate}
		\end{minipage}
		\hfill
		\begin{minipage}[t]{0.4\textwidth}
		Аналогично для $\bar{x}_1$:
		\begin{enumerate}
			\item
			$\exists B_i = 1$
			\item
			$\exists C_i = 1$ и $\exists H_i = 1$
			\item
			$\exists D_i = 1$ и $\exists G_i = 1$
			\item
			$\exists G_i = 1$ и $\exists H_i = 1$
		\end{enumerate}
		\end{minipage}\par
		Остальные условия (т.е. случай образования $x_1$ или $\bar{x}_1$ после цепочки из 2-х применений правила Блейка) избыточны, т.к. включают в себя хотя бы одно из перечисленных выше.\par
		Таким образом достаточно провести только первые 2 этапа метода Блейка, чтобы понять, будет ли образована единичная конъюнкция.
		\subsection{Поглощение в методе Блейка}
			После того, как правило Блейка больше неприменимо, в методе Блейка необходимо выполнить все поглощения ($K_1 \vee K_1K_2 = K_1$).\par
			Для рассмотренных выше видах ДНФ с двумя переменными в различных степенях проверим, какие из конъюнкций точно будут поглощены.\par
			$1^{\circ}.$ Рассмотрим ДНФ, общий вид которой выглядит следующим образом:
			$$
				\mathcal{D} = \bigvee_{i = 1}^{s_1}x_1x_2E_i \vee \bigvee_{i = 1}^{s_2}x_1\bar{x}_2F_i \vee \bigvee_{i = 1}^{s_3}\bar{x}_1x_2G_i \vee \bigvee_{i = 1}^{s_4}\bar{x}_1\bar{x}_2H_i \vee \bigvee_iI_i.
			$$\par
			То, как выглядят полученные применением правила Блейка конъюнкции, подробно описано выше (см. стр. 18-19). Среди них конъюнкциями
			$$x_1EF\quad \bar{x}_1GH \quad x_2EG \quad \bar{x}_2FH \quad EFGH$$
			поглотятся все остальные новые образованные конъюнкции, поскольку буквы в их записи будут являться подмножествами букв конъюнкций большего ранга. Сами они ничем другим поглотиться не могут.\par
			Исходя из этого нетрудно подсчитать число новообразованных конъюнкций, которые не будут поглощены в ходе метода Блейка:
			$$
				f_{new}(\bar{s}) = s_1s_2 + s_1s_3 + s_2s_4 + s_3s_4 + s_1s_2s_3s_4.
			$$\par
			Её максимум также достигается в случае равенства $s_1 = s_2 = s_3 = s_4.$
\par
			\begin{table}[H]
				\centering
				\begin{tabular}{|c|c|c|}
					\hline
					$s$ & $n$ & Число новых конъюнкций\\
					\hline
					1 & 6 & 5\\
					\hline
					2 & 10 & 32\\
					\hline
					3 & 14 & 117\\
					\hline
				\end{tabular}
				\caption*{Примеры для $s = 1, 2, 3$}
			\end{table}\par
			$2^{\circ}.$ Более общий случай:
			$$
				\bigvee_{i = 1}^{t_1}x_1A_i \vee \bigvee_{i = 1}^{t_2}\bar{x}_1B_i \vee \bigvee_{i = 1}^{t_3}x_2C_i \vee \bigvee_{i = 1}^{t_4}\bar{x}_2D_i \bigvee_{i = 1}^{s_1}x_1x_2E_i \vee \bigvee_{i = 1}^{s_2}x_1\bar{x}_2F_i \vee \bigvee_{i = 1}^{s_3}\bar{x}_1x_2G_i \vee \bigvee_{i = 1}^{s_4}\bar{x}_1\bar{x}_2H_i \vee \bigvee_iI_i.
			$$\par
			Описание новых образованных конъюнкций см. на стр. 21-23.\par
			Здесь все остальные образованные конъюнкции поглотят конъюнкции вида
			$$
				x_1EF\quad \bar{x}_1GH\quad  x_2EG\quad  \bar{x}_2FH
			$$
			$$
				x_1CF\quad  x_1DE\quad 
				\bar{x}_1CH\quad  \bar{x}_1DG$$ $$
				x_2AG\quad  x_2BE\quad 
				\bar{x}_2AH\quad  \bar{x}_2BF
			$$
			$$
				EFGH\quad  AB\quad  CD$$
			$$
				ACH\quad  ADG\quad  BCF\quad  BDE
			$$
			$$
				AGH\quad  BEF\quad  CFH\quad  DEG
			$$\par
			Подсчитаем их число:
			$$
				f_{new}(\bar{t}, \bar{s}) = s_1s_2 + s_1s_3 + s_2s_4 + s_3s_4 + s_1s_2s_3s_4 +
			$$
			$$ 
				t_1(s_3 + s_4 + s_3s_4) + t_2(s_1 + s_2 + s_1s_2) + t_3(s_2 + s_4 + s_2s_4) + t_4(s_1 + s_3 + s_1s_3) + 
			$$
			$$
				+ t_1t_2 + t_3t_4 + t_1t_3s_4 + t_1t_4s_3 + t_2t_3s_2 + t_2t_4s_1.
			$$\par
			При этом максимум достигается в случае $t_1 = t_2 = t_3 = t_4 = 0, s_1=s_2=s_3=s_4,$ что соответствует описанному выше случаю $1^{\circ}.$
		\newpage
	\section{Сравнение методов поточечного покрытия и критерия поглощения}
		Пусть необходимо проверить, поглощается ли конъюнкция $K$ ДНФ $\mathcal{D} = \bigvee\limits_{i = 1}^r K_i.$\par
		Пусть $n$ --- число переменных, $q$ --- ранг конъюнкции $K.$\\
		\textbf{Шаг 1.} Определим окрестность 1-го порядка конъюнкции $K$: $S_1(K, \mathcal{D}) = \{K_i\}.$\par
		Для выполнения этой проверки достаточно $\underline{O}(rn)$ операций.\par
		Пусть $|S_1(K, \mathcal{D})| = m$.\\
		\textbf{Шаг 2.} Определим, образуют ли полученные $K_i$ покрытие $K.$ Возможны 2 способа:\par
		\begin{enumerate}
			\item
			Для каждой конъюнкции $K_i$ отмечаем покрываемые ею точки интервала, соответствующего $K.$ Если все точки этого интервала оказались покрыты, то конъюнкция $K$ поглощается, иначе удалить её нельзя.\par
			Сложность этой операции не больше $\underline{O}(rn2^{n-q})$.
			\item
			Применяем критерий поглощения и с помощью метода Блейка проверяем, верно ли, что $\bigvee\limits_{i = 1}^m K'_i \equiv 1.$ Если да, то согласно критерию поглощения конъюнкция $K$ покрывается $\mathcal{D},$ иначе нет.\par
			%Метод Блейка имеет экспоненциальную сложность по $n.$
		\end{enumerate}\par
		%Необходимо принять решение о том, как проверять факт поглощения: прямым поточечным покрытием или с помощью критерия поглощения.
		%\begin{enumerate}
		%	\item
		%	Оцениваем число операций, необходимых для проверки прямым покрытием:
		%	$$
		%		\sum_{i = 1}^r 2^{n-\text{rg}(K \& K_i)}
		%	$$
		%	\item
		%	Оцениваем число конъюнкций, образуемых в ходе метода Блейка для $\bigvee\limits_{i = 1}^m K'_i$.
		%	\item
		%	По числу конъюнкций оцениваем число операций в методе Блейка.
		%\end{enumerate}\par
		Из сказанного выше видно, что применение метода Блейка оказывается эффективным в небольшом ряде случаев, но в некоторых ситуациях он оптимален.
		\begin{example}[Метод Блейка оптимален]
			\hspace{5pt}\parПусть заданы $n = 15$, $K = x_3\bar{x}_4$ и 
			$$\mathcal{D} = x_1x_3 \vee x_2x_3 \vee \bigvee\limits_{i = 5}^{10}\bar{x}_1x_i \vee \bigvee\limits_{i = 11}^{15}\bar{x}_2x_i.$$\par
			В ходе критерия поглощения получаем ДНФ
			$$\mathcal{D}' = x_1 \vee x_2 \vee \bigvee\limits_{i = 5}^{10}\bar{x}_1x_i \vee \bigvee\limits_{i = 11}^{15}\bar{x}_2x_i.$$\\
			В ней в обеих степенях встречаются только переменные $x_1$ и $x_2,$ а в ходе применения правила Блейка образуется 11 новых конъюнкций:
			$$
				\mathcal{D}'' = x_1 \vee x_2 \vee \bigvee\limits_{i = 5}^{10}\bar{x}_1x_i \vee \bigvee\limits_{i = 11}^{15}\bar{x}_2x_i \vee \bigvee\limits_{i = 5}^{15}x_i.
			$$\par
			При поточечном покрытии было бы необходимо перебрать $2\times 2^{13} + {11}\times 2^{12}$ точек, что на порядок превосходит число операций предыдущего метода.
		\end{example}
		\begin{example}[Прямое покрытие оптимально]
			\hspace{5pt}\par Пусть $n = 6$, $K = x_3\bar{x}_4$ и 
			$$
				\mathcal{D} = \bigvee\limits_{i = 3}^6x_1x_2x_i \vee \bigvee\limits_{i = 3}^6x_1\bar{x}_2x_i \vee \bigvee\limits_{i = 3}^6\bar{x}_1x_2x_i \vee \bigvee\limits_{i = 4}^6\bar{x}_1\bar{x}_2x_i.
			$$\par
			В ходе критерия поглощения получаем ДНФ
			$$
				\mathcal{D}' = x_1x_2x \vee x_1\bar{x}_2 \vee \bar{x}_1x_2 \vee \bigvee\limits_{i = 4}^6(x_1x_2x_i \vee x_1\bar{x}_2x_i \vee \bar{x}_1x_2x_i \vee \bar{x}_1\bar{x}_2x_i).
			$$\par
			В ходе применения метода Блейка к $\mathcal{D}'$ образуется не менее $(2^3-1)^4$ новых конъюнкций.\par
			При поточечном покрытии было бы необходимо перебрать $12 \times 2 + 3 \times 2^2$ точек, что значительно меньше.
		\end{example}
		\newpage
	\section{Поиск нуля функции}
		При проверке поглощения конъюнкции с помощью критерия поглощения возникает задача установить, равна ли полученная ДНФ тождественной единице. Если удаётся найти хотя бы один ноль соответствующей этой ДНФ функции, то это означает, что функция тождественной единицей не является. Таким образом проверка выполнена, и рассматриваемая конъюнкция не поглощается своей окрестностью 1-го порядка.\par
		Задача точного установления наличия нуля булевой функции, по-видимому, является сложной. Предложим несколько способов перебора точек функции, которые в некоторых случаях приводят к быстрому обнаружению нулевой точки.\par
		Пусть заданы $n$ --- число переменных, конъюнкция $K = x_1\ldots x_q$ ранга $q$ (с помощью замены переменных всегда можно добиться представления в таком виде) и ДНФ $\bigvee\limits_iK'_i.$\par
		В ходе применения критерия поглощения после удаления букв, входящих в конъюнкцию $K$, образована ДНФ $\mathcal{D} = \bigvee\limits_{i = 1}^rK_i,$ в которую входят $2n - q$ различных букв: $x_{q + 1},\ldots, x_n, \bar{x}_1,\ldots,\bar{x}_n$. Требуется установить, верно ли, что $\mathcal{D} \equiv 1.$\par
		Поскольку в $\mathcal{D}$ переменные $x_1,\ldots, x_q$ могут входить только в степени ноль, то их значение сразу можно положить равным 1, что сразу обнулит какую-то часть конъюнкций. Сосредоточимся на определении значения переменных $x_{q + 1},\ldots, x_n$ и обнулении оставшихся конъюнкций.\par
		$1^{\circ}.$ Рассмотрим следующий алгоритм. Для каждой из $r$ конъюнкций $K$ ДНФ $\mathcal{D}$ будем определять множество остальных конъюнкций, которые обнуляются при обнулении одной из букв $K.$ Если обнуление некоторого множества букв $K$ приводит к обнулению всех остальных конъюнкций, то таким образом найден ноль функции и алгоритм можно завершить.\par
		\begin{algorithm}[H]
				\SetAlgoLined
				\KwData{$\mathcal{D} = \{K_i\}_{i = 1}^r$ --- набор конъюнкций ДНФ $\mathcal{D}$}
		 		\KwResult{$(\alpha_1,\ldots,\alpha_n)$ --- ноль ДНФ, или сообщение о том, что ноль не найден}
				\For{$i = \overline{1,\,r}$}{%\textnormal{каждой $K \in \mathcal{D}$}}{
					$X \mathbin{:=} \varnothing$ --- множество остальных конъюнкций, обращающихся в ноль при обнулении переменных рассматриваемой конъюнкции $K_i = x_{i_1}^{\sigma_1}\ldots x_{i_k}^{\sigma_k}$ ранга $k$\;
					$\tilde \alpha^n \mathbin{:=} (0, \ldots, 0)$ --- искомый ноль\;
					\For{$j = \overline{1,\,k}$}{
						$C_j \mathbin{:=} \{\text{множество конъюнкций, содержащих букву $x_{i_j}^{\sigma_j}$}\}$\;
						$X \mathbin{:=} X \cup C_j$\;
						$\alpha_{i_j} \mathbin{:=} \bar{\sigma}_j$\;
					}
					\If{$X = \mathcal{D},$}{
						все конъюнкции обнулены и нулевая точка найдена: \textbf{вернуть} $\tilde \alpha^n$\;
					}
				}
				\textbf{Вернуть} сообщение о том, что ноль не найден\;
			\caption{Поиск нуля функции}
			\label{all_conj}
		\end{algorithm}\par
		Разумеется, далеко не всегда рассматриваемая конъюнкция будет иметь большое пересечение по буквам с остальными конъюнкциями ДНФ.\par
		\begin{example}
				Рассмотрим ДНФ, уже приведённую в качестве примера в \ref{jab}:
				$$
					\mathcal{D} = x_1\bar{x}_2 \vee x_2\bar{x}_3 \vee \ldots \vee x_n \bar{x}_1.
				$$\par
				В ней каждая буква встречается ровно один раз, и обнуление букв одной конъюнкции не приводит к обнулению никакой другой.
		\end{example}\par
		% Для каждого алгоритма привести три примера: удачный (ноль найден), неудачный (ноль не найден, хотя он есть) и неопределённый (не найден ноль, его нет).
		$2^{\circ}.$ Предложим жадный алгоритм, основанный на частоте встречаемости букв в записи ДНФ $\mathcal{D}.$\par
		Для каждой буквы $x_1,\ldots, x_n,\bar{x}_1,\ldots,\bar{x}_n$ определим конъюнкции, в которые они входят, и упорядочим их убыванию встречаемости. Таким образом если первой оказалась буква $x_i^{\sigma_i},$ то мы положим значение переменной $x_i$ равным $\bar{\sigma}_i,$ что обнулит максимальное число конъюнкций. Заметим, что конъюнкции, содержание букву $x_i^{\bar{\sigma}_i},$ уже не могут быть обращены в ноль с помощью этой переменной.\par
		Формализуем процесс установления значения переменных и вычёркивания обнулённых конъюнкций.\par
		Для каждой ДНФ строится матрица $M \in \{0, 1, \Delta\}^{r \times n},$ где каждая строка $M_i,$\\$i=\overline{1,\,r},$ представляет собой запись конъюнкции $K_i$:
			$$
				M_{ij} = \begin{cases}
					0,\text{ если $K_i$ содержит букву $\bar{x}_j$,}\\
					1,\text{ если $K_i$ содержит букву $x_j$,}\\
					\Delta,\text{ иначе.}\\
				\end{cases}
			$$\par
			Когда мы фиксируем значение некоторой переменной $x_j = \sigma,$ из матрицы $M$ вычёркиваются все строки $M_i,$ для которых $M_{ij} = \sigma,$ а также сам $j$-й столбец матрицы. Затем процедура повторяется для следующей переменной.\par
			Порядок обнуления переменных определяется частотой их встречаемости.\par
			Если на каком-то этапе все строки матрицы оказались вычеркнуты, то ноль функции найден. Если в результате остался один константный столбец, целиком состоящий из нулей или из единиц, то полученная точка действительно является нулевой. В противном случае построенная комбинация не обнуляет ДНФ.\par
		\begin{example}
				Рассмотрим ДНФ
				$$
					\mathcal{D} = x_1x_2\bar{x}_3 \vee x_1x_3 \vee \bar{x}_1.
				$$\par
				Соответствующая ей матрица имеет вид:
				%$$
					%\begin{matrix}
					%	1 & 1 & 0\\
					%	1 & \Delta & 1\\
					%	0 & \Delta & \Delta\\
					%\end{matrix}
					%\SmallMatrix{
					%	1 & 1 & 0\\1 & \Delta & 1\\ 0 & \Delta & \Delta\\
					%}
				%$$
				$$
				\begin{tikzpicture}
					\matrix (magic) [%
					matrix of nodes,
					text width=5mm,
					text badly centered
					] {%
						1 & 1 & 0\\
						1 & $\Delta$ & 1\\
						0 & $\Delta$ & $\Delta$\\
					};
 				\end{tikzpicture}
				$$\par
				Здесь самой часто встречающейся буквой является $x_1.$ Положим значение этой переменной равным нулю, это обнулит первые две конъюнкции, что соответствует вычёркиванию первых двух строк матрицы:
				$$
				\begin{tikzpicture}
					\matrix (magic) [%
					matrix of nodes,
					text width=5mm,
					text badly centered
					] {%
						1 & 1 & 0\\
						1 & $\Delta$ & 1\\
						0 & $\Delta$ & $\Delta$\\
					};
					\draw[thick,black] (magic-1-1.north) -- (magic-3-1.south);
					\draw[thick,black] (magic-1-1.west) -- (magic-1-3.east);
					\draw[thick,black] (magic-2-1.west) -- (magic-2-3.east);
 				\end{tikzpicture}
				$$\par
				В результате в последней конъюнкции не осталось незафиксированных букв, которые могли бы обратить её в ноль. Данный алгоритм не смог найти ноль ДНФ $\mathcal{D},$ хотя он существует, например, точка $(1, 0, 0).$
		\end{example}\par
		$3^{\circ}.$ Предложим модификацию алгоритма из предыдущего пункта. Разобьём конъюнкции на группы в соответствии с их рангом. Будем упорядочивать буквы по частоте их встречаемости не среди всех конъюнкций, а среди конъюнкций наименьшего ранга. Если среди них встретились не все переменные, то их порядок определяется встречаемостью конъюнкций на единицу большего ранга и т.д.\par
		Такой подход основывается на эвристике о том, что чем меньше букв содержит конъюнкция, тем <<сложнее>> её обнулить. Например, конъюнкции ранга 1 образуют необходимое условие на значение соответствующей переменной в нулевой точке.\par
		Можно обобщить эту идею, введя весовую схему, которая будет оценивать <<важность встречаемости>> каждой буквы, и затем определить порядок обнуления переменных в соответствии с отсортированным по весам списком букв.\par
\newpage

\section*{Список литературы}
\noindent
$[1]$ {\it Журавлёв Ю. И.} Оценки сложности алгоритмов построения минимальных дизъюнктивных форм для функций алгебры логики. // Сб. Трудов Института математики Сиб. отд. АНСССР. Дискретный Анализ. 1964. Вып. 3.\\
$[2]$ {\it Журавлёв Ю. И., Коган А. Ю.} Реализация булевых функций с малым числом нулей дизъюнктивными нормальными формами и смежные задачи. // Докл. АН СССР. 1985. Т. 285. № 4. С. 795-799.\\
$[3]$ {\it Журавлёв Ю. И.} О различных понятиях минимальности дизъюнктивных нормальных форм. // Сиб. матем. журнал. 1960. Т. 1. № 4. С. 609-610.\\
$[4]$ {\it Журавлёв Ю. И.} Алгоритмы построения минимальных дизъюнктивных нормальных форм для функций алгебры логики. // Дискретная математика и математические вопросы кибернетики. Под редакцией Яблонского С. В., Лупанова О. Б. М.:Наука, 1974\\
$[5]$ {\it Журавлёв Ю. И., Коган А. Ю.} Алгоритм построения дизъюнктивной нормальной формы, эквивалентной произведению левых частей булевых уравнений нельсоновского типа. // Ж. вычисл. матем. и матем. физ. 1986. Т. 26. № 8. С. 1243-1249.\\
$[6]$ {\it Дьяконов А. Г.} Реализация одного класса булевых функций с малым числом нулей дизъюнктивными нормальными формами. // Ж. вычисл. матем. и матем. физ. 2001. Т. 41. № 5. С. 821-828.\\
$[7]$ {\it Дьяконов А. Г.} Построение дизъюнктивных нормальных форм в задачах распознавания образов с бинарной информацией. // Докл. АН. 2002. Т. 383. № 6. С. 747-749.
21\\
$[8]$ {\it Дьяконов А. Г.} Построение дизъюнктивных нормальных форм в логических алгоритмах распознавания. // Ж. вычисл. матем. и матем. физ. 2002. Т. 42. № 12. С. 1899-1907.\\
$[9]$ {\it Коган А. Ю.} О дизъюнктивных нормальных формах булевых функций с малым числом нулей. // Ж. вычисл. матем. и матем. физ. 1987. Т. 27. № 6. С. 924-931.\\
$[10]$ {\it Журавлёв Ю. И.} Об алгебраическом подходе к решению задач распознавания или классификации. // Пробл. кибернетики. М.: Наука, 1978. Вып. 33. С. 5-68\\
$[11]$ {\it Соколов П. Л.} Об оптимальной расшифровке монотонных функций алгебры логики // Ж. вычисл. матем. и матем. физ. 1982. Т. 22. № 2. С. 499-461.\\
$[12]$ Сборник <<Проблемы кибернетики>>, вып. 8, М.; издательство <<Наука>>, 1962.\\
\end{document}







$[1]$ {\it Журавлёв Ю.\,И., Коган А.\,Ю.} Реализация булвых функций с малым числом нулей дизъюнктивными нормальными формами и смежные задачи // Докл. АН СССР. 

$[1]$ {\it Журавлев Ю.\,И., Коган А.\,Ю.} Реализация булевых функций с малым числом нулей дизъюнктивными нормальными формами и смежные задачи // Докл. АН СССР. 1985. Т. 285. No 4. С. 795-799.\\

$[2]$
{\it Журавлев Ю.\,И., Коган А.\,Ю.} Алгоритм построения дизъюнктивной нормальной формы, эквивалент­ ной произведению левых частей булевых уравнений нельсоновского типа // Ж. вычисл. матем. и матем. физ. 1986. Т. 26. No 8. С. 1243-1249.\\

$[3]$
{\it Соколов П.\,Л.} Об оптимальной расшифровке монотонных функций алгебры логики // Ж. вычисл. ма­тем. и матем. физ. 1982. Т. 22. No 2. С. 449-461.\\

$[4]$
{\it Дьяконов А.\,Г.} Реализация одного класса булевых функций с малым числом нулей тупиковыми дизъюнктивными нормальными формами, Ж. вычисл. матем. и матем. физ., 2001, том 41, номер 5, 821–828\\
\end{document}

			%\section{Новые подходы и~результаты}
%Название этого раздела обязательно надо заменить на~содержательное. 
%В~этом разделе, как правило, много подразделов. 

%В~дипломной работе не~стоит делать более двух уровней, достаточно разделов и~подразделов.
%Будете писать диссертацию или монографию~--- сделаете три уровня. 
  
			%\section{Вычислительные эксперименты}

%Цель данного раздела: продемонстрировать, что предложенная теория работает на практике; показать границы её применимости; рассказать о~новых экспериментальных фактах.
%Чисто теоретические работы могут вообще не~содержать раздела экспериментов (не~работает, ну и~не~надо~--- зато теория красивая).
%Кстати, теоретики имеют право не~догадываться, где, кому и~когда их теории пригодятся.

			%\subsection{Исходные данные и~условия эксперимента}
%Описывается прикладная задача, параметры анализируемых данных (например, сколько объектов, сколько признаков, каких они типов),  параметры эксперимента  (например, как производился скользящий контроль). 

			%\subsection{Результаты эксперимента}
%Результаты экспериментов представляются в~виде таблиц и~графиков. 
%Объясняется точный смысл всех обозначений на графиках, строк и~столбцов в~таблицах. 

			%\subsection{Обсуждение и~выводы}
%Приводятся выводы: 
%в~какой степени результаты экспериментов согласуются с~теорией? 
%Достигнут ли желаемый результат? 
%Обнаружены ли какие-либо факты, не~нашедшие объяснения, и~которые нельзя списать на «грязный» эксперимент?

%Обсуждаются основные отличия предложенных методов от известных ранее. 
%В~чем их преимущества? 
%Каковы границы их применимости? 
%Какие проблемы удалось решить, а~какие остались открытыми? 
%Какие возникли новые постановки задач?

\section*{Заключение}
\addcontentsline{toc}{section}{Заключение}

В~квалификационных работах последний раздел нужен для того, чтобы 
конспективно перечислить основные результаты, полученные лично автором. 

Результатами, в~частности, являются:
\begin{itemize}
\item 
    Предложен новый подход к\dots
\item 
    Разработан новый метод\dots, позволяющий\dots
\item 
    Доказан ряд теорем, подтверждающих (опровергающих), что\dots
\item 
    Проведены вычислительные эксперименты\dots,
    которые подтвердили / опровергли / привели к~новым постановкам задач.\cite{zhuravlev99pria-eng}
\end{itemize}
    
%Цель данного раздела: доказать квалификацию автора.  Даже беглого взгляда на заключение должно быть достаточно, чтобы стало ясно: автору удалось решить актуальную, трудную, ранее не~решённую задачу, предложенные автором решения обоснованы и~проверены.

%Иногда в~Заключении приводится список направлений дальнейших исследований.

\newpage

\renewcommand{\bibname}{Список литературы}
\addcontentsline{toc}{section}{\bibname}

\end{document}

\nocite{hastie09elements,bishop06pattern,zhuravlev06recognition,zhuravlev78prob33,shlezinger04ten,boucheron05theory}

\def\BibUrl#1.{}\def\BibAnnote#1.{}
%\def\BibUrl#1{\\{\footnotesize\tt\def~{\char126} http://#1}}
\bibliographystyle{gost71s}
%\bibliography{MachLearn}

\begin{thebibliography}{99}


$[12]$ {\it Соколов П. Л.} Об оптимальной расшифровке монотонных функций алгебры логики // Ж. вычисл. матем. и матем. физ. 1982. Т. 22. № 2. С. 499-461.

\bibitem{1}
{\it Журавлев Ю.\,И., Коган А.\,Ю.} Реализация булевых функций с малым числом нулей дизъюнктивными нормальными формами и смежные задачи // Докл. АН СССР. 1985. Т. 285. No 4. С. 795-799.

\bibitem{2}
{\it Журавлев Ю.\,И., Коган А.\,Ю.} Алгоритм построения дизъюнктивной нормальной формы, эквивалент­ ной произведению левых частей булевых уравнений нельсоновского типа // Ж. вычисл. матем. и матем. физ. 1986. Т. 26. No 8. С. 1243-1249.

\bibitem{3}
{\it Соколов П.\,Л.} Об оптимальной расшифровке монотонных функций алгебры логики // Ж. вычисл. ма­тем. и матем. физ. 1982. Т. 22. No 2. С. 449-461.

\bibitem{4}
{\it Дьяконов А.\,Г.} Реализация одного класса булевых функций с малым числом нулей тупиковыми дизъюнктивными нормальными формами, Ж. вычисл. матем. и матем. физ., 2001, том 41, номер 5, 821–828

\end{thebibliography}

\end{document}
