% !TEX encoding = UTF-8 Unicode
%-----------------------------------------------------------------------------------------------------------------------------------------------------------
% Description of CC Drell-Yan in SANC
% A. Arbuzov
% W-mass workshop, Milano 17-18 March 2009
%-----------------------------------------------------------------------------------------------------------------------------------------------------------
\documentclass{beamer}
%\usepackage[cp866]{inputenc}
\usepackage[utf8]{inputenc}
%\usepackage[koi8-r]{inputenc}
%\usepackage[russian]{babel}
\usepackage[russian]{babel}
\usepackage{pgf,pgfarrows,pgfnodes,pgfautomata,pgfheaps,pgfshade}
\usepackage{beamerthemesplit}
\usepackage{rotate}
\usepackage{verbatim}
%\usepackage[euler]{textgreek}
\usepackage{bm}
\usepackage{amssymb,amsmath,amscd,epsfig}
%\usepackage[dvips]{color}
\usepackage{lscape}
\usepackage{epsfig}
\usepackage{graphicx}
\usepackage{color}
\usepackage{caption}
%\usepackage{axodraw}
\usepackage{amsmath}
\usepackage{amssymb}
\usepackage{verbatim}
\usepackage{multirow}
\usepackage{array}
\usepackage{graphicx}
\usepackage{booktabs}
%\usetheme{Rochester}
\setbeamercolor{background canvas}{bg=green!0}
\definecolor{colorone}{rgb}{0.0,0.35,0.35}
\definecolor{colortwo}{rgb}{0.0,0.0,0.35}
\definecolor{colorthree}{rgb}{0.85,0.0,0.0}
\definecolor{colorfour}{rgb}{0.05,0.56,0.05}
\definecolor{colorfive}{rgb}{1.0,0.18,0.92}
\definecolor{colorsix}{rgb}{0.5,0.188,0.0}
\newcommand{\mblue}{\color{colorone}}
\newcommand{\darkblue}{\color{colortwo}}
\newcommand{\red}{\color{colorthree}}
\newcommand{\darkgreen}{\color{colorfour}}
\newcommand{\magenda}{\color{colorfive}}
\newcommand{\darkbrown}{\color{colorsix}}
\newcommand{\myfont}{}
\newcommand{\be}{\begin{eqnarray}}
\newcommand{\ee}{\end{eqnarray}}
\newcommand{\bi}{\bibitem}
\newcommand{\lar}{\leftarrow}
\newcommand{\rar}{\rightarrow}
\newcommand{\lrar}{\leftrightarrow}
\newcommand{\mplq}{m_{Pl}^2}
\newcommand{\mnu}{m_\nu}
\newcommand{\nnu}{n_\nu}
\newcommand{\ngam}{n_\gamma}
\newcommand{\rhoij}{\rho_{ij}}
\newcommand{\dm}{\delta m^2}
\newcommand{\tc}{\textcolor}
\newcommand{\tcblue}{\textcolor{blue}}
\newcommand{\tcgreen}{\textcolor{green}}
\newcommand{\tcred}{\textcolor{red}}
\newcommand{\vs}{\vspace}
\newcommand{\nin}{\noindent}
\newcommand{\nonu}{\nonumber}
\newcommand{\rv}{\rho_{vac}}
\newcommand{\rc}{\rho_{c}}
\newcommand{\npni}{\newpage\noindent}
\def\mc{\mathcal}
%\def\be{\begin{equation}}
%\def\ee{\end{equation}}
\def\mpl{m_{Pl}}
\def\mn{{\mu\nu}}
\def\D{\mc D}
\def\-g{\sqrt{-g}}
%\renewcommand\rho{\varrho}
%\newcommand{\npg}{\newpage}
%%\definecolor{gold}{rgb}{0.89,0.78,0}
%%\definecolor{grn05}{rgb}{0,0.5,0}
%%\newcommand{\tcgold}{\textcolor{gold}}
\def\Ob{\Omega_b}
\def\Oc{\Omega_{\rm cdm}}
\def\Ok{\Omega_k}
\def\Ol{\Omega_\Lambda}
\def\Om{\Omega_m}
\def\etal{{\frenchspacing\it et al.}}
\def\mpl{m_{Pl}}
\setbeamertemplate{navigation symbols}{}
\setbeamertemplate{caption}[numbered]
%\setbeamerfont{page number in head/foot}{size=\normalsize}
\setbeamerfont{page number in head/foot}{size=\small}
\setbeamertemplate{footline}[frame number]

\newcolumntype{x}[1]{>{\centering\hspace{0pt}}p{#1}}

\newenvironment{lbmatrix}[1]
  {\left[\array{@{}*{#1}{c}@{}}}
  {\endarray\right]}

\newcommand\myeq{\mathrel{\stackrel{\makebox[0pt]{\mbox{\normalfont\tiny ?}}}{\equiv}}}

%\preauthor{\begin{flushright}\large \lineskip 0.5em}
%\postauthor{\par\end{flushright}}

\title[]
{\LARGE \myfont Построение дизъюнктивных нормальных форм для специальных классов булевых функций}
%\author[shortname]{author1 \inst{1} \and author2 \inst{2}}
\author[Д.\,А. Арбузова]
{\large \myfont {\it Автор}: студент 517 группы\\ {\tcblue {Д.\,А. Арбузова}}\\
{\it Научный руководитель}: профессор, академик РАН\\ {\tcblue {Ю.\,И. Журавлёв}}}
%{\Large \myfont Научный руководитель:{\tcblue {Д. А. Арбузова}}}
\institute[ВМК МГУ] % (optional)
{\large ВМК МГУ}
\date{\large %\myfont \large ВМК МГУ\\
{\tcblue {6 мая 2015}}}
\begin{document}

%---------------------------------------------------------------------------------------------------------------------------------------------------------01
\begin{frame}[plain]
  \titlepage
\end{frame}
%---------------------------------------------------------------------------------------------------------------------------------------------------------02
\frame{
\frametitle {\LARGE \myfont Постановка задачи}
%\Large
\myfont
Пусть булева функция $f(\tilde x^n)$ задана набором $k$ своих нулей
\setlength\abovedisplayskip{5pt}
\setlength\belowdisplayskip{5pt}
	$$(\alpha_{i1}, \ldots, \alpha_{in}),\,\, i = \overline{1,\,k}.$$
Требуется построить ДНФ функции $f$ по возможности меньшей сложности.\vspace{10pt}\\
Совершенная конъюнктивная нормальная форма:
\setlength\abovedisplayskip{0pt}
\setlength\belowdisplayskip{0pt}
	$$
	f(\tilde x^n) = \bigwedge_{i = 1}^k (x_1^{\alpha_{i1}} \vee \ldots \vee x_n^{\alpha_{in}}).
	$$
%
Число образованных конъюнкций до поглощения: $n^k$.\vspace{10pt}\\
Существующие подходы:
	\begin{itemize}
		\item
		метод Журавлёва\,--\,Когана;
		\item
		метод Дьяконова.
	\end{itemize}
}

\frame{
\frametitle {\LARGE \myfont Схема синтеза ДНФ}
\LARGE \myfont
	\begin{enumerate}
		\item
		Построение ДНФ заданной функции, возможно, достаточно большой сложности\vspace{7pt}
		\item
		Получение по ней тупиковой ДНФ
		\begin{enumerate}
			\Large
			\item
			Порядок просматривания конъюнкций
			\item
			Проверка поглощения
		\end{enumerate}
	\end{enumerate}
}

\frame{
\frametitle {\LARGE \myfont Метод разбиения на полосы}
%\Large
 \myfont
Пусть задана матрица нулей $M$ функции $f.$ Разобьём множество строк $M$ на полосы:
\setlength\abovedisplayskip{0pt}
\setlength\belowdisplayskip{3pt}
$$
	M = \begin{pmatrix}%{1}
			M_1\\ \hline
			M_2\\ \hline
			\vdots\\ \hline
			M_r
		\end{pmatrix}
$$
Для каждой из подматриц $M_i, i=\overline{1,\,r},$ построим ДНФ $\mathcal{D}_i,$ соответствующую функции с такой матрицей нулей.\vspace{10pt}\\
Исходная функция $f$ представима в виде
\setlength\abovedisplayskip{0pt}
\setlength\belowdisplayskip{0pt}
		$$
			f = \bigwedge_{i = 1}^r \mathcal{D}_i,
		$$
		и её ДНФ $\mathcal{D}_f$ может быть получена раскрытием скобок в данном выражении и применением правила поглощения.\par
		% Можно выбирать r априори, в совокупности с методом Дьяконова можно предложить разбиение по выбору наибольшей подматрицы нулей.
}

\frame{
\frametitle {\LARGE \myfont Проверка факта поглощения конъюнкции}
%\LARGE
 \myfont
	Пусть задана ДНФ $\mathcal{D} = \bigvee\limits_{i}K_i$ и конъюнкция $K.$ Требуется определить, поглощается ли $K$ данной ДНФ.\vspace{10pt}\\
	Возможны 2 варианта:
	\begin{itemize}
		\item
		\textbf{Поточечная проверка покрытия интервала, соответствующего конъюнкции $K$}\\
		\quad Конъюнкции $K_i \in \mathcal{D}$ упорядочиваются, и их интервалы пересекаются с интервалом $K.$
		\item
		\textbf{Применение критерия поглощения}\\
		\quad Представим каждую конъюнкцию ДНФ $\mathcal{D}$ в виде $K_i = K^1_i \wedge K^2_i,$ где в $K^1_i$ входят все сомножители, общие с $K,$ в $K^2_i$ --- оставшиеся.\\
		\quad \textbf{\textit{Критерий поглощения.}}\\
		\quad \quad $N_K \subseteq \bigcup\limits_{i}N_{K_i} \Leftrightarrow \bigvee\limits_i K^2_i \equiv 1.$ 
	\end{itemize}
}

\frame{
\frametitle {\LARGE \myfont Критерий поглощения и алгоритм Блейка}
%\LARGE
 \myfont
 	В ходе применения критерия поглощения возникает NP-полная задача определить, является ли полученная ДНФ тождественной единицей: $\mathcal{D}^2 = \bigvee\limits_{i} K^2_i \myeq 1.$\vspace{10pt}\\
	В некоторых случаях эта проверка может быть достаточно эффективно выполнена с помощью \textbf{\textit{алгоритма Блейка}} построения сокращённой ДНФ.
	%\vspace{20pt}\\
\begin{enumerate}
		\item
		Выполняются все обобщённые склеивания: $xK_1 \vee \bar{x}K_2 = xK_1 \vee \bar{x}K_2 \vee K_1K_2.$
		\item
		Выполняются все элементарные поглощения: $K_1 \vee K_1K_2 = K_1.$
	\end{enumerate}\vspace{10pt}
	Возможны две альтернативы:
	\begin{enumerate}
		\item
		Сокращённая ДНФ равна 1 $\Rightarrow \mathcal{D}^2 \equiv 1 \Rightarrow$ поглощение есть.
		\item
		Сокращённая ДНФ отлична от 1  $\Rightarrow \mathcal{D}^2 \not\equiv 1 \Rightarrow$ поглощения нет.
	\end{enumerate}
}

% где написать правило Блейка и в чём заключается метод Блейка?

\frame{
\frametitle {\Large \myfont Оценки метода Блейка. Случай одной переменной}
	Общий вид ДНФ:
	$$
		\mathcal{D} = \bigvee_{i = 1}^{t_1}xA_i \vee \bigvee_{i = 1}^{t_2}\bar{x}B_i  \vee \bigvee_iC_i.
	$$
	Верхняя оценка числа новых конъюнкций в сокращённой ДНФ:
	$$
				f_{new}(t_1, t_2) = t_1t_2.
			$$
	Максимум при фиксированной длине ДНФ достигается при\\
	\centering
	$t_1 = t_2.$

}

\frame{
\frametitle {\Large \myfont Оценки метода Блейка. Случай двух переменных}
	Общий вид ДНФ:
	$$
		\mathcal{D} = \bigvee_{i = 1}^{t_1}x_1A_i \vee \bigvee_{i = 1}^{t_2}\bar{x}_1B_i \vee \bigvee_{i = 1}^{t_3}x_2C_i \vee \bigvee_{i = 1}^{t_4}\bar{x}_2D_i \vee$$

	$$ \vee  \bigvee_{i = 1}^{s_1}x_1x_2E_i \vee \bigvee_{i = 1}^{s_2}x_1\bar{x}_2F_i \vee \bigvee_{i = 1}^{s_3}\bar{x}_1x_2G_i \vee \bigvee_{i = 1}^{s_4}\bar{x}_1\bar{x}_2H_i \vee \bigvee_iI_i.
	$$
	Верхняя оценка числа новых конъюнкций в сокращённой ДНФ:
	$$
				f_{new}(\bar{t}, \bar{s}) = s_1s_2 + s_1s_3 + s_2s_4 + s_3s_4 + s_1s_2s_3s_4 +
			$$
			$$ 
				+ t_1(s_3 + s_4 + s_3s_4) + t_2(s_1 + s_2 + s_1s_2) + t_3(s_2 + s_4 + s_2s_4) + t_4(s_1 + s_3 + s_1s_3) + 
			$$
			$$
				+ t_1t_2 + t_3t_4 + t_1t_3s_4 + t_1t_4s_3 + t_2t_3s_2 + t_2t_4s_1.
			$$
	Максимум при фиксированной длине ДНФ достигается при\\
	\centering
	$t_1 = t_2 = t_3 = t_4 = 0,\quad s_1 = s_2 = s_3 = s_4.$
}

\frame{
\frametitle {\Large \myfont Оценки метода Блейка. Случай двух переменных}
Верхняя оценка числа конъюнкций, образованных на 1-м этапе метода Блейка:
$$
f(\overline{t}, \overline{s}) = t_1\big(2((2^{s_2} - 1)(2^{s_3} - 1) + (2^{s_1} - 1)(2^{s_4} - 1) +$$ $$ + (2^{s_1} - 1)s_3(2^{s_4} - 1) + (2^{s_2} - 1)(2^{s_3} - 1)s_4) + s_3 + s_4 + s_3s_4 \big) + 
			$$
			$$
				+ t_2\big(2(s_1(2^{s_2} - 1)(2^{s_3} - 1) + (2^{s_1} - 1)s_2(2^{s_4} - 1) +$$ $$ + (2^{s_1} - 1)(2^{s_4} - 1) + (2^{s_2} - 1)(2^{s_3} - 1)) + s_1 + s_2 + s_1s_2 \big) +
			$$
			$$
				+ t_3\big(2((2^{s_2} - 1)(2^{s_3} - 1) + (2^{s_1} - 1)s_2(2^{s_4} - 1) + $$ $$ + (2^{s_1} - 1)(2^{s_4} - 1) + (2^{s_2} - 1)(2^{s_3} - 1)s_4) + s_2 + s_4 + s_2s_4\big) +
			$$
			$$
				+ t_4\big(2(s_1(2^{s_2} - 1)(2^{s_3} - 1) + (2^{s_1} - 1)(2^{s_4} - 1) + $$ $$ + (2^{s_1} - 1)s_3(2^{s_4} - 1) + (2^{s_2} - 1)(2^{s_3} - 1)) + s_1 + s_3 + s_1s_3\big) +
			$$
			$$
			+ s_1s_2 + s_1(2^{s_2} - 1)(2^{s_3} - 1) + (2^{s_1} - 1)s_2(2^{s_4} - 1) + $$ $$ +s_3s_4 + (2^{s_1} - 1)s_3(2^{s_4} - 1) + (2^{s_2} - 1)(2^{s_3} - 1)s_4 +
			$$
			$$
			+s_1s_3 + s_1(2^{s_2} - 1)(2^{s_3} - 1) + (2^{s_1} - 1)s_3(2^{s_4} - 1) + $$ $$ +s_2s_4 + (2^{s_1} - 1)s_2(2^{s_4} - 1) + (2^{s_2} - 1)(2^{s_3} - 1)s_4 +
			$$
}

\frame{
\frametitle {\Large \myfont Оценки метода Блейка. Случай двух переменных}
\setlength\abovedisplayskip{0pt}
\setlength\belowdisplayskip{5pt}
			$$
				+ 4(t_1 + t_2 + t_3 + t_4 + 1)(2^{s_1} - 1)(2^{s_2} - 1)(2^{s_3} - 1)(2^{s_4} - 1) +
			$$
			$$
			\left(C_{t_1 + t_2 + t_3 + t_4 + 1}^2 + 1\right)(2^{s_1} - 1)(2^{s_2} - 1)(2^{s_3} - 1)(2^{s_4} - 1) + 
			$$
			$$
				+ t_1t_2 + t_3t_4 + t_1t_3s_4 + t_2t_4s_3 + t_2t_3s_2 + t_2t_4s_1 +
			$$
			$$
				+ C_{t_1 + 1}^2 s_3s_4 + C_{t_2 + 1}^2s_1s_2 + C_{t_3 + 1}^2s_2s_4 +C_{t_4 + 1}^2s_1s_3 +
			$$
			$$
				+ (2^{s_2} - 1)(2^{s_3} - 1) \left (C_{t_1 + 1}^2 s_4 + t_1t_2 + t_1t_3s_4 + t_1t_4 +C_{t_2 + 1}^2 s_1 + \right.$$ $$ \left.+ t_2t_3 + t_2t_4s_1 + C_{t_3 + 1}^2 s_4 + t_3t_4 + C_{t_4 + 1}^2 s_1\right) +
			$$
			$$
				+ (2^{s_1} - 1)(2^{s_4} - 1) \left (C_{t_1 + 1}^2 s_3 + t_1t_2 + t_1t_3 + t_1t_4s_3 + C_{t_2 + 1}^2 s_2 + \right.$$ $$ \left.+ t_2t_3s_2 + t_2t_4 + C_{t_3 + 1}^2 s_2 + t_3t_4 + C_{t_4 + 1}^2 s_3\right)\!.
			$$
}


\frame{
\frametitle {\LARGE \myfont Оценки метода Блейка}\Large \myfont
\vspace*{-5mm}
	\begin{table}
		\centering
		\begin{tabular}{|>{\centering\arraybackslash}m{2in}|>{\centering\arraybackslash}m{0.9in}|>{\centering\arraybackslash}m{0.9in}|}
			\hline
			\multirow{2}{*}{} & \multicolumn{2}{>{\centering\arraybackslash}m{2.0in}|}{Число переменных, встречающихся в ДНФ в обеих степенях}\tabularnewline
			\cline{2-3}
			& 1 & 2 \tabularnewline \hline
			Оценка числа образованных конъюнкций в ходе метода Блейка & $\underline{O}(r^2)$ & $\underline{O}(2^r)$\tabularnewline \hline
			Оценка числа новых конъюнкций в сокращённой ДНФ & $\underline{O}(r^2)$ & $\underline{O}(r^4)$\tabularnewline \hline
		\end{tabular}
	\end{table}
	\normalsize $r$ --- длина исходной ДНФ
}
% показывает экспоненциальный рост даже при изначально небольшом количестве применений правила Блейка

\frame{
\frametitle {\LARGE \myfont Поиск нуля функции}
\Large \myfont
	Пусть функция $f$ задана своей ДНФ. Требуется найти хотя бы один ноль этой функции.\vspace{5pt}\\
	\Large
	\begin{enumerate}
		\item
		Подсчёт встречаемости каждой буквы в конъюнкциях ДНФ.
		\item
		Упорядочение букв согласно некоторому функционалу
		\begin{itemize}
		\item
		по частоте встречаемости;
		\item
		весовая схема, основанная на встречаемости буквы и ранге содержащей её конъюнкции.
		\end{itemize}
		\item
		Последовательное обнуление букв и соответствующих конъюнкций.
	\end{enumerate}\vspace{5pt}
	Если в результате все конъюнкции оказались обнулены, то ноль функции найден.
}

\frame{
\frametitle {\LARGE \myfont Результаты}
%\Large 
\myfont
	В ходе исследования были получены следующие основные результаты:
	\begin{itemize}
		\item
		Предложен более совершенный метод построения тупиковой ДНФ, основанный на разбиении матрицы нулей на полосы;
		\item
		Описаны некоторые способы проверки факта поглощения конъюнкции;
		\item
		Для конкретных случаев приведена верхняя оценка на число конъюнкций, образующихся в ходе метода Блейка и оставшихся в сокращённой ДНФ;
		\item
		Найдены отдельные случаи, когда критерий поглощения с методом Блейка эффективнее поточечной проверки покрытия интервала конъюнкции;
		\item
		Предложено несколько вариантов алгоритма поиска нуля булевой функции, заданной своей ДНФ.
	\end{itemize}
}
\end{document}

\frame{
\frametitle {\LARGE \myfont Оценки метода Блейка}
\Large \myfont
	\begin{table}
		\centering
		\begin{tabular}{|>{\centering\arraybackslash}m{2in}|>{\centering\arraybackslash}m{0.9in}|>{\centering\arraybackslash}m{0.9in}|}
			\hline
			\multirow{2}{*}{} & \multicolumn{2}{>{\centering\arraybackslash}m{2.0in}|}{Число переменных, встречающихся в ДНФ в обеих степенях}\tabularnewline
			\cline{2-3}
			& 1 & 2 \tabularnewline \hline
			Оценка числа образованных конъюнкций в ходе метода Блейка & $\underline{O}(r^2)$ & $\underline{O}(2^r)$\tabularnewline \hline
			Оценка числа новых конъюнкций в сокращённой ДНФ & $\underline{O}(r^2)$ & $\underline{O}(r^4)$\tabularnewline \hline
		\end{tabular}
	\end{table}
}
