% !TEX encoding = UTF-8 Unicode
%-----------------------------------------------------------------------------------------------------------------------------------------------------------
% Description of CC Drell-Yan in SANC
% A. Arbuzov
% W-mass workshop, Milano 17-18 March 2009
%-----------------------------------------------------------------------------------------------------------------------------------------------------------
\documentclass{beamer}
%\usepackage[cp866]{inputenc}
\usepackage[utf8]{inputenc}
%\usepackage[koi8-r]{inputenc}
%\usepackage[russian]{babel}
\usepackage[russian]{babel}
\usepackage{pgf,pgfarrows,pgfnodes,pgfautomata,pgfheaps,pgfshade}
\usepackage{beamerthemesplit}
\usepackage{rotate}
\usepackage{verbatim}
%\usepackage[euler]{textgreek}
\usepackage{bm}
\usepackage{amssymb,amsmath,amscd,epsfig}
%\usepackage[dvips]{color}
\usepackage{lscape}
\usepackage{epsfig}
\usepackage{graphicx}
\usepackage{color}
\usepackage{caption}
%\usepackage{axodraw}
\usepackage{amsmath}
\usepackage{amssymb}
\usepackage{verbatim}
\usepackage{multirow}
\usepackage{array}
\usepackage{graphicx}
%\usetheme{Rochester}
\setbeamercolor{background canvas}{bg=green!0}
\definecolor{colorone}{rgb}{0.0,0.35,0.35}
\definecolor{colortwo}{rgb}{0.0,0.0,0.35}
\definecolor{colorthree}{rgb}{0.85,0.0,0.0}
\definecolor{colorfour}{rgb}{0.05,0.56,0.05}
\definecolor{colorfive}{rgb}{1.0,0.18,0.92}
\definecolor{colorsix}{rgb}{0.5,0.188,0.0}
\newcommand{\mblue}{\color{colorone}}
\newcommand{\darkblue}{\color{colortwo}}
\newcommand{\red}{\color{colorthree}}
\newcommand{\darkgreen}{\color{colorfour}}
\newcommand{\magenda}{\color{colorfive}}
\newcommand{\darkbrown}{\color{colorsix}}
\newcommand{\myfont}{}
\newcommand{\be}{\begin{eqnarray}}
\newcommand{\ee}{\end{eqnarray}}
\newcommand{\bi}{\bibitem}
\newcommand{\lar}{\leftarrow}
\newcommand{\rar}{\rightarrow}
\newcommand{\lrar}{\leftrightarrow}
\newcommand{\mplq}{m_{Pl}^2}
\newcommand{\mnu}{m_\nu}
\newcommand{\nnu}{n_\nu}
\newcommand{\ngam}{n_\gamma}
\newcommand{\rhoij}{\rho_{ij}}
\newcommand{\dm}{\delta m^2}
\newcommand{\tc}{\textcolor}
\newcommand{\tcblue}{\textcolor{blue}}
\newcommand{\tcgreen}{\textcolor{green}}
\newcommand{\tcred}{\textcolor{red}}
\newcommand{\vs}{\vspace}
\newcommand{\nin}{\noindent}
\newcommand{\nonu}{\nonumber}
\newcommand{\rv}{\rho_{vac}}
\newcommand{\rc}{\rho_{c}}
\newcommand{\npni}{\newpage\noindent}
\def\mc{\mathcal}
%\def\be{\begin{equation}}
%\def\ee{\end{equation}}
\def\mpl{m_{Pl}}
\def\mn{{\mu\nu}}
\def\D{\mc D}
\def\-g{\sqrt{-g}}
%\renewcommand\rho{\varrho}
%\newcommand{\npg}{\newpage}
%%\definecolor{gold}{rgb}{0.89,0.78,0}
%%\definecolor{grn05}{rgb}{0,0.5,0}
%%\newcommand{\tcgold}{\textcolor{gold}}
\def\Ob{\Omega_b}
\def\Oc{\Omega_{\rm cdm}}
\def\Ok{\Omega_k}
\def\Ol{\Omega_\Lambda}
\def\Om{\Omega_m}
\def\etal{{\frenchspacing\it et al.}}
\def\mpl{m_{Pl}}
\setbeamertemplate{navigation symbols}{}
\setbeamertemplate{caption}[numbered]
%\setbeamerfont{page number in head/foot}{size=\normalsize}
\setbeamerfont{page number in head/foot}{size=\small}
\setbeamertemplate{footline}[frame number]

\newcolumntype{x}[1]{>{\centering\hspace{0pt}}p{#1}}

%\preauthor{\begin{flushright}\large \lineskip 0.5em}
%\postauthor{\par\end{flushright}}

\title[]
{\LARGE \myfont Построение дизъюнктивных нормальных форм для специальных классов булевых функций}
%\author[shortname]{author1 \inst{1} \and author2 \inst{2}}
\author[Д.\,А. Арбузова]
{\large \myfont {\it Автор}: студент 517 группы\\ {\tcblue {Д.\,А. Арбузова}}\\
{\it Научный руководитель}: д.\,ф.-м.\,н., академик РАН\\ {\tcblue {Ю.\,И. Журавлёв}}}
%{\Large \myfont Научный руководитель:{\tcblue {Д. А. Арбузова}}}
\institute[ВМК МГУ] % (optional)
{\large ВМК МГУ}
\date{\large %\myfont \large ВМК МГУ\\
{\tcblue {6 мая 2015}}}
\begin{document}

%---------------------------------------------------------------------------------------------------------------------------------------------------------01
\begin{frame}[plain]
  \titlepage
\end{frame}
%---------------------------------------------------------------------------------------------------------------------------------------------------------02
\frame{
\frametitle {\LARGE \myfont Постановка задачи}
\Large \myfont
Пусть булева функция $f(\tilde x^n)$ задана набором $k$ своих нулей
	$$(\alpha_{i1}, \ldots, \alpha_{in}),\,\, i = \overline{1,\,k}.$$
Требуется построить ДНФ функции $f$ по возможности меньшей сложности.
%про существующие подходы: метод Журавлёва-Когана, алгоритм Дьяконова
%\vspace{10pt}\\
%Совершенная конъюнктивная форма:
%	$$
%	f(\tilde x^n) = \bigwedge_{i = 1}^k (x_1^{\alpha_{i1}} \vee \ldots \vee %x_n^{\alpha_{in}}).
%	$$
%Число образованных конъюнкций : $n^k$.
}

\frame{
\frametitle {\LARGE \myfont Схема синтеза ДНФ}
\LARGE \myfont
	\begin{enumerate}
		\item
		Построение ДНФ заданной функции, возможно, достаточно большой сложности\vspace{7pt}
		\item
		Получение по ней тупиковой ДНФ
		\begin{enumerate}
			\Large
			\item
			Порядок просматривания конъюнкций
			\item
			Проверка поглощения
		\end{enumerate}
	\end{enumerate}
}

\frame{
\frametitle {\LARGE \myfont Метод разбиения на полосы}
\Large \myfont
Пусть задана матрица нулей $M$ функции $f.$ Разобьём множество строк $M$ на полосы:
$$
	M = \begin{bmatrix}
			M_1\\ \hline
			M_2\\ \hline
			\vdots\\ \hline
			M_r
		\end{bmatrix}
$$
Для каждой полосы в отдельности строится соответствующая ей ДНФ $\mathcal{D}_i,$ и впоследствии они перемножаются.
}

\frame{
\frametitle {\LARGE \myfont Проверка факта поглощения конъюнкции}
\LARGE \myfont
	Задана ДНФ $\mathcal{D}$ и конъюнкция $K.$ Требуется определить, поглощается ли $K$ данной ДНФ.
	\Large
	\begin{itemize}
		\item
		Поточечная проверка покрытия интервала, соответствующего конъюнкции $K$
		\item
		Применение критерия поглощения
	\end{itemize}
}

\frame{
\frametitle {\LARGE \myfont Критерий поглощения и алгоритм Блейка}
\LARGE \myfont
	Критерий поглощения: $\mathcal{D} \equiv 1$?\vspace{20pt}\\
	Используем метод Блейка для получения сокращённой ДНФ.
}

\frame{
\frametitle {\LARGE \myfont Оценки метода Блейка}
\Large \myfont
	\begin{table}
		\centering
		\begin{tabular}{|>{\centering\arraybackslash}m{2in}|>{\centering\arraybackslash}m{0.9in}|>{\centering\arraybackslash}m{0.9in}|}
			\hline
			\multirow{2}{*}{} & \multicolumn{2}{>{\centering\arraybackslash}m{2.0in}|}{Число переменных, встречающихся в ДНФ в обеих степенях}\tabularnewline
			\cline{2-3}
			& 1 & 2 \tabularnewline \hline
			Оценка числа образованных конъюнкций в ходе метода Блейка & $\underline{O}(r^2)$ & $\underline{O}(2^r)$\tabularnewline \hline
			Оценка числа новых конъюнкций в сокращённой ДНФ & $\underline{O}(r^2)$ & $\underline{O}(r^4)$\tabularnewline \hline
		\end{tabular}
	\end{table}
}

\frame{
\frametitle {\LARGE \myfont Поиск нуля функции}
\LARGE \myfont
	Пусть функция $f$ задана своей ДНФ. Требуется найти хотя бы один ноль этой функции.\vspace{10pt}\\
	\Large
	\begin{enumerate}
		\item
		Подсчёт встречаемости каждой буквы в конъюнкциях ДНФ
		\item
		Упорядочение букв согласно некоторому функционалу
		\item
		Последовательное обнуление
	\end{enumerate}
}

\frame{
\frametitle {\LARGE \myfont Результаты}
\Large \myfont
	\begin{itemize}
		\item
		Предложен метод построения тупиковой ДНФ, основанный на разбиении матрицы нулей на полосы;
		\item
		Для конкретных случаев приведена верхняя оценка на число конъюнкций, образующихся в ходе метода Блейка и оставшихся в сокращённой ДНФ;
		\item
		Найдены отдельные случаи, когда критерий поглощения с методом Блейка эффективнее поточечной проверки покрытия интервала конъюнкции.
	\end{itemize}
}
\end{document}